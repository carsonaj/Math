
\documentclass[12pt]{amsart}
 \usepackage[margin=1in]{geometry} 
\usepackage{amsmath,amsthm,amssymb,amsfonts,setspace}
\usepackage[shortlabels]{enumitem}

\newtheorem{thm}{Theorem}[section]
\newtheorem{lem}[thm]{Lemma}
\newtheorem{prop}[thm]{Proposition}
\newtheorem{cor}[thm]{Corollary}
\newtheorem{conj}{Conjecture}
\newtheorem{defn}[thm]{Definition}
\newtheorem{note}[thm]{Note}
\newtheorem{ex}[thm]{Exercise}


\newcommand{\al}{\alpha}
\newcommand{\be}{\beta} 
\newcommand{\del}{\delta} 
\newcommand{\Del}{\Delta}
\newcommand{\lam}{\lambda}  
\newcommand{\Lam}{\Lambda} 
\newcommand{\ep}{\epsilon}
\newcommand{\sig}{\sigma} 
\newcommand{\om}{\omega}
\newcommand{\Om}{\Omega}
\newcommand{\C}{\mathbb{C}}
\newcommand{\N}{\mathbb{N}}
\newcommand{\E}{\mathbb{E}}
\newcommand{\Z}{\mathbb{Z}}
\newcommand{\R}{\mathbb{R}}
\newcommand{\Q}{\mathbb{Q}}
\renewcommand{\P}{\mathbb{P}}
\newcommand{\MA}{\mathcal{A}}
\newcommand{\MB}{\mathcal{B}}
\newcommand{\MF}{\mathcal{F}}
\newcommand{\MG}{\mathcal{G}}
\newcommand{\ML}{\mathcal{L}}
\newcommand{\MN}{\mathcal{N}}
\newcommand{\MS}{\mathcal{S}}
\newcommand{\MP}{\mathcal{P}}
\newcommand{\ME}{\mathcal{E}}
\newcommand{\MT}{\mathcal{T}}
\newcommand{\MM}{\mathcal{M}}

\newcommand{\RG}{[0,\infty]}
\newcommand{\Rg}{[0,\infty)}
\newcommand{\limfn}{\liminf \limits_{n \rightarrow \infty}}
\newcommand{\limpn}{\limsup \limits_{n \rightarrow \infty}}
\newcommand{\limn}{\lim \limits_{n \rightarrow \infty}}
\newcommand{\convt}[1]{\xrightarrow{\text{#1}}}
\newcommand{\conv}[1]{\xrightarrow{#1}} 

\newcommand{\Ll}{L^1_{\text{loc}}(\R^n)}
 
\begin{document}

\title{Introduction to Algebraic Number Theory}
\maketitle

\tableofcontents

\section{Algebraic Integers}

In the following section, $K$ is taken to be a number field and thus a subfield of $\overline{\Q}$

\begin{defn}
Let $\al \in K$. Then $\al$ is said to be an \textbf{algebraic integer} if there exists $f(x) \in \Z[x]$ such that $f(x)$ is monic and $p(\al) = 0$. Define $O_K = \{\al \in K: \al \text{ is an algebraic integer}\}$.
\end{defn}

\begin{thm}
Let $\al \in K$. Then $\al$ is an algebraic integer iff $m_{\al, \Q}(x) \in \Z[x]$.
\end{thm}

\begin{proof}
If $m_{\al, \Q}(x) \in \Z[x]$, then clearly $\al$ is an algebraic integer.\vspace{2mm}

Conversely, suppose that $\al$ is an algebraic integer. There exists $f(x) \in \Z[x]$ such that $f(x)$ is monic and $f(\al)=0$. Since $\Z[x]$ is a unique factorization domain and $f(x)$ is not a unit and nonzero, there exist irreducible polynomials $(p_i(x))_{i=1}^n \subset \Z[x]$ such that $f(x) = \prod\limits_{i=1}^np_i(x)$. Since $f(x)$ is monic, for each $i \in \{1,2,\cdots,n\}$, we may take $p_i(x)$ to be monic. Then there exists $k \in \{1,2,\cdots,n\}$ such that $p_k(\al)=0$. Then $m_{\al, \Q}(x)|p_k(x)$ in $\Q[x]$. Thus $p_k(x)=m_{\al, \Q}(x)$. Since $p_k(x)$ is monic and irreducible in $\Z[x]$, it is irreducible in $\Q[x]$. Thus $m_{\al, \Q}(x)=p_k(x)$. 
\end{proof} 

\begin{lem}
Let $M$ be a finitely generated $\Z$-submodule of $K$. Then $M$ is free. 
\end{lem}

\begin{proof}
Since $M$ is finitely genereated and torsion-free, the fundamental theorem of finitetly generated abelian groups shows that $M$ is free.
\end{proof}

\begin{note}
The previous result says that anytime we consider $M$, a finitely generated $\Z$-submodule of $K$, we may choose a basis for $M$. 
\end{note}

\begin{thm}
Let $\al \in K$. Then $\al \in O_K$ iff there exists a finitely generated $\Z$-submodule $M$ of $K$ such that $\al M \subset M$. 
\end{thm}

\begin{proof}
Suppose that $\al \in O_K$. Then there exist $(a_{i})_{i=0}^{n-1} \subset \Z$ such that such that $\al^n + a_{n-1}\al^{n-1}+ \cdots+a_1\al + a_0 = 0$. Then $M = (1, \al, \al^2, \cdots, \al^{n-1})$ is a finitely generated $\Z$-submodule of $K$ and $\al M \subset M$.

Conversely, Suppose that there exists a finitely generated $\Z$-submodule $M$ of $K$ such that $\al M \subset M$. Choose a basis $a=\{\al_1, \al_2, \cdots, \al_n\}$ of $M$. Thus for each $i,j \in \{1,2,\cdots, n\}$, there exists $a_{i,j} \in \Z$ such that $\al \al_j = \sum\limits_{i=1}^na_{i,j}\al_i$. Define $T:M \rightarrow M$ by $T(x) = \al x$. Then $T$ is a linear with matrix representation $[T]_a = (a_{i,j})$ and eigen-value $\al$. Thus $f(x) = \det(xI-T) \in \Z$ is a monic polynomial with root $\al$. So $\al \in O_K$.   
\end{proof}

\begin{thm}
Let $\al, \beta \in O_K$. Then $\al + \beta \in O_K$ and $\al \beta \in O_K$.
\end{thm}

\begin{proof}
Since $\al,\beta \in O_K$, there exist finitely generated $\Z$-submodules $M$ and $N$ of $K$ such that $\al M \subset M$ and $\beta N \subset N$. Choose finite sets $X,Y \subset K$ such that $M = (X)$ and $N=(Y)$. Then $MN = (XY)$ is finitely generated. Let $x \in X$ and $y \in Y$. Then $(\al+\beta)(xy) = (\al x)y+x(\beta y)$ and $(\al \beta)(xy) = (\al x)(\beta y)$. Since $\al x \in M$ and $\beta y \in N$, we have that $(\al+\beta)(xy) \in MN$ and $(\al \beta)(xy) \in MN$. Hence $(\al + \beta)MN \subset MN$, $(\al \beta)MN \subset MN$ and thus $\al+\beta, \al \beta \in O_K$
\end{proof}

\begin{cor}
We have that $O_K$ is a ring.
\end{cor}

\begin{lem}
Let $\al \in O_K$, $(\al_i)_{i=1}$ the conjugates of $\al$, $L = \Q(\al_1, \al_2, \cdots, \al_n)$ and $\sig: K \hookrightarrow \overline{\Q}$ an embedding. Then $\sig(\al) \in O_L$
\end{lem}

\begin{proof}
Since $\al \in O_L$, there exists $f(x) \in \Z[x]$ such that $f(x)$ is monic and $f(\al)=0$. Since $\sig$ permutes $(\al_i)_{i=1}$, $\sig(\al) \in L$. Since $\sig$ fixes $\Q$ we haver that
\begin{align*}
f(\sig(\al)) 
&= \sig(f(\al))\\
&=0
\end{align*}

so $\sig(\al) \in O_L$.
\end{proof}

\begin{lem}
We have that $O_K \cap \Q = \Z$
\end{lem}

\begin{proof}
Clearly $\Z \subset O_K \cap \Q$. Let $\al \in O_K \cap \Q$. If $\al=0$, then $\al \in \Z$. Suppose that $\al \neq 0$. Since $\al \in \Q$, there exists $a,b \in \Z\setminus\{0\}$ such that $\gcd(a,b) =1$ and $\al = ab^{-1}$. Since $\al \in O_K$, there exist $a_0, a_1, \cdots, a_{n-1} \in \Z$ such that $\al^n + a_{n-1}\al^{n-1} + \cdots + a_1\al + a_0=0$. The rational root theorem says that $b|1$, so $b\in \Z^{\times}$ and thus $\al \in \Z$.
\end{proof}

\begin{lem}
Let $\al \in O_K$, $(\al_i)_{i=1}^n \subset \overline{\Q}$ the conjugates of $\al$ and $f(X_1, X_2,\cdots, X_n) \in \Z[X_1, X_2,\cdots, X_n]$ a symmetric polynomial. Then $f(\al_1, \al_2, \cdots, \al_n) \in \Z$.

\begin{proof}
Since $O_K$ is a ring, it is clear that $f(\al_1, \al_2, \cdots, \al_n) \in O_K$. Let $L = \Q(\al_1, \al_2, \cdots, \al_n)$. Since $O_L$ is a ring and for each embedding $\sig : K \hookrightarrow \overline{\Q}$ and $i \in \{1,2,\cdots,n\}$, $\sig(\al_i) \in O_L$, we know that for each embedding $\sig : K \hookrightarrow \overline{\Q}$, $\sig(f(\al_1, \al_2, \cdots, \al_n)) \in O_L$. For each embedding $\sig : K \hookrightarrow \overline{\Q}$, there exists $\tau_\sig \in S_n$ such that for each $i \in \{1,2,\cdots, n\}$, $\sig(\al_i) = \al_{\tau_{\sig}(i)}$. So for each embedding $\sig : K \hookrightarrow \overline{\Q}$, we have
\begin{align*}
\sig(f(\al_1, \al_2, \cdots, \al_n))
&= f(\sig(\al_1),\sig(\al_2), \cdots, \sig(\al_n)) \\
&= f(\al_{\tau_{\sig}(1)}, \al_{\tau_{\sig}(2)},\cdots, \al_{\tau_{\sig}(n)})\\
&= f(\al_1, \al_2, \cdots, \al_n) \\
\end{align*}

which implies that $f(\al_1, \al_2, \cdots, \al_n) \in \Q$. Since $\Q \cap O_L = \Z$, we have that $f(\al_1, \al_2, \cdots, \al_n) \in \Z$.
\end{proof}
\end{lem}

\begin{thm}
Let $\al \in K$. Then there exists $c \in \Z$ such that $c\al \in O_K$.
\end{thm}

\begin{proof}
Consider $m_{\al, \Q}(x) = x^n + a_{n-1}x^{n-1} + \cdots + a_0 \in \Q[x]$. For each $i \in \{1,2,\cdots,n-1\}$, there exist $b_i, c_i \in Z$ such that $c_i \neq 0$ and $a_i=b_ic_i^{-1}$. Define $c = \text{lcm} \{c_i:i=1,2,\cdots,n-1\} \in \Z$ and $f(x) = c^nm_{\al,\Q}(c^{-1}x) = x^n + a_{n-1}cx^{n-1} + \cdots+ a_1c^{n-1}x + a_0c^n \in \Z[x]$. Then $f(x)$ is monic and $f(c\al) =0$. So $c \al \in O_K$. 
\end{proof}

\begin{cor}
Let $K$ be a number field. Then there exists $\al \in O_K$ such that $K=\Q(\al)$. 
\end{cor}

\begin{proof}
Since $K$ is a finite extension of $\Q$, there exists $\theta \in K$ such that $K = \Q(\theta)$. Then the previous result tells us that there exists $c \in \Z$ such that $c\theta \in O_K$. Choose $\al = c\theta$. Then $K = \Q(\theta) = \Q(\al)$.
\end{proof}

\end{document}