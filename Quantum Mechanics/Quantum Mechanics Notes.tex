\documentclass[12pt]{amsart}
\usepackage[margin=1in]{geometry} 
\usepackage{amsmath,amsthm,amssymb,amsfonts,setspace}
\usepackage[shortlabels]{enumitem}
\usepackage{exercise, chngcntr}
\usepackage{cite}


\newtheorem{thm}{Theorem}[section]
\newtheorem{lem}[thm]{Lemma}
\newtheorem{prop}[thm]{Proposition}
\newtheorem{cor}[thm]{Corollary}
\newtheorem{conj}{Conjecture}
\newtheorem{defn}[thm]{Definition}
\newtheorem{intp}[thm]{Interpretation}
\newtheorem{note}[thm]{Note}
\newtheorem{ex}[thm]{Exercise}
\newtheorem{exm}[thm]{Example}
\newtheorem{sol}[thm]{Solution}

\renewcommand{\r}{\rangle}
\renewcommand{\l}{\langle}

\newcommand{\sch}{Schr\"{o}dinger }

\newcommand{\al}{\alpha}
\newcommand{\Gam}{\Gamma}
\newcommand{\gam}{\gamma}
\newcommand{\be}{\beta} 
\newcommand{\del}{\delta} 
\newcommand{\Del}{\Delta}
\newcommand{\lam}{\lambda}  
\newcommand{\Lam}{\Lambda} 
\newcommand{\ep}{\epsilon}
\newcommand{\sig}{\sigma} 
\newcommand{\om}{\omega}
\newcommand{\Om}{\Omega}
\newcommand{\C}{\mathbb{C}}
\newcommand{\N}{\mathbb{N}}
\renewcommand{\H}{\mathbb{H}}
\newcommand{\Z}{\mathbb{Z}}
\newcommand{\R}{\mathbb{R}}
\newcommand{\Q}{\mathbb{Q}}
\renewcommand{\P}{\mathbb{P}}
\newcommand{\MA}{\mathcal{A}}
\newcommand{\MB}{\mathcal{B}}
\newcommand{\MF}{\mathcal{F}}
\newcommand{\MG}{\mathcal{G}}
\newcommand{\ML}{\mathcal{L}}
\newcommand{\MN}{\mathcal{N}}
\newcommand{\MS}{\mathcal{S}}
\newcommand{\MP}{\mathcal{P}}
\newcommand{\ME}{\mathcal{E}}
\newcommand{\MT}{\mathcal{T}}
\newcommand{\MM}{\mathcal{M}}
\newcommand{\MW}{\mathcal{W}}
\renewcommand{\MR}{\mathcal{R}}


\newcommand{\RG}{[0,\infty]}
\newcommand{\Rg}{[0,\infty)}
\newcommand{\limfn}{\liminf \limits_{n \rightarrow \infty}}
\newcommand{\limpn}{\limsup \limits_{n \rightarrow \infty}}
\newcommand{\limn}{\lim \limits_{n \rightarrow \infty}}
\newcommand{\convt}[1]{\xrightarrow{\text{#1}}}
\newcommand{\conv}[1]{\xrightarrow{#1}} 
\newcommand{\p}[1]{\frac{\partial}{\partial{#1}}}
\newcommand{\Ll}{L^1_{\text{loc}}(\R^n)}
\newcommand{\seq}[1]{(x_{#1})_{#1 \in \N}}
\newcommand{\n}{\Vert}

\DeclareMathOperator*{\argmax}{argmax}
\DeclareMathOperator*{\argmin}{argmin}
\DeclareMathOperator*{\E}{\mathbb{E}}
 
\begin{document}

\title{Quantum Mechanics Notes}
\author[James]{Carson James}
\maketitle


\tableofcontents

\section{Introduction}
\subsection{Schr\"{o}dinger Equation}

\begin{defn}
A particle with potential energy $V(r,t)$ is completely decribed by its \textbf{position wavefunction} $\Psi(r,t)$, which satisfies the \textbf{Schr\"{o}dinger equation}: 
\begin{align*}
i\hbar \p{t} \Psi 
&= -\frac{\hbar^2}{2m} \Del \Psi + V \Psi\\
\end{align*}
\end{defn}

\begin{intp}
We interpret $\vert\Psi(r,t)\vert^2$ to be the \textbf{probability density} for the position, $r$, of the particle at time $t$. Therefore, we require that for each $t \in \R$, $$\int_{\R^n}\Psi(r,t)^* \Psi(r,t) dr = 1$$
\end{intp}

\subsection{Operators}

\begin{defn}

We define the jth \textbf{position} and \textbf{momentum} coordinate operators $X_j,P_j$, (in position space) by $$X_j \Psi(r,t) = x_j \Psi(r,t)$$ and $$P_j \Psi(r,t) = -i \hbar \p{x_j} \Psi(r,t)$$ 
If the partical has potential energy $V(r,t)$, we define the \textbf{Hamiltonian} operator, $H$, by $$H = \frac{P^2}{2m} + V$$ Thus the Schr\"{o}dinger equation reads $$i\hbar \p{t}\Psi = H \Psi$$ 
\end{defn}

\begin{note}
If the potential energy doesn't depend on time, we may write $$H = \frac{P^2}{2m} + V(X)$$ meaning Hamiltonian only depends on the position and momentum operators, $X$ and $P$. For the rest of these notes, we assume that the potential energy $V$ does not depend on time.
\end{note}

\begin{defn}
Let $A$ and $B$ be operators. Then $B$ is said to be the \textbf{adjoint} of $A$ if for each $\Psi_1$, $\Psi_2$, $$\l \Psi_1 \vert A\Psi_2 \r = \l B \Psi_1 \vert \Psi_2 \r$$ i.e. $$\int_{\R^n}\Psi_1^* (A\Psi_2) dr = \int_{\R^n}(B\Psi_1)^* \Psi_2 dr$$ If B is the adjoint of $A$, we write $$B = A^{\dagger}$$
\end{defn}

\begin{ex}
Let $A$ be an operator, then \begin{enumerate}
\item for each $\Psi_1, \Psi_2$, $\l A\Psi_1 \vert \Psi_2 \r = \l   \Psi_1 \vert A^{\dagger} \Psi_2 \r$ 
\item $(A^{\dagger})^{\dagger} = A$
\end{enumerate}
\end{ex}

\begin{proof}
\begin{enumerate}
\item For wavefunctions $\Psi_1$, $\Psi_2$, we have
\begin{align*}
\l A \Psi_1 \vert \Psi_2 \r
&= \l \Psi_2 \vert A \Psi_1 \r^*\\
&= \l A^{\dagger}\Psi_2 \vert  \Psi_1 \r^* \hspace{.5cm} \text{(by definition)}\\
&= \l  \Psi_1 \vert A^{\dagger}\Psi_2 \r
\end{align*}
\item For each $\Psi_1, \Psi_2$, we have that
\begin{align*}
\l  A\Psi_1 \vert \Psi_2 \r
&= \l   \Psi_1 \vert A^{\dagger}\Psi_2 \r \\
&= \l  (A^{\dagger})^{\dagger} \Psi_1 \vert \Psi_2 \r
\end{align*}

This implies that for each $\Psi_1, \Psi_2$, $$\big \l \big[A-(A^{\dagger})^{\dagger} \big] \Psi_1, \Psi_2 \big \r = 0$$ Therefore for each $\Psi_1$, $$ \big[A-(A^{\dagger})^{\dagger} \big] \Psi_1 = 0$$ Hence $ \l A-(A^{\dagger})^{\dagger}  = 0$ and $A = (A^{\dagger})^{\dagger}$.
\end{enumerate}
\end{proof}

\begin{defn}
An linear operator $Q$ is \textbf{self-adjoint} if $$Q = Q^{\dagger}$$
\end{defn}

\begin{intp}
For each measurable, observable quantity $\hat{Q}$, there is a self-adjoint operator $Q$ whose eigenvalues are the possible measurment values and whose eigenfunctions are the possible states of the system at measurment.
\end{intp}

\begin{ex}
The operators $X_j, P_j$ and $H$ are self adjoint. 
\end{ex}

\begin{proof}
Since $x_j$ is real, clearly $$\l \Psi_1 \vert X_j \Psi_2 \r = \l X_j\Psi_1 \vert \Psi_2 \r $$ Similarly, we have that 
\begin{align*}
\l \Psi_1 \vert P_j \Psi_2 \r
&=  \int_{\R^n} \Psi_1^* \bigg(-i\hbar\p{x_j} \Psi_2\bigg)dr \\
&= -i \hbar\int_{\R^n} \Psi_1^*  \bigg(\p{x_j} \Psi_2\bigg)dr\\
&= i\hbar \int_{\R_n} \bigg( \p{x_j} \Psi_1^* \bigg) \Psi_2 dr \hspace{1cm } \text{(integration by parts)}\\
&= \int_{\R^n} \bigg( -i \hbar \p{x_j} \Psi_1 \bigg)^* \Psi_2 dr\\
&= \l P \Psi_1 \vert \Psi_2 \r
\end{align*}

Finally 
\begin{align*}
\l \Psi_1 \vert H \Psi_2 \r - \l H \Psi_1 \vert  \Psi_2 \r
&= \int_{\R^n} \Psi_1^* \bigg(-\frac{\hbar^2}{2m}\Del \Psi_2 + V \Psi_2\bigg)dr - \int_{\R^n} \bigg(-\frac{\hbar^2}{2m}\Del \Psi_1 + V\Psi_1\bigg)^*  \Psi_2 dr \\
&= \frac{\hbar^2}{2m}\int_{\R^n} (\Del \Psi_1^* )\Psi_2 - \Psi_1^*(\Del \Psi_2)dr\\
&= 0 \hspace{1cm} \text{(Green's second identity)}
\end{align*}
\end{proof}

\begin{ex}
Let $Q$ be a self-adjoint operator. Then 
\begin{enumerate}
\item the eigenvalues of $Q$ are real.
\item the eigenfunctions of $Q$ corresponding to distinct eigenvalues are orthogonal.
\end{enumerate}
\end{ex}

\begin{proof}
\ \begin{enumerate}
\item Let $\lam$ be an eigenvalue of $Q$ with corresponding eigenfunction $\Psi$. Then 
\begin{align*}
 \lam \l \Psi \vert \Psi\r
&= \l \Psi \vert Q \Psi\r \\
&= \l Q \Psi \vert \Psi\r \\
&= \lam^* \l \Psi \vert \Psi\r
\end{align*}
Thus $\lam = \lam^*$ and is real

\item Let $\lam_1$ and $\lam_2$ be eigenvalues of $Q$ with corresponding eigenfunctions $\Psi_1$ and $\Psi_2$. Suppose that $\lam_1 \neq \lam_2$. Then 
\begin{align*}
\lam_2 \l \Psi_1 \vert  \Psi_2\r
&= \l \Psi_1 \vert Q \Psi_2\r\\
&= \l Q \Psi_1 \vert  \Psi_2\r\\
&= \lam_1 \l \Psi_1 \vert  \Psi_2\r
\end{align*}
So $(\lam_2 - \lam_1)\l \Psi_1 \vert  \Psi_2\r = 0$. Which implies that $\l \Psi_1 \vert  \Psi_2\r=0$
\end{enumerate}
\end{proof}

\begin{defn}
Let $A$ and $B$ be operators. The \textbf{commutator} of $A$ and $B$, $[A,B]$, is defined by $$[A,B] = AB - BA$$
\end{defn}

\begin{ex}
We have $[X_j, P_j] = i\hbar$.
\end{ex}

\begin{proof}
For a position wave function $\Psi$, 
\begin{align*}
[X_j, P_j]\Psi(r,t)
&= [x_j, -i\hbar \p{x_j}]\Psi(r,t)\\
&= (-i\hbar) \bigg[x_j \p{x_j}\Psi(r,t)- \p{x_j}x_j\Psi(r,t)\bigg]\\
&= (-i\hbar)\bigg[ x_j \p{x_j}\Psi(r,t)- \Psi(r,t) - x_j \p{x_j}\Psi(r,t)\bigg]\\
&=i\hbar \Psi(r,t)
\end{align*}

Hence $[X_j, P_j] = i\hbar$
\end{proof}

\subsection{Continuity Equation}

\begin{ex}
If $V$ is real and $\Psi$ satisfies the Schr\"{o}dinger equation, then $$i\hbar \p{t} \Psi^* = -H\Psi^* $$
\end{ex}

\begin{proof}
We have that 
\begin{align*}
i \hbar \p{t} \Psi^{*} 
&= \bigg(-i \hbar \p{t} \Psi\bigg)^*\\
&=\bigg( - \bigg[-\frac{\hbar^2}{2m}\Del \Psi + V \Psi \bigg] \bigg)^*\\
&= - \bigg[ -\frac{\hbar^2}{2m}\Del \Psi^* + V \Psi^*\bigg]\\
&= -H \Psi^*
\end{align*}
\end{proof}

\begin{ex}
We have that $$\p{t} (\Psi^* \Psi) + \frac{\hbar}{2mi} \nabla \cdot \bigg[ \Psi^* (\nabla \Psi) - (\nabla \Psi^*) \Psi\bigg] = 0 $$
\end{ex}

\begin{proof}
\begin{align*}
\p{t}(\Psi^* \Psi) 
&= \bigg(\p{t} \Psi^* \bigg) \Psi + \Psi^* \bigg(\p{t} \Psi \bigg)\\
&= \bigg( \frac{\hbar}{2mi} (\Del \Psi^*) \Psi - \frac{1}{i \hbar }V \Psi^* \Psi\bigg) + \bigg( -\frac{\hbar}{2mi}  \Psi^* (\Del \Psi) + \frac{1}{i \hbar }V \Psi^* \Psi \bigg)\\
&= \frac{\hbar}{2mi} \bigg[ (\Del \Psi^*) \Psi - \Psi^* (\Del \Psi) \bigg]\\
&= -\frac{\hbar}{2mi} \bigg[ \Psi^* (\Del \Psi) - (\Del \Psi^*) \Psi\bigg]\\
&= - \frac{\hbar}{2mi} \nabla \cdot \bigg[\Psi^* (\nabla \Psi) - (\nabla \Psi^*) \Psi \bigg]
\end{align*}

Therefore  $$\p{t} (\Psi^* \Psi) + \frac{\hbar}{2mi} \nabla \cdot \bigg[ \Psi^* (\nabla \Psi) - (\nabla \Psi^*) \Psi\bigg] = 0 $$
\end{proof}

\begin{defn}
We define the \textbf{probability current density}, $j$, of the particle to be $$j = \frac{\hbar}{2mi} \bigg[ \Psi^* (\nabla \Psi) - (\nabla \Psi^*) \Psi\bigg]$$ 
\end{defn}
\subsection{Position and Momentum Space}
\begin{defn}
We define the \textbf{momemtum wavefunction}, $\Phi$, of the particle to be the Fourier transform of the position wavefunction: 
\begin{align*}
\Phi(p,t) 
&= F[\Psi](p,t)\\
&= \frac{1}{(2 \pi \hbar)^{n/2}} \int_{\R ^n}\Psi(r,t)e^{-i \frac{p \cdot r}{\hbar} }dr 
\end{align*}
\end{defn}

\begin{note}
We recall the following facts about Fourier transforms:
\begin{enumerate}
\item $$\Phi(p,t) = \frac{1}{(2 \pi \hbar)^{n/2}} \int_{\R ^n}\Psi(r,t)e^{-i \frac{p \cdot r}{\hbar} }dr $$ and $$\Psi(r,t) = \frac{1}{(2 \pi \hbar)^{n/2}} \int_{\R ^n}\Phi(p,t)e^{i \frac{p \cdot r}{\hbar} }dp $$

\item $$F\bigg[\p{x_j} \Psi \bigg] = \frac{i p_j}{\hbar}F[\Psi]$$
and $$F^{-1}\bigg[\p{p_j} \Phi \bigg] = -\frac{i x_j}{\hbar}F[\Psi]$$

\item $$\int_{\R^n} \Psi_1^* \Psi_2 dr = \int_{\R^n} F[\Psi_1]^* F[\Psi_2]dr$$
\end{enumerate}
\end{note}

\begin{note}
Let $Q(X,P)$ be a self-adjoint operator. Then the properties of the Fourier transform inmply that:
\[
Q(X,P)=
\begin{cases}
Q(x, -i\hbar \nabla) & \text{(position space)}\\
Q(i\hbar \nabla, p) & \text{(momentum space)}
\end{cases}
\]
\end{note}

\begin{ex}
If $\Psi$ satisfies the Schr\"{o}dinger equation, then $\Phi$ satisfies $$i\hbar \p{t}\Phi = \frac{p^2}{2m}\Phi + V(i \hbar \nabla)\Phi$$
\end{ex}

\begin{proof}
Starting with the Schr\"{o}dinger equation, we have 
\begin{align*}
i\hbar \p{t} \Psi 
&= \bigg[\frac{P^2}{2m} + V(X)\bigg] \Psi\\
&= \bigg[\frac{-\hbar^2}{2m}\Del + V(r)\bigg] \Psi \hspace{1cm} \text{(position space)}
\end{align*} 
Taking Fourier transforms of both sides, we see that 
\begin{align*}
i\hbar \p{t} \Phi 
&= \bigg[\frac{P^2}{2m} + V(X)\bigg] \Phi\\
&= \bigg[\frac{p^2}{2m} + V(i \hbar \nabla)\bigg] \Phi \hspace{1cm} \text{(position space)}
\end{align*} 
\end{proof}

\begin{intp}
We interpret $\vert \Phi (p,t) \vert^2$ to be the probability density for the momentum, $p$, of the particle at time $t$.  
\end{intp}

\begin{note}
For a self-adjoint operator $Q(X,P)$, the expected value of $Q$,  is given by 

\[ 
\l Q \r = 
\begin{cases}
\l \Psi(r,t) \vert Q(r, -i\hbar \nabla) \Psi(r,t)\r & \hspace{1cm} \text{(position space)}\\
\l \Phi(p,t) \vert Q(i\hbar \nabla, p) \Phi(p,t) \r & \hspace{1cm} \text{(momentum space)}\\
\end{cases}
\]
\end{note}

\subsection{Stationary States}
\begin{defn}
When the potential energy $V$ doesn't depend on time, we look for solutions to the \sch equation of the form $$\Psi(r,t) = \psi(r) \varphi(t)$$ With a closer look, we find that

\begin{enumerate}
\item $H\psi = E \psi$
\item $\varphi(t) = e^{-i\frac{E}{\hbar}t}$
\end{enumerate}
Eigenfuntions of the Hamiltonian operator are called \textbf{stationary states}. If the possible eigenvalues for the Hamiltonian operator are discreet $(E_n)_{n\in \N}$ with stationary states $(\psi_n)_{n \in \N}$, then the general solution to the \sch equation is $$\Psi(r,t) = \sum_{n \in \N} c_n \psi_n(r) e^{-i\frac{E_n}{\hbar}t}$$ where $$c_n = \int_{\R^n}\psi_n^*(r) \Psi(r,0)dr$$
\end{defn}

\begin{defn}
If the spectrum of the Hamiltonian is discreet, the stationary state with the least energy is called the \textbf{ground state}. The stationary states that are not the ground state are called \textbf{excited states}.
\end{defn}

\section{Fundamental Examples in One Dimension}

\subsection{The Harmonic Oscillator}

\begin{defn}
The \textbf{harmonic oscillator} in one dimension is defined by the potential energy: $$V(x) = \frac{1}{2}m \om^2 x^2$$ We define the \textbf{lowering operator}, $a$, by $$a = \frac{1}{\sqrt{2 \hbar m \om }}\bigg(m\om X +iP\bigg)$$ 
\end{defn}

\begin{ex}
The adjoint of the lowering operator is $$a^{\dagger} = \frac{1}{\sqrt{2 \hbar m \om }}\bigg(m\om X -iP\bigg)$$
\end{ex}

\begin{proof}
For a wave functions $\Psi_1$, $\Psi_2$,
\begin{align*}
\int_{\R}\bigg[ \frac{1}{\sqrt{2 \hbar m \om }}\bigg(m\om X -iP\bigg)\Psi_1 \bigg]^* \Psi_2 dx
&= \frac{1}{\sqrt{2 \hbar m \om }}\int_{\R}(m\om x \Psi_1(x,t)^* \Psi_2(x,t) -\hbar \bigg(\p{x}\Psi_1(x,t)^*\bigg)\Psi_2(x,t) dx\\
&=\frac{1}{\sqrt{2 \hbar m \om }}\int_{\R}(m\om x \Psi_1(x,t)^* \Psi_2(x,t) +\hbar \Psi_1(x,t)^*\bigg(\p{x}\Psi_2(x,t)\bigg) dx \hspace{.5cm}\\
&=\int_{\R}\Psi_1^*\bigg[ \frac{1}{\sqrt{2 \hbar m \om }}\bigg(m\om X +iP\bigg)\Psi_2 \bigg]  dx
\end{align*}
\end{proof}

\begin{defn}
We call $a^{\dagger}$ the \textbf{raising operator} and together, $a$ and $a^{\dagger}$ are called the ladder operators.
\end{defn}

\begin{ex}
We have that 
\begin{enumerate}
\item $aa^{\dagger} = \frac{1}{\hbar \om}H + \frac{1}{2}$
\item $a^{\dagger}a = \frac{1}{\hbar \om}H - \frac{1}{2}$
\item $[a,a^{\dagger}] = 1$
\end{enumerate}
\end{ex}

\begin{proof}
\begin{enumerate}
\item \
\begin{align*}
a a^{\dagger}
&= \frac{1}{2\hbar m \om}\big(m \om X + iP \big) \big( m\om X - iP )\\
&= \frac{1}{2 \hbar m \om} \bigg[ \big(m^2 \om^2 X^2 + P^2 \big) - m\om i\big(XP - PX \big) \bigg]\\
&= \frac{1}{\hbar \om}\big(\frac{1}{2m}P^2 + \frac{1}{2}m \om^2 X^2 \big) - \frac{i}{2 \hbar}\big[X,P \big]\\
&= \frac{1}{\hbar \om}H + \frac{1}{2}
\end{align*}
\item Similar
\item Trivial
\end{enumerate}
\end{proof}

\begin{ex}
If $H\psi = E\psi$, then 
\begin{enumerate}
\item $Ha\psi = (E-\hbar \om) a \psi$
\item $Ha^{\dagger}\psi = (E+\hbar \om) a \psi$
\end{enumerate}
\end{ex}

\begin{proof}\
\begin{enumerate}
\item \
\begin{align*}
Ha\psi 
&= \hbar \om \bigg(aa^{\dagger}-\frac{1}{2}\bigg)a \psi\\
&= \hbar \om \bigg(aa^{\dagger}a-\frac{1}{2}a\bigg) \psi\\
&= \hbar \om a\bigg(a^{\dagger}a-\frac{1}{2}\bigg) \psi\\
&= \hbar \om a\bigg(a^{\dagger}a+\frac{1}{2} -1\bigg) \psi\\
&= \hbar \om a\bigg(\frac{1}{\hbar \om}H -1\bigg) \psi\\
&= a H\psi -\hbar \om a \psi \\
&= (E - \hbar \om)a\psi 
\end{align*}
\item Similar
\end{enumerate}
\end{proof}

\begin{intp}
The lowering operator ``lowers"  a stationary state $\psi$ with energy $E$ to a stationary state $a\psi$ with energy $E-\hbar \om$ and the raising operator ``raises"  a stationary state $\psi$ with energy $E$ to a stationary state $a^{\dagger}\psi$ with energy $E+\hbar \om$.
\end{intp}

\begin{defn}
Since the zero function is a solution to the time-independent \sch equation, we define the ground state, $\psi_0$ of the harmonic oscillator to be the stationary state that satisfies $a\psi_0 = 0$. 
\end{defn}

\begin{ex}
We have that
\begin{enumerate}
\item $\psi_0(x) = $
\item $E_0 = \frac{1}{2}\hbar \om$
\end{enumerate}

\end{ex}

\begin{proof}\
\begin{enumerate}
\item The simple differential equation $a\psi_0 = 0$ has the solution $$\psi_0 = Ae^{-\frac{m \om}{2 \hbar}x^2}$$ If we normalize this function, we obtain $$\psi_0= \sqrt{\frac{2 \pi \hbar}{m \om}}e^{-\frac{m \om}{2 \hbar}x^2}$$
\item It is tedious but straightforward to show that $$H\psi_0 = \frac{1}{2}\hbar \om$$
\end{enumerate} 
\end{proof}

\end{document}

