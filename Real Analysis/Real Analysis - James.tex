
\documentclass[12pt]{amsart}
 \usepackage[margin=1in]{geometry} 
\usepackage{amsmath,amsthm,amssymb,amsfonts,setspace}
\usepackage[shortlabels]{enumitem}
\usepackage{exercise, chngcntr}
\usepackage{mathtools}
\usepackage{comment}
%
%
%
\newif\ifhideproofs
%\hideproofstrue %uncomment to hide proofs
%
%
%
%
\ifhideproofs
\usepackage{environ}
\NewEnviron{hide}{}
\let\proof\hide
\let\endproof\endhide
\fi

\newtheorem{thm}{Theorem}[section]
\newtheorem{lem}[thm]{Lemma}
\newtheorem{prop}[thm]{Proposition}
\newtheorem{cor}[thm]{Corollary}
\newtheorem{conj}{Conjecture}
\newtheorem{defn}[thm]{Definition}
\newtheorem{note}[thm]{Note}
\newtheorem{ex}[thm]{Exercise}


\newcommand{\al}{\alpha}
\newcommand{\Gam}{\Gamma}
\newcommand{\be}{\beta} 
\newcommand{\del}{\delta} 
\newcommand{\Del}{\Delta}
\newcommand{\lam}{\lambda}  
\newcommand{\Lam}{\Lambda} 
\newcommand{\ep}{\epsilon}
\newcommand{\sig}{\sigma} 
\newcommand{\om}{\omega}
\newcommand{\Om}{\Omega}
\newcommand{\C}{\mathbb{C}}
\newcommand{\N}{\mathbb{N}}
\newcommand{\E}{\mathbb{E}}
\newcommand{\Z}{\mathbb{Z}}
\newcommand{\R}{\mathbb{R}}
\newcommand{\Q}{\mathbb{Q}}
\renewcommand{\P}{\mathbb{P}}
\newcommand{\MA}{\mathcal{A}}
\newcommand{\MC}{\mathcal{C}}
\newcommand{\MB}{\mathcal{B}}
\newcommand{\MF}{\mathcal{F}}
\newcommand{\MG}{\mathcal{G}}
\newcommand{\ML}{\mathcal{L}}
\newcommand{\MN}{\mathcal{N}}
\newcommand{\MS}{\mathcal{S}}
\newcommand{\MP}{\mathcal{P}}
\newcommand{\ME}{\mathcal{E}}
\newcommand{\MT}{\mathcal{T}}
\newcommand{\MM}{\mathcal{M}}

\newcommand{\io}{\text{ i.o.}}
\newcommand{\ev}{\text{ ev.}}

\newcommand{\RG}{[0,\infty]}
\newcommand{\Rg}{[0,\infty)}
\newcommand{\limfn}{\liminf \limits_{n \rightarrow \infty}}
\newcommand{\limpn}{\limsup \limits_{n \rightarrow \infty}}
\newcommand{\limn}{\lim \limits_{n \rightarrow \infty}}
\newcommand{\convt}[1]{\xrightarrow{\text{#1}}}
\newcommand{\conv}[1]{\xrightarrow{#1}} 

\newcommand{\Ll}{L^1_{\text{loc}}(\R^n)}
\newcommand{\seq}[1]{(x_{#1})_{#1 \in \N}}

\newcommand{\n}{\Vert}
 
\begin{document}

\title{Real Analysis Notes}
\author[James]{Carson James}
\maketitle

\tableofcontents

\section{Algebra and Analysis of Sets}

\subsection{Limits}

\begin{defn}
Let $X$ be a set and $\MA \subset \MP(X)$. We define $$\inf \MA = \bigcap_{A \in \MA } A ,\hspace{.5cm} \sup \MA = \bigcup_{A \in \MA} A$$

\end{defn}

\begin{defn}
Let $X$ be a set and $(A_n)_{n \in \N} \subset \MP(X)$ a sequence of subsets. We define
$$\liminf_{n \rightarrow \infty} A_n = \sup_{n \in \N} \bigg( \inf_{k \geq n}A_k \bigg), \hspace{.5cm } \limsup_{n \rightarrow \infty} A_n = \inf_{n \in \N} \bigg(\sup_{k \geq n}A_k \bigg)$$
\end{defn}

\begin{note}\
\begin{enumerate}
\item $\liminf\limits_{n \rightarrow \infty} A_n$ is the set of elements that are in all $A_n$ except for finitely many. 

\item $\limsup\limits_{n \rightarrow \infty} A_n$ is the set of elements that are in infinitely many $A_n$.
\end{enumerate}
\end{note}

\begin{ex}
Let $X$ be a set and $(A_n)_{n \in \N} \subset \MP(X)$ a sequence of subsets. Then 
\begin{enumerate}
\item $\liminf\limits_{n \rightarrow \infty}A_n = \bigg \{x \in X: \liminf\limits_{n \rightarrow \infty}\chi_{A_n}(x) = 1\bigg\}$
\item $\limsup\limits_{n \rightarrow \infty}A_n = \bigg \{x \in X: \limsup\limits_{n \rightarrow \infty}\chi_{A_n}(x) = 1\bigg\}$
\end{enumerate}
\end{ex}

\begin{proof}\
\begin{enumerate}
\item Let $x \in \liminf\limits_{n \rightarrow \infty}A_n$. Then there exists $n^* \in \N$ such that for each $k \in \N$, $k \geq n^*$ implies that $x \in A_k$. So for each $k \in \N$, $k \geq n^*$ implies that $\chi_{A_k}(x) = 1$. Then $\inf\limits_{k \geq n^*}\chi_{A_k}(x) = 1$ and thus $$1 = \sup\limits_{n \in \N} \bigg(\inf\limits_{k \geq n} \chi_{A_k}(x) \bigg) = \liminf_{n \rightarrow \infty}\chi_{A_n}(x)$$ \vspace{3mm} \\
Conversely, if $1 = \liminf\limits_{n \rightarrow \infty}\chi_{A_n}(x)$, then choosing $\ep = \frac{1}{2}$, there exists $n \in \N$ such that for each $k \in \N$, $k \geq n$ implies that $\chi_{A_k}(x) > 1-\ep$. Hence for each $k \in \N$, $k \geq n$ implies that $\chi_{A_k}(x) = 1$. So for each for each $k \in \N$, $k \geq n$ implies that $x \in A_k$. So $x \in \liminf\limits_{n \rightarrow \infty} A_n$. 
\item Similar to (1).
\end{enumerate}
\end{proof}

\begin{ex}
Let $A_k = [0, \frac{k}{k+1})$. Then 
\begin{enumerate}
\item $\inf\limits_{k \geq n}A_k = [0, \frac{n}{n+1})$ \\
\item $\sup\limits_{k \geq n}A_k = [0,1)$ \\
\item $\liminf\limits_{n \rightarrow \infty}A_n = [0,1)$ \\
\item $\liminf\limits_{n \rightarrow \infty}A_n = [0,1)$
\end{enumerate}
\end{ex}

\begin{proof}
Straightforward.
\end{proof}

\begin{ex}
Let $X$ be a set and $(A_n)_{n \in \N} \subset \MP(X)$ a sequence of subsets. Then $$\liminf_{n \rightarrow \infty} A_n \subset \limsup_{n \rightarrow \infty} A_n$$
\end{ex}

\begin{proof}
Let $x \in \bigcup\limits_{n=1}^{\infty} \bigcap\limits_{k =n}^{\infty} A_k$. Then there exists $n^* \in \N$ such that for each $k \in \N$, if $k \geq n*$, then $x \in A_k$. Let $n \in \N$. Choose $k = \max\{n^*,n\} \geq n^*$. Then $x \in A_k$. Hence for each $n \in \N$, there exists $k \in \N$ such that $k \geq n$ and $x \in A_k$. So $x \in \bigcap\limits_{n=1}^{\infty} \bigcup\limits_{k=n}^{\infty} A_k$. Thus $\liminf\limits_{n \rightarrow \infty}A_n \subset \limsup\limits_{n \rightarrow \infty}A_n$.
\end{proof}

\begin{defn}
Let $X$ be a set and $(A_n)_{n \in \N} \subset \MP(X)$ a sequence of subsets. If $$\liminf_{n \rightarrow \infty} A_n = \limsup_{n \rightarrow \infty} A_n$$ then we define $$\lim_{n \rightarrow \infty}A_n = \liminf_{n \rightarrow \infty} A_n = \limsup_{n \rightarrow \infty} A_n$$ 
\end{defn}

\begin{ex}
Let $X$ be a set and $(A_n)_{n \in \N}, (B_n)_{n \in \N} \subset \MP(X)$ sequences of subsets. Suppose that for each $n \in \N$, $A_n \subset A_{n+1}$ and $B_{n+1} \subset B_n$. Then 
\begin{enumerate}
\item $\limn A_n = \sup\limits_{n \in \N}A_n = \bigcup\limits_{n=1}^{\infty}A_n$
\item $\limn B_n = \inf\limits_{n \in \N}B_n = \bigcap\limits_{n=1}^{\infty}B_n$
\end{enumerate}
\end{ex}

\begin{proof}\
\begin{enumerate}
\item Let $n \in \N$. Then 
\begin{align*}
\inf\limits_{k \geq n}A_k 
&= \bigcap\limits_{k=n}^{\infty}A_k \\
&= A_n
\end{align*}
Thus 
\begin{align*}
\liminf\limits_{n \rightarrow \infty}A_n 
&= \bigcup\limits_{n=1}^{\infty} \inf\limits_{k \geq n}A_k \\
&= \bigcup\limits_{n=1}^{\infty} A_n 
\end{align*}  

In addition,  
\begin{align*}
\sup_{n \geq k} A_k 
&= \bigcup_{k=n}^{\infty}A_k \\
&= \bigcup_{k=1}^{\infty}A_k 
\end{align*}

Therefore 
\begin{align*}
\limsup\limits_{n \rightarrow \infty}A_n 
&= \bigcap\limits_{n=1}^{\infty} \inf\limits_{k \geq n}A_k \\
&= \bigcup\limits_{n=1}^{\infty} \bigcup_{k=1}^{\infty}A_k  \\
&= \bigcup_{n=1}^{\infty}A_n
\end{align*}
So $$\lim\limits_{n \rightarrow \infty} A_n = \sup_{n \in \N}A_n = \bigcup_{n =1}^{\infty}A_n$$

\item Similar
\end{enumerate}
\end{proof}

\begin{ex}
Let $X$ be a set and $(A_n)_{n \in \N} \subset \MP(X)$ a sequence of subsets and $(A_{n_k})_{k \in \N}$ a subsequence of $(A_n)_{n \in \N}$. Then 
\begin{enumerate}
\item $\limsup\limits_{k \rightarrow \infty}A_{n_k} \subset \limsup\limits_{n \rightarrow \infty}(A_{n})$
\item $\liminf \limits_{n \rightarrow \infty} A_{n} \subset \limsup\limits_{k \rightarrow \infty}(A_{n_k})$
\end{enumerate}
\end{ex}

\begin{proof}\
\begin{enumerate}
\item The elements that are in $A_{n_k}$ for infinitely many $k$ are in $A_n$ for infinitely many $n$.
\item Similar.
\end{enumerate}
\end{proof}

\begin{ex}
Let $X$ be a set and $(A_n)_{n \in \N} \subset \MP(X)$ a sequence of subsets, $(A_{n_k})_{k \in \N}$ a subsequence of $(A_n)_{n \in \N}$ and $A \subset X$. If $A_{n_k} \rightarrow A$, then $$\liminf\limits_{n \rightarrow \infty}A_n \subset A \subset \limsup\limits_{n \rightarrow \infty}A_n$$
\end{ex}

\begin{proof}
The previous exercises tells us that 
\begin{align*}
\liminf\limits_{n \rightarrow \infty}A_n
& \subset \liminf\limits_{k \rightarrow \infty}A_{n_k} \\
&= A \\
&= \limsup\limits_{k \rightarrow \infty}A_{n_k} \\
& \subset \limsup\limits_{n \rightarrow \infty}A_n
\end{align*}
\end{proof}

\begin{ex}
Let $X$ be a set and $(A_n)_{n \in \N}, (B_n)_{n \in \N} \subset \MP(X)$ sequences of subsets. Suppose that for each $n \in \N$, $A_n \subset B_n$. Then 
\begin{enumerate}
\item $\limsup\limits_{n \rightarrow \infty}A_n \subset \limsup\limits_{n \rightarrow \infty}B_n$
\item $\liminf\limits_{n \rightarrow \infty}A_n \subset \liminf\limits_{n \rightarrow \infty}B_n$
\end{enumerate}
\end{ex}

\begin{proof}\
\begin{enumerate}
\item Let $x \in \limsup\limits_{n \rightarrow \infty}A_n$. Then for infinitely many $n \in \N$, $x \in A_n \subset B_n$. So for infinitely many $n \in \N$, $x \in B_n$. Hence $x \in \limsup\limits_{n \rightarrow \infty}B_n$. Therefore $\limsup\limits_{n \rightarrow \infty}A_n \subset \limsup\limits_{n \rightarrow \infty}B_n$.
\item Similar.
\end{enumerate}
\end{proof}

\begin{ex}
Let 
\end{ex}

\begin{ex}
Let $X$ be a set and $(A_n)_{n \in \N} \subset \MP(X)$ a sequence of subsets. Then 
\begin{enumerate}
\item $\limsup\limits_{n \rightarrow \infty}A_n = \bigg(\liminf\limits_{n \rightarrow \infty}A_n^c \bigg)^c$
\item $\liminf\limits_{n \rightarrow \infty}A_n = \bigg(\limsup\limits_{n \rightarrow \infty}A_n^c \bigg)^c$
\end{enumerate}
\end{ex}

\begin{proof}\
\begin{enumerate}
\item \begin{align*}
\bigg( \liminf\limits_{n \rightarrow \infty}A_n^c \bigg)^c 
&= \bigg( \bigcup\limits_{n=1}^{\infty} \bigcap\limits_{k=n}^{\infty}A_k^c \bigg)^c\\
&= \ \bigcap\limits_{n=1}^{\infty}\bigcup\limits_{k=n}^{\infty}A_k  \\
&=  \limsup\limits_{n \rightarrow \infty}A_n
\end{align*}
\item Similar.
\end{enumerate}
\end{proof}\newpage

\begin{ex}
For $n \in \N$, define $$A_n = \bigg\{ \frac{m}{n}: m \in \N \bigg\}$$ 
Then
\begin{enumerate}
\item $\liminf\limits_{n \rightarrow \infty }A_n = \N$ 
\item $\limsup\limits_{n \rightarrow \infty }A_n = \Q \cap (0,\infty )$
\end{enumerate}
\end{ex}

\begin{proof}\
\begin{enumerate}
\item For each $x \in \N$ and $n \in \N$, $x = \frac{nx}{n} \in A_n$ Hence $\N \subset \liminf\limits_{n \rightarrow \infty }A_n$. Conversely, let $x \in \liminf\limits_{n \rightarrow \infty }A_n$. Then there exists $n \in \N$ such that for each $k \in \N$, if $k \geq n$, then $x \in A_k$. In particular, $x \in A_n$. Hence there exists $m_n \in \N$ such that $x = \frac{m_n}{n}$. Choose $s,t \in \N$ such that $x= \frac{s}{t}$ and $\gcd(s,t) = 1$. Suppose that $t \neq 1$. Then choose a prime $p > n$. By assumption, $x \in A_p$. Then there exist $m_p \in \N$ such that $x = \frac{m_p}{p}$. Hence $\frac{s}{t} = \frac{m_p}{p}$ and $tm_p = sp$. Since $t \vert sp$ and $\gcd(s,t) = 1$, we see that $t \vert p$. If $t \geq 1$, then $p$ is not prime, a contradiction. So $t = 1$. Hence $x \in \N$. Thus $\liminf\limits_{n \rightarrow \infty }A_n \subset \N$. 
\item Let $x \in \Q \cap (0, \infty)$. Then there exist $s,t \in \N$ such that $x = \frac{s}{t}$. Define the subsequence $(A_{n_k})_{k \in \N}$ by $A_{n_k} = A_{tk}$. Then for each $k \in \N$, $x = \frac{sk}{tk} \in A_{tk} = A_{n_k}$. Thus $x \in \limsup\limits_{n \rightarrow \infty }A_n$. Conversely, clearly $\limsup\limits_{n \rightarrow \infty }A_n \subset \Q \cap (0, \infty)$ 
\end{enumerate}
\end{proof}

\begin{ex}
Let $X$ be a set and $(A_n)_{n \in \N}, (B_n)_{n \in \N} \subset \MP(X)$ sequences of subsets. Then $$\limsup\limits_{n \rightarrow \infty} A_n \cup B_n= \limsup\limits_{n \rightarrow \infty} A_n \cup \limsup\limits_{n \rightarrow \infty} B_n$$
\end{ex}

\begin{proof}
Let $x \in \limsup\limits_{n \rightarrow \infty} A_n \cup B_n$. Suppose that $x \not \in \limsup\limits_{n \rightarrow \infty} A_n$. Then there exists $n^* \in \N$ such that for each $k \in \N$ if $ k \geq n^*$, then $x \not \in A_k$. Let $n \in \N$. Then there exists $k$ such that $k \geq \max\{n, n^*\}$ and $x \in A_{k} \cup B_k$. Since $k \geq n^*$, $x \not \in A_{k}$ Thus $x \in B_k$. So for each $n \in \N$, there exists $k \in \N$ such that $k \geq n$ and $x \in B_k$. Therefore $x \in \limsup\limits_{n \rightarrow \infty}  B_n$ and $$\limsup\limits_{n \rightarrow \infty} A_n \cup B_n \subset \limsup\limits_{n \rightarrow \infty} A_n \cup \limsup\limits_{n \rightarrow \infty} B_n$$ Conversely, a previous exercise tells us that $\limsup\limits_{n \rightarrow \infty} A_n \subset \limsup\limits_{n \rightarrow \infty} A_n \cup B_n$ and $\limsup\limits_{n \rightarrow \infty}  B_n \subset \limsup\limits_{n \rightarrow \infty} A_n \cup B_n$. Thus $$ \limsup\limits_{n \rightarrow \infty} A_n \cup \limsup\limits_{n \rightarrow \infty} B_n \subset \limsup\limits_{n \rightarrow \infty} A_n \cup B_n$$
\end{proof}

\begin{ex}
Let $X$ be a set and $(A_n)_{n \in \N}, (B_n)_{n \in \N} \subset \MP(X)$ sequences of subsets. Then $$\liminf\limits_{n \rightarrow \infty} A_n \cap B_n= \liminf\limits_{n \rightarrow \infty} A_n \cap \liminf\limits_{n \rightarrow \infty} B_n$$
\end{ex}

\begin{proof}
A previous exercise tells us that 
\begin{align*}
\liminf\limits_{n \rightarrow \infty} A_n \cap B_n
&= \bigg( \limsup\limits_{n \rightarrow \infty} A_n^c \cup B_n^c \bigg)^c \\
&= \bigg( \limsup\limits_{n \rightarrow \infty} A_n^c \cup \limsup\limits_{n \rightarrow \infty}B_n^c \bigg)^c \\
&= \bigg( \limsup\limits_{n \rightarrow \infty} A_n^c \bigg)^c \cap \bigg( \limsup\limits_{n \rightarrow \infty}B_n^c \bigg)^c \\
&= \liminf\limits_{n \rightarrow \infty} A_n \cap \liminf\limits_{n \rightarrow \infty}  B_n
\end{align*}
\end{proof}

\subsection{Classes of sets}

\begin{defn}
Let $X$ be a set and $\MA \subset \MP(X)$. Then $\MA$ is said to be an \textbf{algebra} on $X$ if 
\begin{enumerate}
\item $\MA \neq \varnothing$
\item for each $A \in \MA$, $A^c \in \MA$
\item for each $A,B \in \MA$, $A \cup B \in \MA$
\end{enumerate}
\end{defn}

\begin{defn}
Let $X$ be a set and $\MA \subset \MP(X)$. Then $\MA$ is said to be a $\sigma$\textbf{-algebra} on $X$ if 
\begin{enumerate}
\item $\MA \neq \varnothing$
\item for each $A \in \MA$, $A^c \in \MA$
\item for each $(A_n)_{n \in \N} \subset \MA$, $\bigcup\limits_{n \in \N}A_n \in \MA$
\end{enumerate}
\end{defn}

\begin{ex}
Let $X$ be a set and $\MA$ a $\sig$-algebra on $X$. Then 
\begin{enumerate}
\item $X, \varnothing \in \MA$
\item for each $(A_n)_{n \in \N} \subset \MA$, $\bigcap\limits_{n \in \N} \in \MA$
\item For each $A, B \in \MA$, $A \setminus B \in \MA$  
\end{enumerate}
\end{ex}

\begin{proof}\
\begin{enumerate}
\item Since $\MA \neq \varnothing$, there exists $A \in \MA$. Then $A^c \in \MA$. Hence $X = A \cup A^c \in \MA$ and $\varnothing = X^c \in \MA$.
\item Let $(A_n)_{n \in \N} \subset \MA$. Then $(A_n^c)_{n \in \N} \subset MA$. So $\bigcup\limits_{n \in \N}A_n^c \in \MA$. Therefore \begin{align*}
\bigcap\limits_{n \in \N}A_n 
&= \bigg(\bigcup\limits_{n \in \N}A_n^c\bigg)^c \in \MA
\end{align*}
\item Let $A,B \in \MA$. Then $A \setminus B = A \cap B^c \in \MA$. 
\end{enumerate}
\end{proof}

\begin{ex}
Let $X$ be a set and $(\MA_i)_{i \in I}$ a collection of $\sig$-algebras (resp. algebra) on $X$. Then $\bigcap\limits_{i \in I}\MA_i$ is a $\sig$-algebra (resp. algebra) on $X$.
\end{ex}

\begin{proof}\
\begin{enumerate}
\item For each $i \in I$, $X \in \MA_i$. Thus $X \in \bigcap\limits_{i \in I}\MA_i$ and $\bigcap\limits_{i \in I}\MA_i \neq \varnothing$.
\item Let $A \in \bigcap\limits_{i \in I}\MA_i$. Then for each $i \in I$, $A \in \MA_i$. Hence for each $i \in I$, $A^c \in \MA_i$. Thus $A^c \in \bigcap\limits_{i \in I}\MA_i$. 
\item Let $(A_n)_{n \in \N} \subset \bigcap\limits_{i \in I}\MA_i$. Then for each $i \in I$, $(A_n)_{n \in \N} \subset \MA_i$. Thus for each $i \in I$, $\bigcup\limits_{n \in \N}A_n \in \MA_i$. So $\bigcup\limits_{n \in \N}A_n \in \bigcap\limits_{i \in I}\MA_i$.
\end{enumerate}
\end{proof}

\begin{defn}
Let $X$ be a set and $\MC \subset \MP(X)$. Put $$\MS = \{\MA \subset \MP(X): \MA \text{ is a }\sig\text{-algebra on }X \text{ and } \MC \subset \ML\}$$ We define the \textbf{$\sig$-algebra generated by $\MC$} on $X$, $\sig(\MC)$, by $$\sig(\MC) = \bigcap_{\MA \in \MS} \MA $$
\end{defn}

\begin{note}
Let $X$ be a set, $\MC \subset \MP(X)$ and $\MA$ a $\sig$-alg on $X$. By definition, if $\MC \subset \MA$, then $\sig(\MC) \subset \MA$.
\end{note}

\begin{note}
Let $X$ be a set, $\MT$ an ordered set and $(\MA_t)_{t\in \MT}$ a collection of $\sig$-algebras on $X$. Suppose that for each $s,t \in \MT$, if $s \leq t$, then $\MA_s \subset \MA_t$. If there exists $t \in \MT$ such that $\MA_t = \bigcup\limits_{t \in \MT}\MA_t$, then $\bigcup\limits_{t \in \MT}\MA_t$ is a $\sig$-algebra on $X$. So if $\MT$ is finite or if $(\MA_t)_{t \in \MT}$ termintates, the union is $\sig$-algebra.
\end{note}

\begin{defn}
Let $(X,\MT)$ be a topological space. We define the \textbf{Borel $\sig$-algebra} on $X$, $\MB(X)$, to be  $$\MB(X) = \sig(\MT)$$ The sets of $\MB(X)$ are called \textbf{Borel sets}.
\end{defn}

\begin{ex}
The Borel $\sigma$-algebra on $\R$ with the standard topology is given by 
\[
\MB(\R) =
\begin{cases}
\sig(\{(a,b]:a,b \in \R \text{ and } a<b\}) \\
\sig(\{[a,b]:a,b \in \R \text{ and } a<b\}) \\
\sig(\{[a,b):a,b \in \R \text{ and } a<b\}) \\
\sig(\{(a,b):a,b \in \R \text{ and } a<b\}) \\
\end{cases}
\]
\end{ex}

\begin{proof}
Define 
\begin{enumerate}
\item $\MC_{lo} = \{(a,b]:a,b \in \R \text{ and } a<b\}$\\
\item $\MC_{c} = \{[a,b]:a,b \in \R \text{ and } a<b\}$\\
\item $\MC_{ro} = \{[a,b):a,b \in \R \text{ and } a<b\}$\\
\item $\MC_{o} = \{(a,b):a,b \in \R \text{ and } a<b\}$\\
\end{enumerate} 
Recall that for each open set $A \subset \R$, there exist $(a_i)_{n \in \N}, (b_i)_{i \in \N} \subset \R$ such that for each $i \in \N$, $a_i < b_i$, for each $i,j \in \N$, if $i \neq j$, then $(a_i,b_i) \cap (a_j, b_j) = \varnothing$ and $A = \bigcup\limits_{i \in \N}(a_i, b_i)$. This implies that $\MB(\R) = \sig(\MC_o)$. \vspace{2mm}\\
Now, let $a,b \in \R$. Suppose that $a<b$. Then 
\begin{enumerate}
\item $[a,b] = \bigcap\limits_{n \in \N}(a- \frac{1}{n}, b]$, so $\sig(\MC_{c}) \subset \sig(\MC_{lo})$\\
\item $[a,b) = \bigcup\limits_{n \in \N} [a,b-\frac{1}{n}]$, so $\sig(\MC_{ro}) \subset \sig(\MC_{c})$ \\
\item $(a,b) = \bigcup\limits_{n \in \N} [a+\frac{1}{n},b)$, so $\sig(\MC_{o}) \subset \sig(\MC_{ro})$\\
\item $(a,b] = \bigcap\limits_{n \in \N} (a,b+\frac{1}{n})$, so $\sig(\MC_{lo}) \subset \sig(\MC_{o})$\\
\end{enumerate}
Hence $\MB(\R) = \sig(\MC_o) = \sig(\MC_{ro}) = \sig(\MC_{c}) = \sig(\MC_{lo}) = \sig(\MC_{o})$. 
\end{proof}

\begin{ex}
Let $X$ be a set. Define $\MA = \{A \in \MA: A \text{ is countable or }A^c  \text{is countable}\}$. Then $\MA$ is a $\sig$-algebra on $X$.
\end{ex}

\begin{proof}\
\begin{enumerate}
\item Since $X^c = \varnothing$ is countable, $X \in \MA$.
\item Let $A \in \MA$. Suppose that $A^c$ is  uncountable. Then by assumption, $A = (A^c)^c$ is countable. Hence $A^c \in \MA$.
\item Let $(A_n)_{n \in \N} \subset \MA$. Then for each $n \in \N$, $A_n$ is countable or $A_n^c$ is countable. Suppose that $\bigcup\limits_{n \in \N}A_n$ is uncountable. Then there exists $N \in \N$ such that $A_N$ is uncountable. Hence $A_N^c$ is countable. Thus 
\begin{align*}
\bigg(\bigcup_{n \in \N}A_n \bigg)^c 
&= \bigcap_{n \in \N}A_n^c \\
& \subset A_N^c 
\end{align*}
So $\bigg(\bigcup\limits_{n \in \N}A_n \bigg)^c $ is countable and $\bigcup\limits_{n \in \N}A_n \in \MA$. 
\end{enumerate}
\end{proof}

\begin{defn}
Let $X$ be a set, $\MC \subset \MP(X)$ and $A \subset X$. We define $$\MC \cap A \coloneqq \{S \cap A: S \in \MC\}$$ 
\end{defn}

\begin{ex}
Let $X$ be a set, $\MC \subset \MP(X)$ and $A \subset X$. Then 
$\sig(\MC) \cap A$ is a $\sig$-algebra on $A$. 
\end{ex}

\begin{proof}\
\begin{enumerate}
\item Clearly $\varnothing, A \in \sig(\MC) \cap A$.
\item Let $B \in \sig(\MC) \cap A$. Then there exists $S \in \sig(\MC)$ such that $B = S \cap A$. Then $S^c \in \sig(\MC)$. Thus $$A \setminus B = S^c \cap A \in \sig(\MC) \cap A$$
\item Let $(B_n)_{n \in \N} \subset \sig(\MC) \cap A$. Then for each $n \in \N$, there exists $S_n \in \sig(\MC)$ such that $B_n = S_n \cap A$. So $\bigcup\limits_{n \in \N}S_n \in \sig(\MC)$. Hence 
\begin{align*}
\bigcup_{n \in \N}(B_n) 
&= \bigcup_{n \in \N}(S_n \cap A) \\
&= \bigg( \bigcup_{n \in \N}S_n \bigg) \cap A \\
& \in \sig(\MC) \cap A
\end{align*}
\end{enumerate}
\end{proof}

\begin{ex}
Let $X$ be a set, $\MC \subset \MP(X)$ and $A \subset X$. Let $\sig_A(\MC \cap A)$ be the $\sig$-algebra on $A$ generated by $\MC \cap A$. Define $$\MG = \{S \subset X: S \cap A \in \sig_A(\MC \cap A)\}$$ 
Then $\MG$ is a $\sig$-algebra on $X$. 
\end{ex}

\begin{proof}
\begin{enumerate}
\item Clearly $\varnothing, X \in \MG$. 
\item Let $S \in \MG$. Then $S \cap A \in \sig_A(\MC \cap A)$. Hence $A \setminus (S \cap A)  = S^c \cap A \in \sig_A(\MC \cap A)$. So $S^c \in \MG$. 
\item Let $(S_n)_{n \in \N} \subset \MG$. Then for each $n \in \N$, $S_n \cap A \in \sig_A(\MC \cap A)$. Thus $$\bigg( \bigcup_{n \in \N} S_n \bigg) \cap A = \bigcup\limits_{n \in \N}(S_n \cap A) \in \sig_A(\MC \cap A)$$ Thus $\bigcup\limits_{n \in \N} S_n \in \MG$.
\end{enumerate}
\end{proof}

\begin{ex}
Let $X$ be a set, $\MC \subset \MP(X)$ and $A \subset X$. Then $$\sig(\MC) \cap A = \sig_A(\MC \cap A)$$
\end{ex}

\begin{proof}
Clearly $\MC \cap A \subset \sig(\MC) \cap A$. A previous exercise tells us that $\sig(\MC) \cap A$ is a $\sig$-algebra on $A$. Thus $\sig_A(\MC \cap A) \subset \sig(\MC) \cap A$. \vspace{3mm}\\ Conversely, from the previous exercise, we have that $\MG = \{S \subset X: S \cap A \in \sig_A(\MC \cap A)\}$ is a $\sig$-algebra on $X$. Clearly $\MC \subset \MG$. Then $\sig(\MC) \subset \MG$. The definition of $\MG$ implies that $\sig(\MC) \cap A \subset \sig_A(\MC \cap A)$. Hence $\sig(\MC) \cap A = \sig_A(\MC \cap A)$.
\end{proof}

\begin{defn}
Let $X$ be a set and $\MA$ be a $\sig$-algebra on $X$. Then $(X, \MA)$ is called a \textbf{measurable space}.
\end{defn}

























\section{Measures}

\subsection{Measures}

\begin{defn}
Let $(X, \MA)$ be a measurable space and $\mu:\MA \rightarrow \RG$. Then $\mu$ is said to be a \textbf{measure} on $(X, \MA)$ if 
\begin{enumerate}
\item there exists $A \in \MA$ such that $\mu(A)< \infty$
\item for each $(A_n)_{n \in \N} \subset \MA$. If $(A_n)_{n \in \N}$ is disjoint, then $$\mu\bigg(\bigcup_{n \in \N}A_n \bigg) = \sum_{n \in \N}\mu(A_n)$$
\end{enumerate}
\end{defn}

\begin{defn}
Let $(X,\MA)$ be a measurable space and $\mu$ a measure on $(A, \MA)$. Then $(A, \MA, \mu)$ is called a \textbf{measure space}. 
\end{defn}

\begin{ex}
Let $(X, \MA, mu)$ be a measure space. Then 
\begin{enumerate}
\item (monotonicity): for each $A,B \in \MA$, if $A \subset B$, then $\mu(A) \leq \mu(B)$.
\item (subadditivity): for each $(A_n)_{n \in \N} \subset \MA$, $$\mu \bigg( \bigcup_{n \in \N} A_n \bigg) \leq \sum_{n \in \N}\mu(A_n)$$
\item (continuity from below): for each $(A_n)_{n \in \N} \subset \MA$, if for each $n \in \N$, $A_n \subset A_{n+1}$, then $$\mu\bigg(\sup_{n \in \N} A_n\bigg) = \sup_{n \in \N}\mu(A_n)$$
\item (continuity from above): for each $(A_n)_{n \in \N} \subset \MA$, if for each $n \in \N$, $ A_{n+1} \subset A_n$ and $\mu(A_1) < \infty$, then $$\mu\bigg(\inf_{n \in \N} A_n\bigg) = \inf_{n \in \N}\mu(A_n)$$
\end{enumerate}

\end{ex}

\begin{proof}\
\begin{enumerate}
\item Let $A, B \in \MA$. Suppose that $A \subset B$. Then 
\begin{align*}
\mu(B) 
&= \mu\bigg((B \cap A) \cup (B \cap A^c)\bigg)\\
&= \mu(B \cap A) + \mu(B \cap A^c)\\
&= \mu(A) + \mu(B \cap A^c)\\
&\geq \mu(A)
\end{align*}
\item Let $(A_n)_{n \in \N} \subset \MA$. Define $B_1 = A_1$ and for $n \geq 2$, $B_n = A_n \setminus \bigg( \bigcup\limits_{k=1}^{n-1}A_k \bigg)$. Then $(B_n)_{n \in \N} \subset \MA$, $\bigcup\limits_{n \in \N}B_n = \bigcup\limits_{n \in \N}A_n $, $(B_n)_{n \in \N}$ disjoint and for each $n \in \N$, $B_n \subset A_n$. Thus 
\begin{align*}
\mu\bigg(\bigcup_{n \in \N}A_n \bigg)
&= \mu\bigg(\bigcup_{n \in \N}B_n \bigg)\\
&= \sum_{n \in \N}\mu(B_n) \\
&\leq \sum_{n \in \N}\mu(A_n)
\end{align*} 
\item Let $(A_n)_{n \in \N} \subset \MA$. Suppose that for each $n \in \N$, $A_n \subset A_{n+1}$. Then for each $n \in \N$, $\mu(A_n) \leq \mu(A_{n+1})$ and $\lim\limits_{n \rightarrow \infty}\mu(A_n) = \sup\limits_{n \in \N} \mu(A_n)$. Recall that $\sup\limits_{n \in \N}A_n = \bigcup\limits_{n \in \N}A_n$. 
Define $B_1 = A_1$ and for $n \geq 2$, $B_n = A_n \setminus A_{n-1}$. Then $(B_n)_{n \in \N} \subset \MA$, $(B_n)_{n \in \N}$ is disjoint, $\bigcup\limits_{n \in \N}A_n = \bigcup\limits_{n \in \N}B_n$ and for each $n \in \N$, $\bigcup\limits_{n=1}^{k}B_n = A_k$. Then 
\begin{align*}
\mu\bigg(\sup_{n \in \N}A_n \bigg)
&= \mu\bigg(\bigcup_{n \in \N}A_n \bigg) \\
&= \mu\bigg(\bigcup\limits_{n \in \N}B_n \bigg)\\
&= \sum_{n \in \N} \mu(B_n) \\
&= \lim_{k \rightarrow \infty} \sum_{n=1}^k \mu(B_n) \\
&= \lim_{k \rightarrow \infty} \mu \bigg(\bigcup_{n=1}^k B_n \bigg) \\
&= \lim_{k \rightarrow \infty} \mu(A_k) \\
&= \sup_{n \in \N} \mu(A_n)
\end{align*}
\item Let $(A_n)_{n \in \N} \subset \MA$. Suppose that for each $n \in \N$, $ A_{n+1} \subset A_n$ and $\mu(A_1) < \infty$. Then for each $n \in \N$ $\mu(A_{n+1}) \leq \mu(A_n) \leq \mu(A_1) < \infty$ and the arithmetic that follows is well defined. Recall that $\inf\limits_{n \in \N}A_n = \bigcap\limits_{n \in \N}A_n$. For each $n\in \N$, define $B_n = A_1 \cap A_n$. Then for each $n \in \N$, $B_n \subset B_{n+1}$ and
\begin{align*}
\sup\limits_{n \in \N}B_n 
&= \bigcup\limits_{n \in \N}B_n \\
&= A_1 \setminus \ \bigcap\limits_{n \in \N}A_n \\
&= A_1 \setminus \inf_{n \in \N}A_n
\end{align*}  
So $(3)$ implies that
\begin{align*}
\sup_{n \in \N}\mu(B_n) 
&= \mu \bigg(\sup\limits_{n \in \N}B_n \bigg)\\
&= \mu \bigg(A_1 \setminus \inf_{n \in \N}A_n \bigg)\\
&= \mu(A_1) - \mu\bigg(\inf_{n \in \N}A_n \bigg)\\
\end{align*}
On the other hand, 
\begin{align*}
\sup_{n \in \N}\mu(B_n)
&= \sup_{n \in \N}\mu(A_1 \setminus A_n)\\
&= \sup_{n \in \N} \bigg[ \mu(A_1) - \mu(A_n) \bigg]\\
&= \mu(A_1) - \inf_{n \in \N}\mu(A_n)
\end{align*}
Therefore $$\mu\bigg(\inf_{n \in \N}A_n \bigg) = \inf_{n \in \N}\mu(A_n)$$
\end{enumerate}
\end{proof}

\begin{ex}
Let $(X, \MA, \mu)$ be a measure space, $(A_n)_{n \in \N} \subset \MA$ and $A \in \MA$. Then 
\begin{enumerate}
\item $\mu \bigg(\liminf\limits_{n \rightarrow \infty} A_n \bigg) \leq \liminf\limits_{n \rightarrow \infty} \mu(A_n)$
\item If $\mu\bigg(\sup\limits_{n \in \N}A_n \bigg) < \infty$, then $\limsup\limits_{n \rightarrow \infty}\mu(A_n) \leq \mu \bigg( \liminf\limits_{n \rightarrow \infty}A_n\bigg) $
\end{enumerate} 
\end{ex}

\begin{proof}\
\begin{enumerate}
\item Since $\bigg(\inf_{k \geq n}A_k \bigg)_{n \in \N}$ is an increasing sequence and for each $n \in \N$ $\inf\limits_{k \geq n}A_k \subset A_n$, we have that
\begin{align*}
\mu\bigg(\liminf\limits_{n \rightarrow \infty} A_n \bigg) 
& = \mu \bigg[\sup_{n \in \N} \bigg(\inf\limits_{k \geq n} A_k \bigg) \bigg] \\
&= \sup_{n \in \N}  \mu\bigg( \inf_{k \geq n}A_k\bigg)\\
& = \liminf_{n \rightarrow \infty} \mu\bigg( \inf_{k \geq n}A_k\bigg) \\
& \leq \liminf_{n \rightarrow \infty}  \mu(A_n) \\
\end{align*}
\item Since $\mu\bigg(\sup\limits_{ \geq 1}A_k \bigg) < \infty$, $\bigg(\sup\limits_{k \geq n} \bigg)_{n \in \N}$ is a decreasing and for each $n \in \N$, $A_n \subset \sup_{k \geq n}A_n$, we have that 
\begin{align*}
\mu \bigg(\limsup_{n \rightarrow \infty} A_n 
\bigg) 
&= \mu \bigg[\inf_{n \in \N} \bigg(\sup_{k  \geq n} A_k \bigg) \bigg] \\
&= \inf_{n \in \N}\mu\bigg( \sup_{k \geq n} A_k\bigg) \\
& = \limsup_{n \rightarrow \infty} \mu \bigg( \sup_{k \geq n}A_k \bigg) \\
& \geq \limsup_{n \rightarrow \infty} \mu ( A_n )\
\end{align*} 
\end{enumerate}
\end{proof}

\begin{ex}
Let $(X, \MA, \mu)$ be a measure space, $(A_n)_{n \in \N} \subset \MA$ and $A \in \MA$. Suppose that $\mu\bigg(\sup\limits_{n \in \N}A_n\bigg) < \infty$. Then $A_n \rightarrow A$ implies that $\mu(A_n) \rightarrow \mu(A)$. 
\end{ex} 

\begin{proof}
Suppose that $A_n \rightarrow A$. Then the previous exercise tells us that 
\begin{align*}
\mu(A)
&= \mu\bigg(\liminf\limits_{n \rightarrow \infty}A_n \bigg)\\
& \leq \liminf_{n \rightarrow \infty} \mu(A_n)\\
& \leq \limsup_{n \rightarrow \infty}\mu (A_n) \\
& \leq \mu( \limsup_{n \rightarrow \infty} A_n) \\
&= \mu (A)
\end{align*}

Thus $\mu(A) = \limsup\limits_{n \rightarrow \infty}\mu(A_n) = \liminf\limits_{n \rightarrow \infty}\mu(A_n)$ and $\mu(A_n) \rightarrow \mu(A)$
\end{proof}


















\subsection{Outer Measures}

\begin{defn}
Let $X$ be a set and $\mu* : \MP(X) \rightarrow [0, \infty]$. Then $\mu^*$ is said to be an \textbf{outer measure on X} if 
\begin{enumerate}
\item $\mu^*(\varnothing) = 0$
\item for each $A,B \subset X $, if $A \subset B$, then $\mu^*(A) \leq \mu^*(B)$.
\item for each $(A_n)_{n \in \N} \subset \MP(X)$, $$\mu^*\big(\bigcup\limits_{n \in \N} A_n\big) \leq \sum\limits_{n \in \N}\mu^*(A_n) $$
\end{enumerate}
\end{defn}

\begin{thm}\textbf{Construction of Outer Measures:} \\
Let $X$ be a set and $\ME \subset \MP(X)$ and $\rho: \ME \rightarrow [0, \infty]$. Suppose that $\varnothing, X \in \ME$ and $\rho(\varnothing) = 0$. Define $\mu^*:\MP(X) \rightarrow [0, \infty]$ by $$\mu^*(A) = \inf \bigg \{\sum_{n \in \N}\rho(E_n): (E_n)_{n \in \N} \subset \ME \text{ and }A \subset \bigcup_{n \in \N}E_n \bigg \}$$ Then $\mu^*$ is an outer measure on $X$.
\end{thm}

\begin{note}
In particular, for each $A \in \ME$, $\mu^*(A) = \rho(A)$.
\end{note}

\begin{defn}
Let $X$ be a set and $\ME \subset \MP(X)$ and $\rho: \ME \rightarrow [0, \infty]$. Suppose that $\varnothing, X \in \ME$ and $\rho(\varnothing) = 0$. Let $\mu^*$ be the outer measure on $X$ defined as in the last theorem. Then $\mu^*$ is called the \textbf{outer measure on $X$ induced by $\rho$}.
\end{defn}

\begin{defn}
Let $X$ be a set, $\mu^*$ an outer measure on $X$ and $A \subset $X. Then $A$ is said to be $\mu^*$-\textbf{outer measurable} if for each $E \subset X$, $$\mu^*(E) = \mu^*(E \cap A) + \mu^*(E \cap A^c)$$ 
\end{defn}

\begin{thm}
Let $X$ be a set and $\mu^*$ an outer measure on $X$. Define $\MA = \{A \subset X: A \text{ is }\mu^*\text{-measurable}\}$. Then $\MA$ is a $\sig$-algebra on $X$ and $\mu^*|_{\MA}$ is a complete measure on $(X, \MA)$.
\end{thm}

\begin{defn}
Let $X$ be a set, $\MA_0$ be an algebra on $X$ and $\mu_0:\MA_0 \rightarrow [0, \infty]$. Then $\mu_0$ is said to be a \textbf{premeasure on $(X,\MA_0)$} if 
\begin{enumerate}
\item there exists $A \in \MA_0$ such that $\mu_0(A)< \infty$
\item for each $(A_n)_{n \in \N} \subset \MA_0$, if $(A_n)_{n \in \N}$ is disjoint and $\bigcup\limits_{n \in \N}A_n \in \MA_0$, then $$\mu_0(\bigcup_{n\in \N}A_n) = \sum_{n \in \N}\mu_0(A_n)$$
\end{enumerate}
\end{defn}

\begin{note}
The same reasoning applied to measures shows that $\mu_0(\varnothing) = 0$.
\end{note}

\begin{thm}
Let $X$ be a set, $\MA_0$ an algebra on $X$, $\mu_0$ a premeasure on $(X,\MA_0)$ and $\mu^*$ the outer measure on $X$ induced by $\mu_0$. Put $\MA = \sig(\MA_0)$. If $\mu_0$ is $\sig$-finite, then there exists a unique measure $\mu$ on $(X, \MA)$ such that $\mu|_{\MA_0} = \mu^*|_{\MA_0} = \mu_0$. 
\end{thm}























\subsection{Product Measures}

\begin{defn}
Let $(X,\MA, \mu), (Y,\MB, \nu)$ be $\sig$-finite measurable spaces. Put $\ME = \{A \times B: A \in \MA \text{ and } B \in \MB\}$. Then $\ME$ is an elementary family and thus $\MM_0 = \{\bigcup_{i =1}^n M_i: (M_i)_{i=1 }^n \subset \ME \text{ are disjoint}\}$ is an algebra on $X \times Y$. We define $\pi_0: \MM_0 \rightarrow \RG$ by $$\pi_0\bigg(\bigcup_{i=1}^n A_i \times B_i \bigg) = \sum_{i=1}^n\mu(A_i)\nu(B_i)$$ Then $\pi_0$ is a premeasure on $(X \times Y, M_0)$. Since $\MA \otimes \MB = \sig(\MM_0)$, we define the \textbf{product measure}, $\mu \times \nu$ on $(X \times Y, \MA \otimes \MB)$, to be the unique extension of $\pi_0$ to $\MA \otimes \MB$. The existence of which is guaranteed by a theorem in the previous section. In particular,

\begin{align*}
\mu \times \nu(E) 
&= \inf \bigg\{\sum_{n \in \N}\pi_0(E_i): (E_i)_{i \in \N} \subset \MM_0 \text{ and } E \subset \bigcup_{i \in \N} E_i \bigg\}\\
&= \inf \bigg\{\sum_{n \in \N}\mu(A_i)\nu(B_i): (A_i \times B_i)_{i \in \N} \subset \ME \text{ and } E \subset \bigcup_{i \in \N} A_i \times B_i \bigg \}
\end{align*}
\end{defn}

\section{Integration}

\subsection{Measurable Functions}

\begin{defn}
Let $(X,\MA)$ and $(Y, \MB)$ be measurable spaces and $f:X \rightarrow Y$. Then $f$ is said to be $\MA$-$\MB$ \textbf{measurable} if for each $B \in \MB$, $f^{-1}(B) \in \MA$. When $(Y, \MB) = (\R, \MB(\R))$ we say that $f$ is $\MA$-\textbf{measurbale}. If $(Y,\MB) = (\R, \MB(\R))$ and $(X,\MA) = (\R, \MB(\R))$ or $(\R, \ML)$, then we say that $f$ is \textbf{Borel measurable} or \textbf{Lebsgue measurbale} respectively.
\end{defn}

\begin{ex}
Let $(X,\MA), (Y,\MB)$ be measurable spaces and $f: X \rightarrow Y$. Then 
\begin{enumerate}
\item $\{B \subset Y: f^{-1}(B) \in \MA\}$ is a $\sig$-algebra on $Y$
\item $\{f^{-1}(B): B \in \MB\}$ is a $\sig$-algebra on $X$
\end{enumerate}
\end{ex}

\begin{proof}\
\begin{enumerate}
\item Define $\ML = \{B \subset Y: f^{-1}(B) \in \MA\}$.  Clearly $Y \in \ML$. Let $B \in \ML$. Then $f^{-1}(B) \in \MA$. Hence $$f^{-1}(B^c) = (f^{-1}(B))^c \in \MA$$ Thus $B^c \in \ML$. Now, let $(B_n)_{n \in \N} \subset \ML$. Then for each $n \in \N$, $f^{-1}(B_n) \in \MA$. Thus $$f^{-1}\bigg(\bigcup_{n \in \N} B_n \bigg) = \bigcup_{n \in \N} f^{-1}(B_n) \in \MA$$ Hence $\bigcup\limits_{n \in \N} B_n \in \ML$.
\item Similar to (1).
\end{enumerate}
\end{proof}

\begin{ex}
Let $(X,\MA)$ and $(Y, \MB)$ be measurable spaces. Suppose that there exists $\ME \subset Y$ such that $\sig(\ME) = \MB$. Let $f:X \rightarrow Y$. Then $f$ is $\MA$-$\MB$ measurable iff for each $B \in \ME$, $f^{-1}(B) \in \MA$.
\end{ex}

\begin{proof}
By definition, if $f$ is $\MA$-$\MB$ measurable, then for each $B \in \ME$, $f^{-1}(B) \in \MA$. Conversely, suppose that for each $B \in \ME$, $f^{-1}(B) \in \MA$. The previous lemma tells us that $\ML = \{B \subset Y: f^{-1}(B) \in \MA\}$ is a $\sig$-algebra on $Y$. Since $\ME \subset \ML$, we have that $\MB = \sig(\ME) \subset \ML$. So $f$ is $\MA$-$\MB$ measurable.
\end{proof}

\begin{ex}
Let $X,Y$ be sets, $f:X \rightarrow Y$ and $\ME \subset \MP(Y)$. Then $\sig(f^{-1}(\ME)) = f^{-1}(\sig(\ME))$. 
\end{ex}

\begin{proof}
Clealy $f^{-1}(\ME) \subset f^{-1}(\sig(\ME))$. Since $f^{-1}(\sig(\ME))$ is a $\sig$-algebra, we have that $\sig(f^{-1}(\ME)) \subset f^{-1}(\sig(\ME))$. Since $f^{-1}(\ME) \subset f^{-1}(\sig(\ME))$, the previous exercise tells us that $f$ is $f^{-1}(\sig(\ME))$-$\sig(\ME)$ measurable. Then $f^{-1}(\sig(\ME)) \subset f^{-1}(\sig(\ME))$. So $\sig(f^{-1}(\ME)) = f^{-1}(\sig(\ME))$.  
\end{proof}

\begin{ex}
Let $(X_1,\MT_1), (X_2,\MT_2)$ be topological spaces and $f: X \rightarrow Y$. If $f$ is continuous, then $f$ is $\MB(X)$-$\MB(Y)$ measurable.
\end{ex}

\begin{proof}
Recall that $\MB(Y) = \sig(\MT_2)$ and continuity tells us that for each $U \in \MT_2$, $f^{-1}(U) \in \MT_1 \subset \MB(X)$. 
\end{proof}

\begin{defn}
Let $X$ be a set and $f:X \rightarrow \C$. Then $f$ is said to be \textbf{simple} if $f(X)$ is finite.
\end{defn}

\begin{defn}
Let $(X,\MA)$ be a measurable space. We define $S^+(X,\MA) = \{f:X \rightarrow \Rg: f \text{ is simple, measurable}\}$ and $S(X,\MA) = \{f: X \rightarrow \C: f \text{ is simple, measurable}\}$
\end{defn}

\begin{thm}
Let $(X, \MA)$ be a measurable space. Then 
\begin{enumerate}
\item If $f: X \rightarrow \RG$ is measurable, then there exists a sequence $(\phi_n)_{n \in \N} \subset S^+$ such that for each $n \in \N$, $\phi_n \leq \phi_{n+1} \leq f$ and $\phi_n \rightarrow f$ pointwise and $\phi_n \rightarrow f$ uniformly on any set on which $f$ is bounded.

\item If $f: X \rightarrow \C$ is measurable, then there exists a sequence $(\phi_n)_{n \in \N} \subset S$ such that for each $n \in \N$, $|\phi_n| \leq |\phi_{n+1}| \leq |f|$ and $\phi_n \rightarrow f$ pointwise and $\phi_n \rightarrow f$ uniformly on any set on which $f$ is bounded.
\end{enumerate}
\end{thm}

\subsection{Integration of Nonnegative Functions}

\begin{defn}
Let $(X, \MA, \mu)$ be a measure space. Define $$L^{+}(X, \MA, \mu) = \{f:X \rightarrow \RG : f \text{ is measurable}\}$$ We will typically just write $L^{+}$.
\end{defn}

\begin{thm}\textbf{Monotone Convergence Theorem:} 
Let $(f_n)_{n \in \N} \subset L^+$. Suppose that for each $n \in \N$, $f_n \leq f_{n+1}$. Then $$\sup_{n \in \N} \int f_n = \int \sup_{n \in \N} f_n$$.
\end{thm}

\begin{ex}
Let $\mu_1, \mu_2$ be measures on $(X, \MA)$ and $f \in L^+$. Then $$\int f d (\mu_1 + \mu_2) = \int f d\mu_1 + \int f d\mu_2$$.  
\end{ex}

\begin{proof}
Suppose that $f$ is simple. Then there exist $(a_n)_{i=1}^n \subset \Rg$ and $(E_i)_{i=1}^n \subset \MA$ such that $f = \sum\limits_{i =1}^n a_i \chi_{E_i}$. Then 
\begin{align*}
\int f d(\mu_1 + \mu_2) 
&= \sum\limits_{i =1}^n a_i (\mu_1 + \mu_2)(E_i)\\
&= \sum\limits_{i =1}^n a_i (\mu_1(E_i) + \mu_2(E_i))\\
&= \sum\limits_{i =1}^n a_i \mu_1(E_i) + a_i \mu_2(E_i)\\
&= \int f d\mu_1 + \int f d\mu_2
\end{align*}

Now for a general $f$, choose $(\phi_n)_{n \in \N} \subset S^+$ such that $\phi_n \rightarrow f$ pointwise and for each $n \in \N$, $\phi_n \leq \phi_{n+1} \leq f$. Then monotone convergence tells us that 
\begin{align*}
\int f d(\mu_1 + \mu_2) 
&= \limn \int \phi_n d(\mu_1 + \mu_2)\\
&= \limn \int \phi_n d \mu_1 + \limn \int \phi_n d \mu_2 \\
&= \int f d \mu_1 + \int f d \mu_2
\end{align*}

\end{proof}


\begin{ex}
Let $\mu_1, \mu_2$ be measures on $(X,\MA)$. Suppose that $\mu_1 \leq \mu_2$. Then for each $f \in L^+$, $$\int f d\mu_1 \leq \int f d\mu_2$$
\end{ex}

\begin{proof}
First suppose that $f$ is simple. Then there exist $(a_n)_{i=1}^n \subset \Rg$ and $(E_i)_{i=1}^n \subset \MA$ such that $f = \sum\limits_{i =1}^n a_i \chi_{E_i}$. Then 
\begin{align*}
\int f d\mu_1 
&= \sum\limits_{i =1}^n a_i \mu_1(E_i)\\
& \leq \sum\limits_{i =1}^n a_i \mu_2(E_i)\\
&= \int f d \mu_2
\end{align*} 

for general $f$, 
\begin{align*}
\int f d\mu_1 
&= \sup_{\substack{s \in S^+\\s \leq f}} \int s d \mu_1 \\
& \leq \sup_{\substack{s \in S^+\\s \leq f}} \int s d\mu_2\\
&= \int f d\mu_2
\end{align*}

\end{proof}

\begin{thm}{Fatou's Lemma}
Let $(f_n)_{n \in \N} \subset L^+$. Then $$\int \limfn f_n \leq \limfn \int f_n.$$
\end{thm}

\begin{thm}
Let $(f_n)_{n \in \N} \subset L^+$. Then $$\int \sum_{n \in \N} f_n= \sum_{n \in \N} \int f_n.$$
\end{thm}

\begin{ex}
Let $f \in L^+$ and suppose that $\int f < \infty$. Put $N = \{x \in X: f(x) = \infty\}$ and $S = \{x \in X: f(x) > 0\}$. Then $\mu(N) = 0$ and $S$ is $\sig$-finite.
\end{ex}

\begin{proof}
Suppose that $\mu(N) > 0$. Define $f_n = n \chi_{N} \in L^+$. Then for each $n \in \N$, $f_n \leq f_{n+1} \leq f$ on $N$. So 
\begin{align*}
\int f 
&\geq \int_N f\\ 
&= \lim\limits_{n \rightarrow \infty} \int_N f_n\\ 
&= \lim\limits_{n \rightarrow \infty} n\mu(N)\\
&= \infty \text{, a contradiction.}
\end{align*}
Hence $N$ is a null set. Now, put $S_n = \{x \in X: f(x)>1/n\}$. Then $S = \bigcup \limits_{n \in \N}S_n$. Suppose that there exists some $n \in \N$ such that $\mu(S_n) = \infty$. Then 
\begin{align*}
\int f 
&\geq \int_{S_n} f \\
&\geq \frac{1}{n}\mu(S_n) \\
&= \infty \text{, a contradiction.}
\end{align*}

So for each $n \in \N$, $\mu(S_n) < \infty$ and $S$ is $\sig$-finite.

\end{proof}

\begin{ex}
Let $f \in L^+$. Then $f =0$ a.e. iff for each $E \in \MA$, $\int_E f =0$.
\end{ex}

\begin{proof}
$f = 0$ a.e. implies that for each $E \in \MA$, $\int_E f =0$ is clear. Conversely, suppose that for each $E \in \MA$, $\int_E f$ = 0. For $n \in \N$ put $N_n = \{x \in X: f(x) > 1/n\}$ and define $N = \{x \in X: f(x)>0\}$. So $N = \bigcup\limits_{n \in \N} N_n$. Let $n \in \N$. Then our assumption tells us that 
\begin{align*}
0 
&= \int_{N_n} f \\
& \geq \frac{1}{n}\mu(N_n)\\
& \geq 0.
\end{align*} 

Hence for each $n \in \N$, $\mu(N_n) = 0$. Thus $\mu(N) = 0$ and $f =0$ a.e. as required.

\end{proof}

\begin{ex}

Let $(f_n)_{n \in \N} \subset L^+$ and $f \in L^+$. Suppose that $f_n \xrightarrow{\text{p.w.}} f$, $\lim \limits_{n \rightarrow \infty} \int f_n = \int f$ and $\int f < \infty$. Then for each $E \in \MA$, $\lim \limits_{n \rightarrow \infty} \int_E f_n = \int_E f$. This result may fail to be true if $\int f = \infty$

\end{ex}

\begin{proof}

Let $E \in \MA$. By Fatou's lemma, $\int_E f \leq \limfn \int_E f_n$. Note that since $\int f < \infty$, we have that $\int_{E^c} f \leq \int f < \infty$. Thus we may write
\begin{align*}
\int_E f 
&= \int f - \int_{E^c} f\\
&\geq \int f - \limfn \int_{E^c} f_n\\
&= \int f - \limfn \bigg(\int f_n - \int_{E} f_n\bigg)\\
&= \int f - \int f  + \limpn \int_{E} f_n\\
&= \limpn \int_E f_n.
\end{align*}

Hence $$\limpn \int_E f_n \leq \int_E f \leq \limfn \int_E f_n$$ and therefore $$\limn \int_E f_n = \int_E f.$$ 

If we drop the assumption that $\int f < \infty$, then the result would fail to be true for the functions $f = \infty \chi_{(0,1)}$ and $ f_n = \infty \chi_{(0,1)} + n \chi_{(1,1+1/n)}$. Here $f_n \xrightarrow{\text{p.w.}} f$, $\limn \int f_n = \int f = \infty$ and $\limn \int_{(1,\infty)} f_n = 1$ while $\int_{(1,\infty)} f = 0$.  

\end{proof}

\begin{ex}
Let $f \in L^+$. Define $\lam: \MA \rightarrow \RG$ by $\lam(E) = \int_E f d\mu$ for $E \in \MA$
Then $\lam$ is a measure on $(X, \MA)$ and for each $g \in L^+$, $\int g d\lam = \int g f d\mu$.
\end{ex}

\begin{proof}
Clearly $\lam(\varnothing) = 0$. Let $(A_j)_{j \in \N} \subset \MA$ and suppose that for each $i, j \in \N$, if $i \neq j$, then $A_i \cap A_j = \varnothing$. For now, suppose that $f$ is simple. Then there exist $E_1, E_2, \cdots, E_n \in \MA$ and  $a_1, a_2, \cdots, a_n \in \Rg$ such that $f = \sum\limits_{i=1}^n a_i \chi_{E_i}$.  Then 
\begin{align*}
\lam\bigg(\bigcup_{j \in \N} A_j\bigg) 
&= \int_{\bigcup_{j \in \N} A_j} f\\
&= \sum_{i = 1} ^n a_i\mu\bigg(E_i \cap \bigg(\bigcup_{j \in \N} A_j\bigg)\bigg)\\
&= \sum_{i = 1} ^n a_i\mu\bigg(\bigcup_{j \in \N} E_i \cap A_j\bigg)\\
&= \sum_{i = 1} ^n a_i \sum_{j \in \N} \mu(E_i \cap A_j)\\
&= \sum_{j \in \N} \sum_{i = 1} ^n a_i \mu(E_i \cap A_j)\\
&= \sum_{j \in \N} \int_{A_j} f\\
&= \sum_{j \in \N} \lam(A_j)
\end{align*} 
Hence $\lam$ is a measure on $(X, \MA)$. Now, for a general f, there exist $(\phi_n)_{n \in \N} \subset L^+$ such that for each $n \in \N$, $\phi_n$ is simple, $\phi_n \leq \phi_{n+1} \leq f$ and $\phi_n \xrightarrow{\text{p.w.}} f$. Put $A = \bigcup_{j \in \N}A_j$ and define the measures $\lam_n$ by $\lam_n(E) = \int_E \phi_n$. Note that we may define a monotonically increasing sequence of functions $g_n: \N \rightarrow \RG$ by $g_n(j) = \int_{A_j} \phi_n$. Using monotone convergence three times and a nice application of the counting measure on $\N$, we may write

\begin{align*}
\lam(A) 
&= \int_A f\\
&= \limn \int_A \phi_n\\
&= \limn \sum_{j \in \N} \int_{A_j} \phi_n\\
&= \sum_{j \in \N} \limn \int_{A_j} \phi_n \hspace{4mm} \text{(by the above)}\\
&= \sum_{j \in \N} \int_{A_j} f\\
&= \sum_{j \in \N} \lam(A_j).
\end{align*} 

Hence $\lam$ is a measure on $(X, \MA)$. Let $g \in L^+$. First assume that $g$ is simple. Then there exist $E_1, E_2, \cdots, E_n \in \MA$ and  $a_1, a_2, \cdots, a_n \in \Rg$ such that $g = \sum\limits_{i=1}^n a_i \chi_{E_i}$.
In this case, we have that 
\begin{align*}
\int g d\lam 
&= \sum_{i=1}^n a_i \lam(E_i)\\
&= \sum_{i=1}^n a_i \int_{E_i} f d\mu\\
&= \int \bigg(\sum_{i=1}^n a_i\chi_{E_i} \bigg) f d\mu\\
&= \int gf d\mu.
\end{align*}

Now for a general $g \in L^+$, there exist $(\psi_n)_{n \in \N} \subset L^+$ such that for each $n \in \N$, $\psi_n$ is simple, $\psi_n \leq \psi_{n+1} \leq f$ and $\psi_n \xrightarrow{\text{p.w.}} g$. Monotone convergence then gives us 
\begin{align*}
\int g d\lam 
&= \limn \int \psi_n d\lam\\
&= \limn \int \psi_n f d\mu\\
&= \int g f d\mu \text{ as required.}
\end{align*}
\end{proof}

\begin{ex}
Let $(f_n)_{n \in \N} \subset L^+$ and $f \in L^+$. Suppose that for each $n \in \N$, $f_n \geq f_{n+1}$, $f_n \xrightarrow{\text{p.w.}} f$ and $\int f_1 < \infty$. Then $\limn \int f_n = \int f$.
\end{ex}

\begin{proof}
First we note that since $\int f_1 < \infty$, $f_1 < \infty$ a.e., for each $n \in \N$, $f_1 - f_n$ and $\int f_1 - \int f_n$ are well defined and $\int f_n \leq \int f_1 < \infty$. Also, for $n \in \N$, $f_1 -f_n \in L^+$. So we may write 
\begin{align*}
\int (f_1 - f_n) 
&= \int (f_1 - f_n)  + \int f_n - \int f_n\\
&= \int [(f_1 - f_n) + f_n] - \int f_n\\
&= \int f_1 - \int f_n
\end{align*}

Put $g_n = f + (f_1 - f_n)$. Then $g_n \in L^+$, for each $n \in \N$, $g_n \leq g_{n+1}$ and $g_n \xrightarrow{\text{p.w.}} f_1$. Monotone convergence tells us that 
\begin{align*}
\int f_1 
&= \limn \int g_n\\
&= \limn \bigg[\int f + (f_1-f_n)\bigg]\\
&= \limn \bigg[ \int f + \int (f_1-f_n)\bigg] \\
&= \limn \bigg[ \int f + \int f_1- \int f_n\bigg] 
\end{align*}

Since $\limn \int f$ and $\limn \int f_1$ exist, $\limn \int f_n = \int f$ as required.  

\end{proof}

\subsection{Integration of Complex Valued Functions}

\begin{defn}
Let $f:X \rightarrow \C$ be measurable. Then $f$ is said to be \textbf{integrable} if $$\int |f| d\mu < \infty$$
\end{defn}

\begin{defn}
Let $(X, \MA, \mu)$ be a measure space. Define $L^1(X, \MA, \mu) = \{f:X \rightarrow \C : f \text{ is measurable and } \int |f| < \infty \}$
\end{defn}

\begin{lem}
Let $f:X \rightarrow \R$ be measurable. Then $f$ is integrable iff $f^+$ and $f^-$ are integrable. 
\end{lem}

\begin{proof}
$f^+,f^- \leq |f| = f^+ + f^-$
\end{proof}

\begin{defn}
Let $f:X \rightarrow \R$ be measurable. Then $f$ is said to be \textbf{extended integrable} if $$\int f^+ d\mu  < \infty \text{ or } \int f^- d\mu < \infty$$
\end{defn}

\begin{lem}
Let $f:X \rightarrow \R$ be measurable. Then $f$ is integrable iff $Re(f)$ and $Im(f)$ are integrable.
\end{lem}

\begin{proof}
$|Re(f)|, |Im(f)| \leq |f| \leq |Re(f)| + |Im(f)|$
\end{proof}

\begin{thm}{Dominated Convergence}
Let $(f_n)_{n \in \N} \subset L^1$, $f$ measurable and $g \in L^1$. Suppose that $f_n \xrightarrow{\text{a.e.}} f$ and for each $n \in \N$, $|f_n| \leq g_n$. Then $f \in L^1$ and $\int f_n \rightarrow \int f$. 
\end{thm}

\begin{ex}
Let $\mu_1, \mu_2$ be measures on $(X, \MA)$. Then
\begin{enumerate}
\item $L^1(\mu_1 + \mu_2) = L^1(\mu_1) \cap L^1(\mu_2)$

\item for each $f \in L^1(\mu_1 + \mu_2)$, we have that $$\int f d(\mu_1 + \mu_2) = \int f d \mu_1 + \int f d\mu_2$$
\end{enumerate}

\end{ex}

\begin{proof}
\begin{enumerate}
\item The firt part is clear since similar exercise from the section on nonnegative funtions tells us that $$\int |f| d(\mu_1 + \mu_2) = \int |f| d \mu_1 + \int |f| d\mu_2$$


\item Suppose that $f$ is simple. Then there exist $(a_n)_{i=1}^n \subset \C$ and $(E_i)_{i=1}^n \subset \MA$ such that $f = \sum\limits_{i =1}^n a_i \chi_{E_i}$. Then 
\begin{align*}
\int f d(\mu_1 + \mu_2) 
&= \sum\limits_{i =1}^n a_i (\mu_1 + \mu_2)(E_i)\\
&= \sum\limits_{i =1}^n a_i (\mu_1(E_i) + \mu_2(E_i))\\
&= \sum\limits_{i =1}^n a_i \mu_1(E_i) + a_i \mu_2(E_i)\\
&= \int f d\mu_1 + \int f d\mu_2
\end{align*}

Now for general $f$, choose $(\phi_n)_{n \in \N} \subset S$ such that $\phi_n \rightarrow f$ pointwise and for each $n \in \N$, $|\phi_n| \leq |\phi_{n+1}| \leq |f|$. Then dominated convergence tells us that 
\begin{align*}
\int f d(\mu_1 + \mu_2) 
&= \limn \int \phi_n d(\mu_1 + \mu_2)\\
&= \limn \int \phi_n d \mu_1 + \limn \int \phi_n d \mu_2 \\
&= \int f d \mu_1 + \int f d \mu_2
\end{align*}

\end{enumerate}
\end{proof}

\begin{thm}
Let $(f_n)_{n \in \N} \subset L^1$. Suppose that $$\sum_{n \in \N} \int |f_n| < \infty.$$ Then after redefinition on a set of measure zero, $\sum_{n \in \N}f_n \in L^1$ and $$\int \sum_{n \in \N}f_n = \sum_{n \in \N} \int f_n$$
\end{thm}

\begin{thm}
Let $f \in L^1$. Then for each $\ep > 0$, there exists $\phi \in L^1$ such that $\phi$ is simple and $\int |f - \phi| < \ep$. 
\end{thm}

\begin{ex}{Generalized Fatou's Lemma:}
Let $(f_n)_{n \in \N}$ be a sequence of measurable real valued functions. Suppose that there exists $g \in L^1$ such that $g \geq 0$ and for each $n \in \N$, $f_n \geq -g$. Then $ \int \limfn f_n \leq \limfn \int f_n$. What is the analogue of Fatou's lemma for measurable, real valued functions that are appropriately bounded above?  
\end{ex}

\begin{proof}
First note that for each $n \in \N$, $\int f_n$ is well defined since $f_n^- \leq g \in L^1$. Since $g + f_n \geq 0$, we may use Fatou's lemma to write
\begin{align*}
\int g + \int \limfn f_n
&= \int \limfn (g+f_n) \\
& \leq \limfn \int (g + f_n)\\
&= \int g + \limfn \int f_n
\end{align*}

Since $\int g < \infty$, $\int \limfn f_n \leq \limfn \int f_n$ as required. The analogue is as follows: Let $(f_n)_{n \in \N}$ be a sequence of measurable real valued functions. Suppose that there exists $g \in L^1$ such that $g \geq 0$ and for each $n \in \N$, $f_n \leq g$. Then $\limpn \int f_n \leq \int \limpn f_n$. To show this, just use the result from above with the sequence $(g_n)_{n \in \N}$ given by $g_n = -f_n$.

\end{proof}

\begin{ex}
Let $(f_n)_{n \in \N} \subset L^1(X, \MA, \mu)$ and $f:X \rightarrow \C$. Suppose that $f_n \xrightarrow{\text{uni}} f$. Then 
\begin{enumerate}
\item if $\mu(X) < \infty$, then $f \in L^1(X, \MA, \mu)$ and $\limn \int f_n = \int f$
\item if $\mu(X) = \infty$, then the conclusion of $(1)$ may fail (find an example on $\R$ with Lebesgue measure).
\end{enumerate}
\end{ex}

\begin{proof}
Choose $N \in \N$ such that for $n \geq N$ and $x \in X$, $|f(x) - f_n(x)| < 1$. Then $||f| - |f_N|| < 1$ and so $|f| < |f_N| +1$. Thus $\int |f| \leq \int |f_N| +\mu(X) < \infty$ and $f \in L^1$. Similarly for $n \geq N$, $|f_n| < |f|+ 1$. Dominated convergence then gives us that $\limn \int f_n = \int f$ as required. To see the necessity that $\mu(X) < \infty$, consider $f \equiv 0$ and $f_n = (1/n) \chi_{(0,n)}$. Then $f_n \xrightarrow{\text{uni}} f$, but $1 = \limn \int f_n \neq \int f = 0$.  
\end{proof}

\begin{ex}{Generalized Dominated Convergence}
Let $f_n,g_n,f,g \in L^1$. Suppose that $f_n \xrightarrow{\text{a.e.}} f$, $g_n \xrightarrow{\text{a.e.}} g$, $|f_n| \leq g_n$ and $\int g_n \rightarrow \int g$. Then $\int f_n \rightarrow \int f$.
\end{ex}


\begin{proof}
We simply use Fatou's lemma. Put $h_n = (g + g_n) - |f_n - f|$. Since for each $n \in \N$, $|f_n| \leq g_n$, we know that $|f| \leq g$. So $h_n \geq 0$ and $h_n \xrightarrow{\text{p.w.}} 2g$. Thus 
\begin{align*}
2\int g 
&= \int \limfn h_n\\
&\leq \limfn \bigg[ \bigg(\int g +\int g_n\bigg) - \int |f_n -f|\bigg]\\
&= 2\int g + \limfn \bigg( - \int |f_n - f| \bigg)\\
&= 2\int g - \limpn \int |f_n - f| 
\end{align*}

Hence $\limpn \int |f_n - f|  \leq 0$ which implies that $\int |f_n - f| \rightarrow 0$ and $\int f_n \rightarrow \int f$ as required. 
\end{proof}

\begin{ex}
Let $(f_n)_{n \in \N} \subset L^1$ and $f \in L^1$. Suppose that $f_n \xrightarrow{\text{a.e.}} f$. Then $\int |f_n - f| \rightarrow 0$ iff $\int |f_n| \rightarrow \int |f|$.
\end{ex}

\begin{proof}
Suppose that $\int |f_n - f| \rightarrow 0$. Since 
\begin{align*}
\bigg|\int |f_n| - \int |f|\bigg| 
&= \bigg|\int (|f_n| - |f|)\bigg|\\
&\leq \int ||f_n| - |f||\\
&\leq \int |f_n - f|,
\end{align*}
we see that $\int |f_n| \rightarrow \int |f|$. Conversely, suppose that $\int |f_n| \rightarrow \int |f|$. Put $h_n = |f_n-f|$,  $g_n = |f_n| + |f|$, $h \equiv 0$ and $g = 2f$. Then $h_n \xrightarrow{\text{a.e.}} h$, $g_n \xrightarrow{\text{a.e.}} g$ and for each $n \in \N$, $h_n \leq g_n$. Our assumption implies that $\int g_n \rightarrow \int g$. Thus the last exercise tells us that $\int h_n \rightarrow \int h$ as required. 

\end{proof}

\begin{ex}
Let $(r_n)_{n \in \N}$ be an enummeration of the rationals. Define $f: \R \rightarrow \Rg$ by 

\[ f(x) = \begin{cases} 
     x^{-\frac{1}{2}} & x \in (0,1) \\
      0 & x \not\in (0,1)
   \end{cases}
\]

and define $g: X \rightarrow \RG$ by $$g(x) = \sum_{n \in \N}2^{-n}f(x -r_n).$$


Then 
\begin{enumerate}
\item $g \in L^1$ (perhaps after redefinition on a null set) and particularly $g < \infty$ a.e. 
\item $g^2 < \infty$ a.e., but $g^2$ is not integrable on any subinterval of $\R$
\item Taking $g \in L^1$, $g$ is unbounded on each subinterval of $\R$ and discontinuous everywhere and remains so after redefinition on a null set
\end{enumerate}
\end{ex}

\begin{proof} For convenience, define $f_n: \R \rightarrow \Rg$ by $f_n(x) = f(x-r_n)$ for $x \in \R$.
To show $(1)$ we note that for each $n \in \N$, $f_n \in L^1$ and
\begin{align*}
\int |2^{-n} f_n| 
&= 2^{-n}\int_0^1 x^{-1/2}dx\\ 
&= 2^{n-1}
\end{align*}

Hence $$\sum_{n \in \N} \int |2^{-n} f_n| = 2 < \infty.$$

Therefore after redefinition on a null set, $g \in L^1.$ In particular $\int |g| < \infty$ and so $|g|$ (and hence $g$) are finite almost everywhere. For $(2)$, since $g < \infty$ a.e., so too is $g^{2}$. Let $a,b \in \R$ and suppose that $a<b$. Choose $N \in \N$ such that $r_N \in (a,b)$. Since all the terms in the sum are nonnegative, $g^{2} \geq \sum_{n \in \N} 2^{-2n}f_n^2$ and so 

\begin{align*}
\int_{(a,b)} g^2 
&\geq \int_{(a,b)} \sum_{n \in \N} 2^{-2n}f_n^2\\
&= \sum_{n \in \N} 2^{-2n} \int_{(a,b)} f_n^2\\
&\geq 2^{-2N} \int_{(a,b)} f_N^2\\
&\geq 2^{-2N} \int_{r_N}^{b \wedge (r_N+1)} \frac{1}{x-r_N} dx\\
&= \infty
\end{align*}

So $g^2$ is not integrable on any subinterval of $\R$. For $(3)$, note that redefining $g$ on a null set does not change the result of $(2)$. Suppose that there is a finite subinterval $I \subset \R$ such that $g$ is bounded on $I$. Hence there exists $M >0$ such that for each $x \in I$, $g(x)^2 \leq M$. Then 
\begin{align*}
\int_I g^2
&\leq M^2 m(I)\\
&< \infty
\end{align*}

which is a contradiction. So $g$ is not bounded on any subinterval of $\R$. Now, suppose that there exists $x_0 \in \R$ such that $g$ is continuous at $x_0$. Choose $\del > 0$ such that for each $x \in \R$, if $|x-x_0|< \del$, then $|g(x) - g(x_0)| < 1$. The reverse triangle inequality tells us that for each $x \in (x_0-\del, x_0 +\del)$, $|g(x)| < 1 + |g(x_0)|$. Hence $g$ is bounded on $(x_0-\del, x_0 +\del)$ which is a contradiction. So $g$ is discontinuous everywhere.
\end{proof}

\begin{ex}
Let $f \in L^1$. 
\begin{enumerate}
\item If $f$ is bounded, then for each $\ep >0$, there exists $\del >0$ such that for each $E \in \MA$, if $\mu(E) < \del$, then $\int_E |f| < \ep $.
\item The same conclusion holds for $f$ unbounded.
\end{enumerate} 
\end{ex}

\begin{proof}
$(1)$ Since $f$ is bounded, there exists $M >0$ such that $|f| \leq M$. Let $\ep >0$. Choose $\del = \ep/2M$. Let $E \in \MA$. Suppose that $\mu(A) < \del$. Then 
\begin{align*}
\int_E|f| 
& \leq M \mu(E)\\
&= M\frac{\ep}{2M}\\
&= \frac{\ep}{2}\\
&< \ep
\end{align*}

$(2)$ Suppose that $f$ is unbounded. Let $\epsilon >0$. Then there exists $\phi \in L^1$ such that $\phi$ is simple and $\int|f-\phi| < \ep/2$. Since $\phi$ is bounded, there exists $\del >0 $ such that for each $E \in \MA$, if $\mu(E) < \del$, then $\int_E |\phi| < \ep/2$. Let $E \in \MA$. Suppose that $\mu(E) < \del$. Then 
\begin{align*}
\int_E|f|
& \leq \int_E |f-\phi| + \int_E |\phi|\\
& < \ep/2 + \ep/2\\
& = \ep
\end{align*}   
\end{proof}

\begin{ex}
Let $f \in L^1(\R, \ML, m)$. Define $F: \R \rightarrow \R$ by $$F(x) = \int_{(-\infty,x]}fdm.$$

Then $F$ is continuous.
\end{ex}

\begin{proof}
Let $x_0 \in \R$ and $\epsilon >0$. Since $f \in L^1$, there exists $\del >0$ such that for $x \in \R$, if $|x-x_0| < \del$, then $$\int_{(x \wedge x_0,x \vee x_0]}|f|dm < \ep.$$ Let $x \in \R$. Suppose that $|x-x_0|< \del$. Then 
\begin{align*}
|F(x)-F(x_0)|
&= \bigg|\int_{(x \wedge x_0,x \vee x_0]}fdm\bigg|\\
& \leq \int_{(x \wedge x_0,x \vee x_0]}|f|dm\\
& < \ep
\end{align*} 

So $F$ is continuous.

\end{proof}

\begin{ex}
Denote by $\del_x$ the point mass measure at $x \in X$ on  measurable space $(X, \MP(X))$. Let $f:X \rightarrow \C$. Then $$\int f d \del_x = f(x)$$  
\end{ex}

\begin{proof}
First assume that $f$ is simple. Then there exist $a_1, a_2, \cdots, a_n \in \C$ and $E_1, E_2, \cdots , E_n \in \MP (X)$ such that $f = \sum_{i = 1}^n a_i\chi_{E_i}$ Thus $\int f d\del_x = f(x)$. Now assume that $f$, which is measurable by choice of $\sig$-algebra, satisfies $f(X) \subset \Rg$. Choose a sequence $(\phi_n)_{n \in \N} \subset L^+$ such that for each $n \in \N$, $\phi_n$ is simple, $\phi_n \leq \phi_{n+1}$ and $\phi_n \xrightarrow{\text{p.w}} f$. From before, we see that for each $n \in \N$, $\int \phi_n d\del_x = \phi_n(x)$. Monotone convergence tells us that $\int f d\del_x = f(x)$. Now just extend to complex valued functions.

\end{proof}

\begin{ex}
Denote by $\#$ the counting measure on the measurable space $(X, \MP(X))$. Let $f:X \rightarrow \C$ and suppose that $f \in L^1$. Then $$\int f d\# = \sum_{x \in X}f(x).$$ In particular, if $f$ is integrable, then $\{x \in X: f(x) \neq 0\}$ is countable.
\end{ex}

\begin{proof}
Please refer to the definition of the sum in the appendix. First suppose that $f(X) \subset \Rg$. For $n \in \N$, put $X_n = \{x \in X: f(x) > 1/n\}$ and define $X^* = \{x \in X: f(x) > 0\}$, $X_0 = \{x \in X: f(x) = 0\}$ Then $X^* = \bigcup\limits_{n \in \N}X_n$. Since $f \in L^1$, we have that for each $n \in \N$,
\begin{align*}
\infty 
&> \int f d\#\\
&\geq \int_{X_n} f d\# \\
&\geq \frac{1}{n} \#(X_n).
\end{align*}

Thus for each $n \in \N$, $X_n$ is finite and $X^*$ is countable. Thus there exists $\{x_n\}_{n \in \N} \subset X$ such that $X^* = \{x_n\}_{n \in \N}$. For $n \in \N$, define $E_n = \{x_1, x_2, \cdots, x_n\}$ and 
\begin{align*}
f_n 
&= f \chi_{E_n}\\
&= \sum_{i = 1}^n f(x_i)\chi_{\{x_i\}}
\end{align*}

Then $f_n \xrightarrow{\text{p.w.}} f\chi_{X^*} = f$ and for each $n \in \N, f_n \leq f_{n+1}$. So
\begin{align*}
\int f 
&= \sup_{n \in \N} \int f_n\\
&= \sup_{n \in \N} \sum_{i =1}^n f(x_i)\\
&= \sum_{x \in X^*} f(x)\\
&=\sum_{x \in X} f(x).
\end{align*} 

For $f:X \rightarrow \C$, our $L^1$ assumption and the result above tell us that $$\sum_{x \in X}|f(x)| < \infty.$$ Thus writing $f = g+ih$, we see that the same is true for $f^+,f^-,g^+,g^-$. Simply using the definitions of the sum and the integral, as well as the result from above, we have that $$\int fd\# = \sum_{x \in X}f(x).$$

\end{proof}

\begin{ex}
Let $f,g:X \rightarrow \R$. Suppose that $f,g \in L^1$. Then $f \leq g$ a.e. iff for each $E \in \MA$, $\int_E f \leq \int_E g$.  
\end{ex}

\begin{proof}
Suppose $f \leq g$ a.e. Put $N = \{x\in X: f(x) > g(x)\} \subset N$. Then $\mu(N) = 0$ and $g-f \geq 0$ on $N^c$. So for each $E \in \MA$,
\begin{align*}
\int_E g - \int_E f 
&= \int_E (g-f)\\
&= \int_{E \cap N^c} (g-f)\\
& \geq 0
\end{align*} 

Conversely, suppose that for each $E \in \MA$, $\int_E f \leq \int_E g$. Put $N_n = \{x \in X: f(x) - g(x) > 1/n\}$ and $N = \{x \in X: f(x) > g(x)\}$. Then $N = \bigcup\limits_{n \in \N}N_n$. Let $n \in \N$. Then our assumption tells us that 
\begin{align*}
0 
&\geq \int_{N_n} f-g\\
& \geq \frac{1}{n} \mu(N_n)\\
& \geq 0.
\end{align*} 

So that $\mu(N_n) = 0$. Thus for each $n \in \N$, $\mu(N_n) = 0$ which implies $\mu(N) = 0$. Therefore $f \leq g$ a.e. as required. 
\end{proof}

\begin{defn}
Let $\MF \subset L^1$. Then $\MF$ is said to be \textbf{uniformly integrable} if for each $\ep >0$, there exists $K \in \N$ such that for each $k \in \N$, if $k \geq K$, then $\sup\limits_{f \in \MF} \int_{\{|f|>k\}}|f| < \ep$. (i.e. $\lim\limits_{k \rightarrow \infty} \sup\limits_{f \in \MF} \int_{\{|f| > k\}} |f| = 0$).
\end{defn}

\begin{ex}

Suppose that $\mu$ is finite. Let $\MF \subset L^1$. Then $\MF$ is uniformly integrable iff 
\begin{enumerate}
\item there exists $M >0$ such that $\sup\limits_{f \in \MF}\int |f| \leq M$
\item for each $\ep >0$, there exists $\del >0$ such that for each $E \in \MA$, if $\mu(E) < \del$, then $\sup\limits_{f \in \MF} \int_E |f| < \ep$.
\end{enumerate}
\end{ex}

\begin{proof}
($\Rightarrow$): (1) Suppose that $\MF$ is uniformly integrable. Then there exists $K \in \N$ such that for each $k \in \N$, if $k \geq K$, then $\sup\limits_{f \in \MF} \int_{\{|f|>k\}} |f| < 1$. Choose $M = \mu(X)K + 1$. Then for each $f \in \MF$, 
\begin{align*}
\int |f| 
&= \int_{\{|f|>K\}} |f| + \int_{\{|f| \leq K|\}}|f|\\
& \leq 1 + K\mu(X)\\
&=M
\end{align*}

(2) Let $\ep >0$. Then choose $K \in \N$ such that $\sup\limits_{f \in \MF}\int_{\{|f|>K\}} |f| < \ep/2$ and choose $\del = \ep/2K$. Let $E \in \MA$. Suppose that $\mu(E) < \del$. Then for $f \in \MF$, 
\begin{align*}
\int_E |f| 
&= \int_{E \cap \{|f| > K\}} |f| + \int_{E \cap \{|f| \leq K\}} |f|\\
& \leq \ep/2 + K\del \\
&=  \ep
\end{align*}

($\Leftarrow$): Choose $M >0$ as in (1). Suppose that there exists $\ep >0$ such that for each $K \in \N$, there exists $f \in \MF$ such that $\mu(\{|f| > K\}) \geq \ep$. Choose $K \in \N$ such that $K > M/\ep$. Then choose $f_K \in \MF$ such that $\mu(\{|f_K| > K\}) \geq \ep$. Then 
\begin{align*}
\int |f_K| 
&\geq \int_{\{|f_K| > K\}} |f|\\
& \geq K\mu(\{|f_K| > K\})\\
& > \frac{M}{\ep} \cdot \ep\\
&= M, \\
\end{align*}  
which is a contradiction. Hence for each $\ep >0$, there exists $K \in \N$ such that for each $f \in \MF$, $\mu(\{|f| > K\}) < \ep$. Since $\mu(\{|f| > k\})$ is a decreasing sequence in $k$, we have that $\lim\limits_{k \rightarrow \infty} \sup\limits_{f \in \MF} \mu(\{|f| > k\}) = 0$. Now, let $\ep > 0$. Choose $\del >0$ as in (2). Choose $K \in \N$ such that for each $k \in \N$, if $k \geq K$, then for each $f \in \MF$, $\mu(\{|f| > k\}) < \del$. Then for each $k \in \N$, if $k \geq K$, then for each $f \in \MF$, 
$$\int_{\{|f| > k\}} |f| < \ep.$$ Thus $\lim\limits_{k \rightarrow \infty} \sup\limits_{f \in \MF} \int_{\{|f|>k\}} |f| = 0$ as required.

\end{proof}

\subsection{Integration on Product Spaces}

\begin{defn}
Let $X$, $Y$, and $Z$ be sets, $E \subset X \times Y$ and $f :X \times Y \rightarrow Z$. For each $x \in X$, define $E_x = \{y \in Y: (x,y) \in E\}$ and $f_x:Y \rightarrow Z$ by $f_x(y) = f(x,y)$. For each $y \in Y$, define $E^y = \{x \in X: (x,y) \in E\}$ and $f^y:X \rightarrow Z$ by $f^y(x) = f(x,y)$. 
\end{defn}

\begin{note}
It is often helpful to observe that $(\chi_E)_x = \chi_{E_x}$ and $(\chi_E)^y = \chi_{E^y}$.
\end{note}

\begin{lem}
Let $(X,\MA), (Y, \MB)$ be measurable spaces, $Z = \RG$ or $\C$ and $f:X\times Y \rightarrow Z$. 
\begin{enumerate}
\item For each $E \in \MA \otimes \MB$, $x \in X$, $y \in Y$, we have that $E_x \in \MB$ and $E^y \in \MA$
\item If $f$ is  $\MA \otimes \MB$-measurable, then for each $x \in X$, $y \in Y$, we have that $f_x$ is $\MB$-measurable and $f^y$ is $\MA$-measurable.   
\end{enumerate}
\end{lem}

\begin{thm}
Let $(X,\MA, \mu), (Y, \MB, \nu)$ be $\sig$-finite measure spaces. Then for each $E \in \MA \otimes \MB$, the maps $\phi:X \rightarrow \RG$ and $\psi: Y \rightarrow \RG$ defined by $\phi(x) = \nu(E_x)$ and $\psi(y) = \mu(E^y)$ are $\MA$-measurable and $\MB$-measurable, respectively and $$\mu \times \nu(E) = \int_X \nu(E_x)d\mu(x) = \int_Y \mu(E^y)d\nu(y)$$ 
\end{thm}

\begin{thm}{Fubini, Tonelli:}
Let $(X,\MA, \mu), (Y, \MB, \nu)$ be $\sig$-finite measure spaces. 

\begin{enumerate}
\item (Tonelli) For each $f \in L^+(X \times Y)$, the functions $g:X \rightarrow \RG$, $h:Y \rightarrow \RG$ defined by $g(x) = \int_Y f_x(y)d\nu(y)$ and $h(y) = \int_X f^y(x) d \mu(x)$ are $\MA$-measurable and $\MB$-measurable respectively and $$\int_{X \times Y}f d \mu \times \nu = \int_X g d\mu = \int_Y h d\nu$$

\item (Fubini) For each $f \in L^1(X \times Y)$, $f_x \in L^1(\nu)$ for $\mu$-a.e. $x \in X$ and $f^y \in L^1(\mu)$ for $\nu$-a.e. $y \in Y$, respectively and  the functions (after redefinition of $f$ on a null set) $g:X \rightarrow \C$, $h:Y \rightarrow \C$ defined by $g(x) = \int_Y f_x(y)d\nu(y)$ and $h(y) = \int_X f^y(x) d \mu(x)$ are in $L^1(\mu)$ and $L^1(\nu)$ respectively. Furthermore 
$$\int_{X \times Y}f d \mu \times \nu = \int_X g d\mu = \int_Y h d\nu$$
\end{enumerate}
\end{thm}

\begin{note}
We usually just write $\int \int f d\mu d\nu$ and $\int \int f d\nu d\mu$ instead of $\int h d\nu$ and $\int g d\mu$ respectively. We have a similar result for complete product measure spaces. See 
\end{note}

\begin{ex}
Take $X=Y= [0,1]$, $\MA = \MB([0,1]), \MB = \MP([0,1])$ and $\mu,\nu$ to be Lebesgue measure and counting measure respectively. Define $D = \{(x,y) \in [0,1]^2: x=y\}$ Show that $$\int \chi_D d\mu \times \nu, \int \int \chi_D d\mu d \nu \text{ and } \int \int \chi_D d\nu d\mu$$ are all different. (Hint: for the first integral, use the definition of $\mu \times \nu$)
\end{ex}

\begin{proof}
Let $x,y \in [0,1]$. Then $(\chi_D)_x = \chi_{D_x} = \chi_{x}$ and $(\chi_D)^y = \chi_{D^y} = \chi_{y}$. Thus

\begin{align*}
\int \int \chi_D d\mu d \nu
&= \int \mu(\{y\}) d\nu\\
&= \int 0 d\nu\\
&= 0
\end{align*}

and
 
\begin{align*}
\int \int \chi_D d\mu d \nu
&= \int \nu(\{x\}) d\mu\\
&= \int 1 d\mu\\
&= 1
\end{align*}
 
Now, Observe that $\int \chi_D d\mu \times \nu = \mu \times \nu(D)$. Recall from the section on product measures that $\mu \times \nu(D) = \inf \{\sum_{n \in \N}\mu(A_n)\nu(B_n): (A_n \times B_n)_{n \in \N} \subset \ME \text{ and } D \subset \bigcup_{n \in \N} A_n \times B_n \}$. Let $(A_n \times B_n)_{n \in \N} \subset \ME$. Suppose that $D \subset \bigcup_{n \in \N}A_n \times B_n$. Then for each $x \in [0,1]$, $(x,x) \in  \bigcup_{n \in \N} A_n \times B_n$. So for each $x \in [0,1]$, there exists $n \in \N$, such that $x \in A_n \cap B_n$. Thus $[0,1] \subset \bigcup_{n \in \N} A_n \cap B_n.$ Since $1  = \mu([0,1]) \leq \sum_{n \in \N}\mu(A_n \cap B_n)$, we know that there exists $n \in \N$ such that $0 < \mu(A_n \cap B_n)$. Thus $\mu(A_n)> 0$ and $\mu(B_n) > 0$. Since $\mu(B_n) > 0$, $B_n$ must be infinite and therefore $\nu(B_n) = \infty$. So $\sum_{n \in \N} \mu(A_n)\nu(B_n) = \infty$.

\end{proof}

\begin{ex}
Let $(X, \MA, \mu)$ be a $\sig$-finite measure space and $f:X \rightarrow \Rg \in L^+$. Show that $G = \{(x,y) \in X \times \Rg: f(x) \geq y\} \in \MA \otimes \MB(\Rg)$ and $\mu \times m (G) = \int_X f d \mu$. The same is true if we replace "$\geq$" with "$>$". (Hint: to show that $G$ is measurable, split up $(x,y) \mapsto f(x) - y$) into the composition of measurable functions. 
\end{ex}

\begin{proof}
Define $\phi: X \times \Rg \rightarrow \Rg^2$ and $\psi: \Rg^2 \rightarrow \Rg$ by $\phi(x,y) = (f(x),y)$ and $\psi(z,y) = z-y$. Then $G = \{(x,y) \in X \times \Rg: \psi \circ \phi(x,y) \geq 0\}$. Let $A, B \in \MB(\Rg)$. Then $\phi^{-1}(A \times B) = f^{-1}(A) \times B \in \MA \times \MB(\Rg)$. Since $\MB(\Rg^2) = \MB(\Rg) \otimes \MB(\Rg) = \sig(\{A \times B: A, B \in \MB(\Rg)\})$, we have that $\phi$ is $\MA \otimes \MB(\Rg)$-$\MB(\Rg^2)$ measurable. Since $\psi$ is continuous, we have that $\psi$ is $\MB(\Rg^2)$-$\MB(\Rg)$ measurable. This implies that $\psi \circ \phi$ is $\MA \otimes \MB(\Rg)$-$\MB(\Rg)$ measurable. Thus $G = \psi \circ \phi^{-1}(\Rg) \in \MA \otimes \MB(\Rg)$. Now for $x \in X$, $G_x = \{y \in \Rg: f(x) \geq y\} = [0, f(x)]$. Thus 

\begin{align*}
\mu \times m(G) 
&= \int \chi_G d\mu \times m\\
&= \int_X \int_{\Rg} \chi_{G_x} dm d\mu(x)\\
&= \int_X f(x) d\mu(x) 
\end{align*}

The same reasoning holds if we replace "$\geq$" with "$>$".
\end{proof}

\begin{ex}
Let $(X, \MA, \mu), (Y, \MB, \nu)$ be $\sig$-finite measure spaces and $f:X \rightarrow \C$, $g:Y \rightarrow \C$. Define $h:X \times Y \rightarrow \C$ by $h(x,y) = f(x)g(y)$.

\begin{enumerate}
\item If $f$ is $\MA$-measurable and $g$ is $\MB$-measurable, then $h$ is $\MA \otimes \MB$-measurable.

\item If $f \in L^1(\mu)$ and $g \in L^1(\nu)$, then $h \in L^1(\mu \times \nu)$ and $$\int_{X \times Y}hd \mu \times \nu = \int_X f d\mu \int_Y g d\nu$$
\end{enumerate}
\end{ex}

\begin{proof}
\begin{enumerate}
\item First suppose that $f$, $g$ are simple. Then there exist $(A_i)_{i=1}^n \subset \MA$, $(B_j)_{j=1}^m \subset \MB$ and $(a_i)_{i=1}^n, (b_i)_{j=1}^m \subset \C$ such that $f = \sum_{i=1}^n a_i \chi_{A_i}$ and $g = \sum_{j=1}^m b_j \chi_{B_j}$. Then $h = \sum_{i=1}^n \sum_{j=1}^m a_i b_j \chi_{A_i \times B_j}$. So $h$ is $\MA \otimes \MB$-measurable. For general $f,g$, there exist $(f_n)_{n \in \N} \subset S(X, \MA)$ and $(g_n)_{n \in \N} \subset S(Y, \MB)$ such that $f_n \rightarrow f$ pointwise, $g_n \rightarrow g$ pointwise and for each $n \in \N$, $|f_n| \leq |f_{n+1}| \leq |f|$ and $|g_n| \leq |g_{n+1}| \leq |g|$. For $n \in \N$, define $h_n \in S(X \times Y, \MA \otimes \MB)$ by $h_n = f_n g_n$. Then $h_n \rightarrow h$ pointwise and for each $n \in \N$, $|h_n| \leq |h_{n+1}| \leq |h|$. Thus $h$ is $\MA \otimes \MB$-measurable.

\item First suppose $f$ and $g$ are simple as before. Then  
\begin{align*}
\int_{X \times Y} |h| d \mu \times \nu 
& \leq \sum_{i=1}^n \sum_{j=1}^m |a_i b_j| \mu(A_i) \nu(B_j)\\ 
&= \big(\sum_{i=1}^n |a_i| \mu(A_i) \big) \big( \sum_{j=1}^m |b_j| \nu(B_j) \big)\\
&= \int_X |f| d\mu \int_Y |g| d \nu\\
&< \infty
\end{align*}

So $h \in L^1(\mu \times \nu)$. Furthermore, 

\begin{align*}
\int_{X \times Y} h d \mu \times \nu 
&= \sum_{i=1}^n \sum_{j=1}^m a_i b_j \mu(A_i) \nu(B_j)\\ 
&= \big(\sum_{i=1}^n a_i \mu(A_i) \big) \big( \sum_{j=1}^m b_j \nu(B_j) \big)\\
&= \int_X f d\mu \int_Y gd \nu
\end{align*}

For general $f \in L^1(\mu), g \in L^1(\nu)$, take $(h_n)_{n \in \N}$ as before. Monotone convergence and the result above say that 

\begin{align*}
\int_{X \times Y} |h| d\mu \times d\nu 
&= \limn \int_{X \times Y} |h_n|d \mu \times \nu\\
&=  \limn \bigg( \int_X |f_n| d\mu \int_Y |g_n| d\nu \bigg) \\
&= \int_X |f| d\mu \int_Y |g| d\nu\\
& < \infty
\end{align*}

So $h \in L^1(\mu \times \nu)$. Dominated convergence and the result above then tell us that 

\begin{align*}
\int_{X \times Y} h d\mu \times d\nu 
&= \limn \int_{X \times Y} h_n d\mu \times d\nu \\
&= \limn \bigg( \int_X f_n d\mu \int_Y g_n d\nu \bigg)\\
&= \int_X f d\mu \int_Y g d\nu
\end{align*}

\end{enumerate}
\end{proof}

\begin{note}
In the above exercise part (2), we can replace $L^1$ with $L^+$ and get the same result by the same method.
\end{note}

\begin{ex}
Let $f:\R \rightarrow \Rg \in L^+$. Show that $$\int_{\R}fdm = \int_{\Rg}m(\{x \in \R: f(x) \geq t\}) dm(t)$$
\end{ex}

\begin{proof}
Note that $$\int_{\Rg}m(\{x \in \R: f(x) \geq t\}) = \int_{\Rg} \bigg[\int_{\R} \chi_{\{x \in \R: f(x) \geq t\}}dm \bigg]dm(t)$$
Comparing this with Tonelli's theorem, we can put $\chi_{\{x \in \R: f(x) \geq t\}} = (\chi_{E})^t = \chi_{E^t}$. Then $E = \{(x,t) \in \R \times \Rg: f(x) \geq t\}$ and $E_x = \{t \in \Rg: f(x) \geq t\} = [0,f(x)]$. Tonelli's theorem tells us that 
\begin{align*}
\int_{\Rg} \bigg[\int_{\R} \chi_{\{x \in \R: f(x) \geq t\}}(x) dm(x) \bigg]dm(t)
&= \int_{\R} \bigg[ \int_{\Rg} \chi_{[0,f(x)]}(t) dm(t) \bigg] dm(x)\\
&= \int_{\R} f(x) dm(x)
\end{align*} 
\end{proof}

\subsection{Convergence}

\begin{defn}
Let $(X, \MA)$ be a measurable space. For convencience we will define $L^0 = \{f:X \rightarrow \C: f \text{ is measurable}\}$.
\end{defn}

\begin{defn}
Let $(f_n)_{n \in \N} \subset L^0$ and $f \in L^0$. Then $f_n$ converges to $f$ \textbf{in measure}, denoted $f_n \xrightarrow{\mu} f$, if for each $\ep > 0$, $\mu(\{x \in X: |f_n(x) - f(x)| \geq \ep \}) \rightarrow 0$.
\end{defn}

\begin{note}
It is useful to observe that $$\bigcup_{\ep >0}\limsup\limits_{n \rightarrow \infty} \{x \in X: |f_n(x) - f(x)| \geq \ep \} = \{x \in X: f_n(x) \not \rightarrow f(x) \}$$ and $$\bigcap_{\ep > 0} \liminf_{n \rightarrow \infty}\{x \in X: |f_n(x) - f(x)| < \ep \} = \{x \in X: f_n(x) \rightarrow f(x) \}$$ 
\end{note}

\begin{defn}
Let $(f_n)_{n \in \N} \subset L^0$ and $f \in L^0$. Then $f_n$ converges to $f$ \textbf{almost uniformly} if for each $\ep >0$, there exists $N \in \MA$ such that $\mu(N) < \ep$ and $f_n \xrightarrow{\text{uni}} f$ on $N^c$. This is written $f_n \xrightarrow{\text{a.u.}} f$.
\end{defn}

\begin{thm}
Let $(f_n)_{n \in \N} \subset L^0$ and $f \in L^0$. If $f_n \xrightarrow{\mu} f$, then there exists a subsequence $(f_{n_k})_{k \in \N}$ of $(f_n)_{n \in \N}$ such that $f_{n_k} \xrightarrow{\text{a.e.}} f$.
\end{thm}

\begin{ex}\textbf{Egoroff's Theorem:}
Suppose that $\mu(X) < \infty$. Let $(f_n)_{n \in \N} \subset L^0$ and $f \in L^0$. Suppose that $f_n \xrightarrow{\text{a.e.}}$. Then $f_n \xrightarrow{\text{a.u.}}f$.
\end{ex}

\begin{proof}
Let $\ep >0$. For each $n, k \in \N$, define $E_{n, k} = \{x \in X: \vert f_n(x) - f(x) \vert \geq \frac{1}{k} \}$ and $F_{n,k} = \bigcup\limits_{m \geq n}E_{m,k}$. Then $F_{n,k}$ is decreasing in $n$ and $\bigcap\limits_{n \in \N}F_{n,k} \subset \{x: f_n(x) \not \rightarrow f(x)\}$. Thus $\mu(\bigcap\limits_{n \in \N}F_{n,k}) = 0$. Since $\mu(X) < \infty$, $\inf\limits_{n \in \N}\mu(F_{n,k}) = 0$. Hence we may choose a strictly increasing sequence $(n_k)_{k \in \N} \subset \N$ such that  $\mu(F_{n_k,k}) \leq \frac{\ep}{2^{k}}$. Put $N = \bigcup\limits_{k \in \N}F_{n_k,k}$. Then 
\begin{align*}
\mu(N) 
&\leq \sum\limits_{k \in \N}\mu(F_{n_k,k}) \\
& \leq \sum\limits_{k \in \N} \frac{\ep}{2^k}\\
& = \ep
\end{align*} 
Let $\del > 0$. Choose $K \in \N$ such that $\frac{1}{K} < \del$. Then for each $m \geq n_K$ and $x \in N^c =\bigcap\limits_{k \in \N}\bigcap\limits_{m \geq n_k}E_{m,k}^c$, $|f_m(x)- f(x)| < \frac{1}{K} < \del$. So $f_n \convt{uni} f$ on $N^c$. 
\end{proof}

\begin{ex}
Let $(f_n)_{n \in \N} \subset L^1$ and $f \in L^1$. If $f_n \xrightarrow{L^1}f$, then $f_n \conv{\mu} f$.
\end{ex}

\begin{proof}
Let $\ep >0$. for $n \in \N$, define $E_{e,n} = \{x \in X: |f(x) - f_n(x)|\geq \ep\}$. Then for $n \in \N$,
\begin{align*}
\int |f - f_n|
& \geq \int_{E_{\ep,n}} |f- f_n|\\
& \geq \ep \mu(E_{\ep,n}).
\end{align*}

So for each $n \in \N$, $\mu(E_{\ep, n}) \leq \ep^{-1}\int |f - f_n|$. Since $\int |f - f_n| \conv{} 0$, we have that $\mu(E_{\ep,n}) \conv{} 0$. Since $\ep >0$ is arbitrary, $f_n \conv{\mu} f$ as required. 
\end{proof}

\begin{ex}
Suppose $\mu(X) < \infty$. Define $d:L^0 \times L^0 \rightarrow \Rg$ by $$d(f,g) = \int \frac{|f-g|}{1+|f-g|} \hspace{5mm} f,g \in L^0$$
Then $d$ is a metric on $L^0$ if we identify functions that are equal a.e. and convergence in this metric is equivalent to convergence in measure. Note that for each $f,g \in L^0$, $d(f,g) \leq \mu(X)$.
\end{ex} 

\begin{proof}
Let $f,g \in L^0$. Clearly $d(f,g) = d(g,f)$. If $f = g$ a.e. then clearly $d(f,g) = 0$. Conversely, if $d(f,g) = 0$, then $\frac{|f-g|}{1 + |f-g|} = 0$ a.e and so $|f-g| = 0$ a.e. which implies $f =g$ a.e. It is not hard to show that $\phi: \Rg \rightarrow \Rg$ given by $\phi(x) = \frac{x}{1+x}$ satisfies $\phi(x+y) \leq \phi(x)+\phi(y)$. Thus satisfies the triangle inequality. Now, let $(f_n)_{n \in \N} \subset L^0$. Suppose that $f_n \not \conv{\mu} f$. Then there exists $\ep>0, \del>0$ and a subsequence $(f_{n_k})_{k \in \N}$ such that for each $k \in \N$, $\mu(E_{\ep,n_k}) = \mu(\{x \in X: |f_{n_k} - f| \geq \ep\}) \geq \del $. It is not hard to show that $\phi$ from earlier is increasing. Thus for each $k \in \N$, 
\begin{align*}
d(f_{n_k},f)
&= \int \frac{|f_{n_k} -f|}{1+|f_{n_k} -f|}\\
& \geq \int_{E_{\ep,n_k}} \frac{|f_{n_k} -f|}{1+|f_{n_k} -f|}\\
& \geq \int_{E_{\ep, n_k}} \frac{\ep}{1+\ep}\\
& \geq \frac{\ep\del}{1+\ep}
\end{align*}

So $f_{n_k} \not \conv{d} f$. Hence $f_{n_k} \conv{d} f$ implies that $f_{n_k} \conv{\mu} f$. Conversely, suppose that $f_{n_k} \conv{\mu} f$. Let $\ep >0.$ Then $\del = \frac{\ep}{1+\mu(X)} > 0$. Choose $N \in \N$ such that for each $n \in \N$, if $n \geq N$, then $\mu(E_{\del, n}) < \frac{\del}{1+\del}$. Let $n \in \N$. Suppose that $n \geq N$. Since $\phi$ is increasing and $\phi \leq 1$, we have that 
\begin{align*}
d(f_n,f)
&= \int \frac{|f_n -f|}{1+|f_n -f|}\\
&= \int_{E_{\del,n}} \frac{|f_n -f|}{1+|f_n -f|} + \int_{E_{\del,n}^c} \frac{|f_n -f|}{1+|f_n -f|}\\
&\leq \mu(E_{\del,n}) + \mu(X)\frac{\del}{1+\del}\\
& < \frac{\del}{1+\del}(1+\mu(X))\\
& \leq \del(1+\mu(X))\\
& = \ep
\end{align*}
\end{proof}

\begin{ex}
Let $(f_n)_{n \in \N} \subset L^0$ and $f \in L^0$. Suppose that for each $n \in \N$, $f_n \geq 0$ and $f_n \conv{\mu} f$. Then $f \geq 0$ a.e. and $\int f \leq \limfn \int f_n$. 
\end{ex}

\begin{proof}
Since $f_n \conv{\mu} f$, there is a subsequence converging to $f$ a.e. So clearly $f \geq 0$ a.e. Now, choose a subsequence $(f_{n_k})_{k \in \N}$ of $(f_n)_{n \in \N}$ such that $\int f_{n_k} \conv{} \limfn \int f_n$. Since $f_n \conv{\mu} f$ so does $(f_{n_k})_{k \in N}$. Therefore there exists a subsequence $(f_{n_{k_j}})_{k \in \N}$ of $(f_{n_k})_{k \in \N}$ such that $f_{n_{k_j}} \convt{a.e.} f$. Thus $f \geq 0 $ a.e. and Fatou's lemma tells us that 
\begin{align*}
\int f 
&\leq \liminf_{j \in \N} \int f_{n_{k_j}}\\
&= \limfn \int f_n.
\end{align*}
\end{proof}

\begin{ex}
Let $(f_n)_{n \in \N} \subset L^0$ and $f \in L^0$. Suppose that there exists $g \in L^1$ such that for each $n \in \N$, $|f_n| \leq g$. Then $f_n \conv{\mu} f$ implies that $f \in L^1$ and $f_n \conv{L^1} f$. 
\end{ex}

\begin{proof}
Clearly $(f_n)_{n \in \N} \subset L^1$. Since $f_n \conv{\mu} f$, there exists a subsequence $(f_{n_k})_{k \in \N} \subset (f_n)_{n \in \N}$ such that $f_{n_k} \convt{a.e.} f$. This implies that $|f| \leq g$ a.e. and so $f \in L^1$. For $n \in \N$, put $h_n = 2g - |f_n-f|$. Then for each $n \in \N$, $h_n \geq 0$ and $h_n \conv{\mu}2g$. By the previous exercise 
\begin{align*}
\int 2g 
&\leq \limfn \int (2g - |f_n -f|)\\
& = \int 2g - \limpn \int|f_n -f|.
\end{align*}

So $\limpn \int|f_n -f| \leq 0$ which implies that $\int|f_n -f| \rightarrow 0$ and $f_n \conv{L^1} f$ as required. 
\end{proof}

\begin{ex}
Let $(f_n)_{n \in \N} \subset L^0$, $f \in L^0$ and $\phi :\C \rightarrow \C$. 
\begin{enumerate}
\item If $\phi$ is continuous, and $f_n \convt{a.e.} f$ then $\phi \circ f_n \convt{a.e.} \phi \circ f$.
\item If $\phi$ is uniformly continuous and $f_n \rightarrow f$ uniformly, almost uniformly or in measure, then $\phi \circ f_n \rightarrow \phi \circ f$ uniformly, almost uniformly or in measure, respectively.
\item Find a counter example to (2) if we drop the word "uniform".
\end{enumerate} 
\end{ex}

\begin{proof}
\begin{enumerate}
\item Clear
\item Suppose that $\phi$ is unifomly continuous. 

(uniform conv.) Suppose that $f_n \convt{uni} f$. Let $\ep > 0$. Choose $\del >0$ such that for each $z,w \in \C$, if $|z-w|<\del$, then $|\phi(z) - \phi(w)| < \ep$. Now choose $N \in \N$ such that for each $n \in \N$ if $n \geq n$ then for each $x \in X$, $|f_n(x)-f(x)| < \del$. Let $n \in \N$, suppose $n \geq N$, Let $x \in X$. Then $|\phi(f_n(x)) - \phi(f(x))| < \ep$. Thus $\phi \circ f_n \convt{uni} \phi \circ f$.

(almost uni.) Suppose that $f_n \convt{a.u.} f$. Let $\ep > 0$. Choose $N \in \MA$ such $\mu(N) < \ep$ and $f_n \convt{uni} f$ on $N^c$. Then from above, we know that $\phi \circ f_n \convt{uni} \phi \circ f$ on $N^c$. Thus $\phi \circ f_n \convt{a.u.} \phi \circ f$.

(measure) Suppose that $f_n \conv{\mu} f$. Let $\ep > 0$. Choose $\del >0$ such that for each $z,w \in \C$, if $|z-w|<\del$, then $|\phi(z) - \phi(w)| < \ep$. Observe that for $x \in X$, if $|f_n(x) - f(x)| < \del$, then $|\phi(f_n(x)) - \phi(f(x))| < \ep$. Hence $E_{n,\ep} = \{x \in X: |\phi(f_n(x)) - \phi(f(x))| \geq \ep\} \subset F_{n,\del} = \{x \in X: |f_n(x) - f(x)| \geq \del\}$. By definition of convergence in measure, $\mu(F_{n,\del}) \rightarrow 0$. Thus $\mu(E_{n,\ep}) \rightarrow 0$. Hence $\phi \circ f_n \conv{\mu} \phi \circ f$.

\item
\end{enumerate}
\end{proof}

\begin{ex}
Let $(f_n)_{n \in \N} \subset L^0$ and $f \in L^0$. Suppose that $f_n \convt{a.u} f$. Then $f_n \conv{\mu}f$ and $f_n \convt{a.e.}f$. 
\end{ex}

\begin{proof}
(measure) Let $\ep>0$, $\del >0$. Choose $M \in \MA$ such that $\mu(M) < \del$ and $f_n \convt{uni} f$ on $M^c$. Choose $N \in \N$ such that for each $n \in \N$, if $n \geq N$, then for each $x \in M^c$, $|f_n(x) - f(x)| < \ep$. Let $n \in \N$. Suppose $n \geq N$. Then $E_{\ep,n} \subset M$ and $\mu(E_{\ep,n}) < \del$. Thus $\mu(E_{\ep,n}) \rightarrow 0$ and $f_n \conv{\mu} f$.

(a.e.) For each $n \in \N$, Choose $N_n \in \MA$ such that $\mu(N_n) < 1/n$ and $f_n \convt{uni} f$ on $N_n^c$. Observe that for $x \in X$, if $x \in \bigcup_{n \in \N}N_n^c$, then $f_n(x) \rightarrow f(x)$. Thus $N = \{x \in X: f_n(x) \not \rightarrow f(x)\} \subset \bigcap_{n \in \N} N_n$. Therefor $\mu(N) = 0$ and $f_n \convt{a.e.} f$.
\end{proof}

\begin{ex}
Let $(f_n)_{n \in \N}, (g_n)_{n \in \N} \subset L^0$ and $f,g \in L^0$. Suppose that $f_n \conv{\mu} f$ and $g_n \conv{\mu}g$. Then 
\begin{enumerate}
\item $f_n + g_n \conv{\mu} f+g$
\item if $\mu(X) < \infty$, then $f_n g_n \conv{\mu} fg$
\end{enumerate}
\end{ex}

\begin{proof}

\begin{enumerate}
\item Let $\ep > 0$. For convenience, put $F_{n,\ep/2} = \{x \in X: |f_n(x) - f(x)| \geq \ep/2\}$, $G_{n, \ep/2} = \{x \in X: |g_n(x) - g(x)| \geq \ep/2\}$, and $(F+G)_{n,\ep} = \{x \in X: |f_n(x)+g_n(x) - (f(x) + g_n(x))| \geq \ep\}$ Observe that for $x \in X$, $|f_n(x) + g_n(x) - (f(x) + g(x))| \leq |f_n(x) - f(x)| + |g_n(x) - g(x)|$. Thus $(F+G)_{n,\ep} \subset F_{n,\ep/2} \cup G_{n, \ep/2}$. Since $\mu(F_{n,\ep/2} \cup G_{n, \ep/2}) \leq \mu(F_{n,\ep/2}) + \mu(G_{n, \ep/2}) \rightarrow 0$, we have that $\mu((F+G)_{n,\ep}) \rightarrow 0$. Hence $f_n + g_n \conv{\mu} f+g$.

\item Suppose that $\mu(X) < \infty$. Let $(f_{n_k}g_{n_k})_{k \in \N}$ be a subsequence of $(f_ng_n)_{n \in \N}$. Choose a subsequence $(f_{n_{k_j}}g_{n_{k_j}})_{j \in \N}$ such that $f_{n_{k_j}} \convt{a.e} f$ and $g_{n_{k_j}} \convt{a.e} g$. Then $f_{n_{k_j}}g_{n_{k_j}} \convt{a.e.} fg$. Egoroff's theorem tells us that $f_{n_{k_j}}g_{n_{k_j}} \convt{a.u.} fg$, which implies that $f_{n_{k_j}}g_{n_{k_j}} \conv{\mu} fg$. Thus for each subsequence $(f_{n_k}g_{n_k})_{k \in \N}$ of $(f_ng_n)_{n \in \N}$, there exists a subsequence $(f_{n_{k_j}}g_{n_{k_j}})_{j \in \N}$ of $(f_{n_k}g_{n_k})_{k \in \N}$ such that $f_{n_{k_j}}g_{n_{k_j}} \conv{\mu} fg$. Using the fact that this is equivalent to convergence in a metric defined in an earlier exercise,
we have that $f_ng_n \conv{\mu} fg$.
\end{enumerate}

\end{proof}

\begin{ex}
Let $(f_n)_{n \in \N}, \subset L^0$ and $f \in L^0$. Suppose that $\mu(X) < \infty$. Then $f_n \conv{\mu}f_n$ iff for each subsequence $(f_{n_k})_{k \in \N}$, there exists a subsequence $(f_{n_{k_j}})_{j \in \N}$ such that $f_{n_{k_j}} \convt{a.e.} f$.
\end{ex}

\begin{proof}
Suppose that $f_n \conv{\mu} f$. Let $(f_{n_k})_{k \in \N}$ be a subsequence. Then $f_{n_k} \conv{\mu} f$. By a previous theorem, there exists a subsequence $(f_{n_{k_j}})_{j \in \N}$ such that $f_{n_{k_j}} \convt{a.e.} f$. Conversely, suppose that for each subsequence $(f_{n_k})_{k \in \N}$, there exists a subsequence $(f_{n_{k_j}})_{j \in \N}$ such that $f_{n_{k_j}} \convt{a.e.} f$. Let $\ep >0$. For $n \in \N$, define $E_{n} = \{x \in X: |f_n(x) - f(x) | \geq \ep\}$ and define $E = \{x \in X: f_n(x) \not \rightarrow f(x)\}$. Let $(f_{n_k})_{k \in \N}$ be a subsequence. Choose a subsequence $(f_{n_{k_j}})_{j \in \N}$ such that $f_{n_{k_j}} \convt{a.e.} f$. Since $\bigg \{x \in X: \limsup\limits_{j \rightarrow \infty} \chi_{E_{n_{k_j}}}(x) = 1\bigg \} = \limsup\limits_{j \rightarrow \infty} E_{n_{k_j}} \subset E$ and $\mu(E) = 0$, we have that $\limsup\limits_{j \rightarrow \infty} \chi_{E_{n_{k_j}}} = 0$ a.e. and $\chi_{E_{n_{k_j}}} \convt{a.e} 0$. Since $\mu(X) < \infty$, the dominated convergence theorem implies that 
\begin{align*}
\mu(E_{n_{k_j}}) 
&= \int \chi_{E_{n_{k_j}}} d \mu  \rightarrow 0
\end{align*} 
So for each subsequence $(\mu(E_{n_k}))_{k \in \N}$, there exists a subsequence $(\mu(E_{n_{k_j}}))_{j \in \N}$ such that $\mu(E_{n_{k_j}}) \rightarrow 0$. Thus $\mu(E_n) \rightarrow 0$ and $f_n \conv{\mu} f$.
\end{proof}

\begin{ex}
Let $(f_n)_{n \in \N}, \subset L^0$, $f \in L^0$ and $\phi: \C \rightarrow \C$. Suppose that $\mu(X) < \infty$. If $\phi$ is continuous and $f_n \conv{\mu} f$, then $\phi \circ f_n \conv{\mu} \phi \circ f$.
\end{ex}

\begin{proof}
Suppose that $\phi$ is continuous and $f_n \conv{\mu} f$. Let $(\phi \circ f_{n_k})_{k \in \N}$ be a subsequence of $(\phi \circ f_{n})_{n \in \N}$. Then $(f_{n_k})_{k \in \N}$ is a subsequence of $(f_{n})_{n \in \N}$. Since $f_n \conv{\mu} f$, the previous exercise tells us that there exists a subsequence $(f_{n_{k_j}})_{j \in \N}$ such that $f_{n_{k_j}} \convt{a.e.} f$. A previous exercise implies that $\phi \circ f_{n_{k_j}}\convt{a.e.} \phi \circ f$. The previous exercise implies that $\phi \circ f_{n}\conv{\mu} \phi \circ f$.
\end{proof}


\begin{ex}
Let $(f_n)_{n \in \N} L^0$ and $f \in L^0$. Suppose that for each $\ep >0$, $$\sum_{n \in \N}\mu(\{x \in X: |f_n(x)-f(x)| > \ep\}) < \infty$$
Then $f_n \convt{a.e.} f$.
\end{ex}

\begin{proof}
Let $\ep>0$. By assumption we know that
\begin{align*}
\int \bigg[ \sum_{n \in \N}\chi_{\{x \in X: |f_n(x)-f(x)| > \ep\}}\bigg] d \mu 
&= \sum_{n \in \N}\int \chi_{\{x \in X: |f_n(x)-f(x)| > \ep\}}d \mu\\
&=\sum_{n \in \N}\mu(\{x \in X: |f_n(x)-f(x)| > \ep\})\\
& < \infty
\end{align*}
Thus we also know that $\sum_{n \in \N}\chi_{\{x \in X: |f_n(x)-f(x)| > \ep\}} < \infty$ a.e. Equivalently, we could say that for a.e. $x \in X$, $|\{n \in \N: f_n(x) - f(x) > \ep\}| < \infty$. For $k \in \N$, define $N_k = \{x \in X: \sum_{n \in \N}\chi_{\{x \in X: |f_n(x)-f(x)| > 1/k\}} = \infty\}$. Then for each $k \in \N$, $\mu(N_k) = 0$. Define $N = \bigcup_{k \in \N} N_k$. Then $\mu(N) = 0$. Let $x \in N^c$ and $\ep > 0$. Choose $k \in \N$ such that $1/k < \ep$. Then $\{n \in \N: f_n(x) - f(x) > \ep\} \subset \{n \in \N: f_n(x) - f(x) > 1/k\}$ which is finite because $x \in N_k^c$. Put $M = \max\{n \in \N: f_n(x) - f(x) > \ep\}$. Then for $m \geq M$, $|f_m(x) - f(x) \leq \ep|$. Thus $f_n(x) \rightarrow f(x)$. Hence $f_n \convt{a.e.} f$.
\end{proof}

\section{Differentiation}

\subsection{Signed Measures}

\begin{defn}
Let $(X, \MA)$ be a measurable space and $\nu : \MA \rightarrow [-\infty, \infty]$. Then $\nu$ is said to be a \textbf{signed measure} if 
\begin{enumerate}
\item for each $E \in \MA$, $\nu(E) < \infty$ or for each $E \in \MA$, $\nu(E) > -\infty$.
\item $\nu(\varnothing) = 0$
\item for each $(E_n)_{n \in \N} \subset \MA$ if $(E_n)_{n \in \N} \subset \MA$ is disjoint, then $\nu(\bigcup\limits_{n \in \N} E_n) = \sum\limits_{n \in \N} \nu(E_n)$ and if $|\sum\limits_{n \in \N} \nu(E_n)| < \infty$, then $\sum\limits_{n \in \N} \nu(E_n)$ converges absolutely.
\end{enumerate}
\end{defn}

\begin{ex}
Let $\nu: \MA \rightarrow \RG$ be a signed measure and $(E_n)_{n \in \N}$, $(F_n)_{n \in \N} \subset \MA$. If $(E_n)_{n \in \N}$ is increasing, then $\nu(\bigcup\limits_{n \in \N} E_n) = \limn \nu(E_n)$. If $(F_n)_{n \in \N}$ is decreasing and $|\nu(E_1)| < \infty$, then $\nu(\bigcap\limits_{n \in \N} F_n) = \limn \nu(F_n)$. 
\end{ex}

\begin{proof}
Put $E'_1 = E_1$, $F'_1 = F_1$ and for $n \in \N$, $n \geq 2$, put $E'_n = E_n \setminus E_{n-1}$ and $F'_n = F_1 \setminus F_n$. Then $(E'_n)_{n \in \N} \subset \MA$ is disjoint. Thus 
\begin{align*}
\nu(\bigcup\limits_{n \in \N} E_n) 
&= \nu(\bigcup\limits_{n \in \N} E'_n)\\
&= \sum\limits_{n \in \N} \nu(E'_n)\\
&= \limn \sum_{n=1}^n \nu(E'_n)\\
&= \limn \nu(E_n)
\end{align*}

Since $(F'_n)_{n \in \N}$ is increasing, we now know that 
\begin{align*}
\nu(F_1) - \nu(\bigcap\limits_{n \in \N} F_n)
&= \nu(F_1 \setminus \bigcap\limits_{n \in \N} F_n)\\
&= \nu(\bigcup\limits_{n \in \N} F'_n) \\
&= \limn \nu(F'_n) \\
&= \limn \nu(F_1 \setminus F_n) \\ 
&= \nu(F_1) - \limn \nu(F_n)
\end{align*}

Since $|\nu(F_1)| < \infty$, we see that $\nu(\bigcap\limits_{n \in \N} F_n) = \limn \nu(F_n)$.
\end{proof}

\begin{defn}
Let $(X, \MA)$ be a measurable space and $\nu: \MA \rightarrow [-\infty, \infty]$ a signed measure and $E \in \MA$. Then $E$ is said to be $\nu$-\textbf{positive}, $\nu$-\textbf{negative} and $\nu$-\textbf{null} if for each $F \in \MA$, $F \subset E$ implies that $\nu(F) \geq 0$, $\nu(F) \leq 0$, $\nu(F) = 0$ respectively.
\end{defn}

\begin{ex}
Let $E \subset \MA$. If $E$ is positive, negative or null, then for each $F \in \MA$, if $F \subset E$, then $F$ is positive, negative or null respectively.
\end{ex}

\begin{proof}
Clear
\end{proof}

\begin{ex}
Let $(E_n)_{n \in \N} \subset \MA$ be positive, negative or null. Then $\bigcup\limits_{n \in \N} E_n$ is positive, negative or null respectively. 
\end{ex}

\begin{proof}
Suppose that $(E_n)_{n \in \N} \subset \MA$ is positive. Let $F \in \MA$. Suppose that $F \subset \bigcup\limits_{n \in \N} E_n$. Put $P_1 = E_1$ and for $n \in \N$, $n \geq 2$, put $P_n = E_n \setminus (\bigcup\limits_{j=1}^{n-1} E_j)$. So $\bigcup\limits_{n \in \N} P_n = \bigcup\limits_{n \in \N} E_n$ and $(P_n)_{n \in \N}$ is disjoint. Thus 
\begin{align*}
\nu(F) 
&= \nu(F \cap \bigcup_{n \in \N} P_n)\\
&= \nu(\bigcup_{n \in \N} (F \cap P_n))\\
&= \sum_{n \in \N} \nu(F \cap P_n)\\
& \geq 0 
\end{align*}

The process is the same if $(E_n)_{n \in \N}$ is negative and null.
\end{proof}

\begin{thm}{Hahn Decomposition:}
Let $\nu$ be a signed measure on $(X, \MA)$. Then there exist $P,N \in \MA$ such that $P$ is positive, $N$ is negative, $X = N \cup P$ and $N \cap P = \varnothing$. Furthermore, these two sets are unique in the following sense: For any $P',N' \in \MA$, if $N,P$ satisfy the properties above, $P' \Delta P = N' \Delta N$ is null.
\end{thm}

\begin{defn}
Let $\nu$ be a signed measure on $(X, \MA)$ and $P,N \in \MA$. Then $P$ and $N$ are said to form a \textbf{Hahn decomposition} of $X$ with respect to $\nu$ if $P,N$ satisfy the results in the above theorem.
\end{defn}

\begin{defn}
Let $\mu, \nu$ be signed measures on $(X, \MA)$. Then $\mu$ and $\nu$ are said to be \textbf{mutually singular} if there exist $E, F \in \MA$ such that $X = E \cup F$, $E \cap F = \varnothing$ and $E$ is $\mu$-null and $F$ is $\nu$-null. We will denote this by $\mu \perp \nu$.
\end{defn}

\begin{thm}{Jordan Decomposition:}
Let $\nu$ be a signed measure on $(X, \MA)$. Then there exist unique positive measures $\nu^+$ and $\nu^-$ on $(X, \MA)$ such that $\nu = \nu^+ - \nu^-$ and $\nu^+ \perp \nu^-$. 
\end{thm}

\begin{proof}
Choose a Hahn decomposition $P,N$ of $X$ with respect to $\nu$. Define $\nu^+, \nu^-$ by $\nu^+(E)= \nu(E \cap P)$ and $\nu^-(E) = \nu(E \cap N)$.
\end{proof}

\begin{defn}
Let $\nu$ be a signed measure on $(X, \MA)$. Then $\nu^+$ and $\nu^-$ from the last theorem are called the \textbf{positive} and \textbf{negative variations} of $\nu$ respectively. We define the \textbf{total variation} measure $|\nu|$ on $(X, \MA)$ by $|\nu| = \nu^+ + \nu^-$. 
\end{defn}

\begin{defn}
Let $\nu$ be a signed measure on $(X,\MA)$. Then $\nu$ is said to be $\sig$-finite if $|\nu|$ is $\sig$-finite.
\end{defn}

\begin{ex}
Let $\nu$ be a signed measure and $\lam, \mu$ positive measures on $(X,\MA)$. Suppose that $\nu = \lam - \mu$. Then $\lam \geq \nu^+$ and $\mu \geq \nu^-$.
\end{ex}

\begin{proof}
Choose a Hahn decomposition $P,N$ of $X$ with respect to $\nu$. Let $E \in \MA$. Then 
\begin{align*}
\lam(E \cap P) - \mu(E \cap P) 
&= \nu(E \cap P)\\
&= \nu^+(E \cap P)
\end{align*}
So $\lam(E \cap P) \geq  \nu^+(E \cap P) $ and therefore 
\begin{align*}
\lam(E) 
&= \lam(E \cap P) + \lam(E \cap N)\\
& \geq \nu^+(E \cap P) + \lam (E \cap N)\\
& \geq \nu^+(E \cap P)\\
& = \nu^+(E)
\end{align*} 

Similarly $\mu(E \cap N) \geq \nu^-(E \cap N)$ and $\mu(E) \geq \nu^-(E)$.
\end{proof}

\begin{ex}
Let $\nu_1, \nu_2$ be signed measures on $(X, \MA)$. Suppose that $\nu_1 + \nu_2$ is a signed measure. Then $|\nu_1 + \nu_2| \leq |\nu_1|+ |\nu_2|$. (Hint: use the last exercise)
\end{ex}

\begin{proof}
Since 
\begin{align*}
\nu_1 + \nu_2 
&= (\nu_1^+ - \nu_1^-) + (\nu_2^+ - \nu_2^-)\\
&= (\nu_1^+ + \nu_2^+) - (\nu_1^- + \nu_2^-)
\end{align*}
the previous exercise tells us that $\lam = \nu_1^+ + \nu_2^+ \geq (\nu_1 + \nu_2)^+$ and $ \mu = \nu_1^- + \nu_2^- \geq (\nu_1 + \nu_2)^-$. Therefore 
\begin{align*}
|\nu_1 + \nu_2| 
&= (\nu_1 + \nu_2)^+  + (\nu_1 + \nu_2)^-\\
& \leq (\nu_1^+ + \nu_2^+) + (\nu_1^- + \nu_2^-)\\
&= (\nu_1^+ + \nu_1^-) + (\nu_2^+ + \nu_2^-)\\
&= |\nu_1| + |\nu_2|
\end{align*}
\end{proof}

\begin{note}
Recall that a previous exercise from the section on complex valued functions tells us that $L^1(|\nu|) = L^1(\nu^+) \cap L^1(\nu^-)$.
\end{note}

\begin{defn}
Let $\nu$ be a signed measure on $(X, \MA)$. Then we define $L^1(\nu) = L^1(|\nu|)$. For $f \in L^1(\nu)$, we define $$\int f d \nu = \int f d \nu^+ - \int f d\nu^-$$
\end{defn}

\begin{ex}
Let $\nu_1, \nu_2$ be signed measures on $(X, \MA)$. Suppose that $\nu_1 + \nu_2$ is a signed measure. Then 
$L^1(\nu_1)\cap L^1(\nu_2) \subset L^1(\nu_1 + \nu_2)$
\end{ex}

\begin{proof}
The previous exercise tells us that $|\nu_1 + \nu_2| \leq |\nu_1| + |\nu_2|$. Two previous exercises from the section on nonnegative functions tells us that 
\begin{align*}
\int |f|d |\nu_1 + \nu_2| 
& \leq \int |f| d(|\nu_1|+|\nu_2|)\\
&= \int |f|d |\nu_1| + \int |f| d|\nu_2|
\end{align*}
\end{proof}

\begin{ex}
Let $\nu, \mu$ be signed measures on $(X,\MA)$ and $E \in \MA$. Then 
\begin{enumerate}
\item $E$ is $\nu$-null iff $|\nu|(E) = 0$
\item $\nu \perp \mu$ iff $|\nu| \perp \mu$ iff $\nu^+ \perp \mu$ and $\nu^- \perp \mu$.
\end{enumerate}
\end{ex}

\begin{proof}
\begin{enumerate}
\item Suppose that $E$ is $\nu$-null. Choose a Hahn decomposition $P,N$ of $X$ with respect to $\nu$. Then $\nu^+(E) = \nu(E \cap P) = 0$ and $\nu^-(E) = \nu(E \cap N) = 0$. Therefore $|\nu|(E) = \nu^+(E) + \nu^-(E) = 0$. Conversely, suppose that $|\nu|(E) = 0$. Then $\nu^+(E) = \nu^-(E) = 0$. Let $F \in \MA$. Suppose that $F \subset E$. Then $\nu^+(F) = 0$ and $\nu^-(F) = 0$. Therefore $\nu(F) = \nu^+(F) - \nu^-(F) = 0$. So $E$ is $\nu$-null.

\item Suppose that $\nu \perp \mu$. Then there exist $E,F \in \MA$ such that $E \cup F = X$, $E \cap F = \varnothing$, $E$ is $\mu$-null and $F$ is $\nu$-null. By (1), $F$ is $|\nu|$-null and thus $|\nu| \perp \mu$. If $|\nu| \perp \mu$, choose $E,F \in \MA$ as before. Since $F$ is $|\nu|$-null, we know that $\nu^+(F) + \nu^-(F) = |\nu|(F) = 0$. This implies that $F$ is $\nu^+$-null and $F$ is $\nu^-$-null. So $\nu^+ \perp \mu$ and $\nu^- \perp \mu$. Finally assume that $\nu^+ \perp \mu$ and $\nu^- \perp \mu$. \textbf{FINISH!!!!}

\end{enumerate}
\end{proof}

\begin{ex}
Let $\nu$ be a signed measure on $(X, \MA)$. Then 
\begin{enumerate}
\item for $f \in L^1(\nu)$, $|\int f d \nu| \leq \int |f| d |\nu|$
\item if $\nu$ is finite, then for each $E \in \MA$, $|\nu|(E) = \sup \{|\int_E f d \nu |: f  \text{ is measurable and } |f| \leq 1 \}$
\end{enumerate}
\end{ex}

\begin{proof}
\begin{enumerate}
\item Let $f \in L^1(\nu)$. Then 
\begin{align*}
\bigg|\int f d \nu \bigg| 
&= \bigg|\int f d \nu^+ - \int f d \nu^-\bigg|\\
& \leq \bigg|\int f d \nu^+\bigg| + \bigg|\int f d \nu^-\bigg|\\
& \leq \int |f| d\nu^+ + \int |f| d\nu^-\\
&= \int |f| d (\nu^+ + \nu^-)\\
&= \int |f| d |\nu|
\end{align*}

\item Let $E \in \MA$. Let $f:X \rightarrow \R$ be measurable and suppose that $|f| \leq 1$. Since $\nu$ is finite, so is $|\nu|$ and thus $f \in L^1(\nu)$. Then (1) tells us that 
\begin{align*}
|\int_E f d \nu| 
& \leq \int_E |f| d |\nu|\\
& \leq |\nu|(E) 
\end{align*}

Now, choose a Hahn decomposition $P,N$ of $X$ with respect to $\nu$. Define $f = \chi_{P} - \chi_{N}$. Then $|f| \leq 1$, $f$ is measurable and 
\begin{align*}
\bigg|\int_E f d\nu\bigg|
&= \bigg|\int_E f d \nu^+ - \int_E f d \nu^-\bigg|\\
&= | \nu^+(E \cap P) + \nu^-(E \cap N)|\\
&= \nu^+(E) + \nu^-(E)\\
&= |\nu|(E).
\end{align*}

\end{enumerate}
\end{proof}

\begin{ex}
Let $\mu$ be a positive measure on $(X, \MA)$ and $f \in L^0(X, \MA)$ extended $\mu$-integrable. Define $\nu$ on $(X, \MA)$ by $\nu(E) = \int_E f d \mu$. Then
\begin{enumerate}
\item $\nu$ is a signed measure
\item for each $E\in \MA$, $|\nu|(E) = \int_E|f|d\mu$.
\end{enumerate} 
\end{ex}

\begin{proof}

\begin{enumerate}

\item Clearly $\nu(\varnothing) = 0$ and $\nu$ is finte by assumption. Let $(E_n)_{n \in \N} \subset \MA$. Suppose that $(E_n)_{n \in \N}$ is disjoint. Then 
\begin{align*}
\nu(\bigcup_{n \in \N} E_n)
&= \int_{\bigcup_{n \in \N} E_n} f d \mu \\ 
&= \int_{\bigcup_{n \in \N} E_n} f^+ d \mu - \int_{\bigcup_{n \in \N} E_n} f^- d \mu\\
&= \sum_{n \in \N} \int_{ E_n} f^+ d \mu - \sum_{n \in \N} \int_{E_n} f^- d \mu\\
&= \sum_{n \in \N} \bigg[ \int_{ E_n} f^+ d \mu - \int_{ E_n} f^- d \mu \bigg]\\
&= \sum_{n \in \N} \int_{ E_n} f d \mu\\
&= \sum_{n \in \N} \nu(E_n)
\end{align*}

If $|\nu(\bigcup_{n \in \N}E_n)| < \infty$, then $ \int_{\bigcup_{n \in \N} E_n} f^+ d  \mu < \infty$ and $ \int_{\bigcup_{n \in \N} E_n} f^- d  \mu < \infty$ because
\begin{align*}
|\nu(\bigcup_{n \in \N}E_n)|
&=\bigg |\int_{\bigcup_{n \in \N} E_n} f d \mu\bigg| \\
&= \bigg |\int_{\bigcup_{n \in \N} E_n} f^+ d \mu - \int_{\bigcup_{n \in \N} E_n} f^- d \mu \bigg|
\end{align*} Therefore, we have that

\begin{align*}
\sum_{n \in \N} |\nu(E_n)|
&=  \sum_{n \in \N} \bigg|\int_{E_n} f d \mu \bigg|\\
&= \sum_{n \in \N} \bigg| \int_{E_n} f^+ d\mu -  \int_{E_n} f^- d\mu \bigg|\\
& \leq \sum_{n \in \N}  \int_{E_n} f^+ d\mu  + \sum_{n \in \N}  \int_{E_n} f^- d\mu \\
&= \int_{\bigcup_{n \in \N} E_n} f^+ d \mu + \int_{\bigcup_{n \in \N} E_n} f^- d \mu\\
& < \infty
\end{align*}
So the sum $\sum_{n \in \N} \nu(E_n)$ converges absolutely and $\nu$ is a signed measure. 

\item Put $P = \{x \in X: f(x) \geq 0\}$ and $N = \{x \in X: f(x) < 0\}$. Then $P,N$ form a Hahn decomposition of $X$ with respect to $\nu$. Thus for $E \in \MA$, $$\nu^+(E) = \int_{E \cap P} f d \mu = \int_E f^+ d \mu$$ and $$\nu^-(E) = \int_{E \cap N} f d \mu = \int_E f^- d \mu$$. So for $E \in \MA$, $$|\nu|(E) = \int_E f^+ d\mu + \int_E f^- d\mu = \int_E |f| d\mu$$
\end{enumerate}
\end{proof}

\subsection{The Lebesgue-Radon-Nikodym Theorem}

\begin{defn}
Let $(X, \MA)$ be a measureable space, $\nu$ be a signed measure on $(X, \MA)$ and $\mu$ a measure on $(X,\MA)$. Then $\nu$ is said to be \textbf{absolutely continuous} with respect to $\mu$, denoted $\nu \ll \mu$, if for each $E \in \MA$, $\mu(E) = 0$ implies that $\nu(E) =0$. 
\end{defn}

\begin{note}
If there exists an extended $\mu$-integrable $f \in L^0(X, \MA)$ such that for each $E \in \MA$, $\nu(E) = \int_E f d\mu$, then we write $d\nu = f d\mu$.
\end{note}

\begin{thm}
Let $(X, \MA)$ be a measureable space, $\nu$ be a $\sig$-finite signed measure on $(X, \MA)$ and $\mu$ a $\sig$-finite measure on $(X,\MA)$. Then there exist unique $\sig$-finite signed measures $\lam$, $\rho$ on $(X, \MA)$ such that $\lam \perp \mu$, $\rho \ll \mu$ and $\nu = \lam + \rho$, and there exists an extended $\mu$-integrable $f \in L^0(X, \MA)$ such that $d\rho = f d \mu$ and $f$ is unique $\mu$-a.e.  
\end{thm}

\begin{defn}
The decomposition $\nu = \lam + \rho$ is referred to as the \textbf{Lebesgue decomposition of $\nu$ with respect to $\mu$}. In the case $\nu \ll \mu$, we have $\lam = 0$ and $\rho = \nu$ and we define the \textbf{Radon-Nikodym derivative of $\nu$ with respect to $\mu$}, denoted by $d\nu/d\mu$, to be $d\nu/d\mu = f$ where $d\nu = fd\mu$.   
\end{defn}

\begin{thm}
Let $\nu$ be a $\sig$-finite signed measure on $(X, \MA)$ and $\mu$, $\lam$ $\sig$-finite measures on $(X,\MA)$. Suppose that $\nu \ll \mu$ and $\mu \ll \lam$. Then 
\begin{enumerate}
\item for each $g \in L^1(\nu)$, $g(d\nu/d\mu) \in  L^1(\mu)$ and $$\int g d\nu = \int g \frac{d\nu}{d\mu} d\mu$$
\item $\nu \ll \lam$ and $$\frac{d \nu}{d\lam} = \frac{d \nu}{d\mu} \frac{d\mu}{d\lam} \hspace{4mm} \lam \text{-a.e.}$$
\end{enumerate}
\end{thm}

\begin{ex}
Let $(\nu_n)_{n \in \N}$ be a sequence of measures and $\mu$ a measure. 
\begin{enumerate}
\item If for each $n \in \N$, $\nu_n \ll \mu$, then $\sum_{n \in \N} \nu_n \ll \mu$. 
\item If for each $n \in \N$, $\nu_n \perp \mu$, then $\sum_{n \in \N} \nu_n \perp \mu$.
\end{enumerate} 
\end{ex}

\begin{proof}
\begin{enumerate}
\item Let $E \in \MA$. Suppose that $\mu(E) = 0$. Then for each $n \in \N$, $\nu_i(E) = 0$ and thus $\sum_{n \in \N} \nu_n(E) = 0$. Hence $\sum_{n \in \N} \nu_n \ll \mu$.
\item For each $n \in \N$, there exist $N_i, M_i \in \MA$ such that $N_i \cap M_i = \varnothing$, $N_i \cup M_i = X$ and $\nu_i(M_i) = \mu(N_i) = 0$. Put $N = \bigcup_{n \in \N} N_i$ and $M = N^c$. Note that for each $n \in \N$, $M \subset N_i^c = M_i$. So $\mu(N) \leq \sum_{n \in \N} \mu(N_i) = 0$ and $(\sum_{n \in \N} \nu_i) (M) \leq \sum_{n \in \N} \nu_i(M_i) = 0$. Thus $\sum_{n \in \N} \nu_i \perp \mu$.
\end{enumerate}
\end{proof}


\begin{ex}
Choose $X = [0,1]$, $\MA = \MB_{[0,1]}$. Let $m$ be Lebesgue measure and $\mu$ the counting measure. 

Then 
\begin{enumerate}
\item $m \ll \mu$ but for each $f \in L^+$, $dm \neq f d\mu$
\item There is no Lebesgue decomposition of $\mu$ with respect to $m$.
\end{enumerate}
\end{ex}

\begin{proof}
\begin{enumerate}
\item Let $E \in \MA$. If $\mu(E) = 0$, then $E = \varnothing$ and $m(E) = 0$. So $m \ll \mu$. Suppose for the sake of contradiction that there exists $f \in L^+$ such that $dm = f d\mu$. Then 
\begin{align*}
1
&= m(X) \\
&= \sum_{x \in X} f(x)
\end{align*}

Put $Z = \{x \in X: f(x) \neq 0 \}$. Then $Z$ is countable. So 
\begin{align*}
1
&= m(X \setminus Z) \\
&= \sum_{x \in X \setminus Z} f(x)\\
&= 0
\end{align*}

This is a contradiction, so no such $f$ exists.

\item Suppose for the sake of contradiction that there is a Lebesgue decomposition for $\mu$ with respect to $m$ given by $\mu = \lam + \rho$ where $\lam \perp m$ and $\rho \ll m$. We may assume $\lam$ and $\rho$ are positive. Then for each $x \in X$, $m(\{x\})=0$ which implies that $\rho(\{x\}) = 0$. Let $E \subset X$, if $E$ is countable, then $\lam(E) = \mu(E)$. If $E$ is uncountable, choose $F \subset E$ such that $F$ is countable. Then 
\begin{align*}
\lam(E) 
& \geq \lam(F) \\
& = \mu(F) \\
&= \infty
\end{align*}

So $\lam = \mu$. This is a contradiction since $\mu \not \perp m$.
\end{enumerate}
\end{proof}

\begin{ex}
Let $(X, \MF, \mu)$ be a measure space and $\ME$ a sub $\sig$-alg of $\MF$ and $f \in L^1(\mu)$. Define $\nu: \ME \rightarrow [0,\infty]$ by $\nu(E) = \int_E f d \mu $. Then $\nu$ is $\sig$-finite. Let $\overline{\mu}$ be the restriction of $\mu$ to $\ME$. So $\nu \ll \overline{\mu}$. Define the \textbf{expectation of} $f$ \textbf{given} $\ME$ to be $E[f|\ME] = d\nu / d \overline{\mu} \in L^1(X, \MF, \overline{\mu})$. Then for each $E \in \ME$, $$\int_E E[f|\ME] d\mu = \int_E f d\mu$$
\end{ex}

\begin{proof}
Let $E \in \ME$. By definition, 
\begin{align*}
\int_E E[f|\ME] d\mu 
&= \int_E d\nu / d \overline{\mu} d\mu \\
&= \int_E d\nu / d \overline{\mu} d \overline{\mu} \hspace{1cm} \text{(since $E \in \ME$)} \\
&= \nu(E)\\
&= \int_E f d \mu
\end{align*}
\end{proof}

\subsection{Complex Measures}

\begin{defn}
Let $(X, \MA)$ be a measurable space and $\nu:\MA \rightarrow \C$. Then $\nu$ is said to be a \textbf{complex measure} if 
\begin{enumerate}
\item $\nu (\varnothing) = 0$
\item for each sequence $(E_n)_{n \in \N} \subset \MA$, if $(E_n)_{n \in \N}$ is disjoint, then $\nu(\bigcup_{n \in \N} E_n) = \sum_{n \in \N} \nu(E_n)$ and $\sum_{n \in \N} \nu(E_n)$ converges absolutely. 
\end{enumerate}
\end{defn}

\begin{note}
We use the same definitions for mutual orthogonality and absolute continuity when discussing complex measures instead of signed measures.
\end{note}

\begin{defn}
Let $(X,\MA)$ be a measurable space and $\nu = \nu_1 + i\nu_2$ a complex measure on $(X,\MA)$. We define $L^1(\nu) = L^1(\nu_1)\cap L^1(\nu_2)$. For $f \in L^1(\nu)$, we define $$\int f d\nu = \int fd\nu_1 + i \int f d \nu_2$$
\end{defn}

\begin{thm}
Let $(X,\MA)$ be a measurable space, $\nu$ a complex measure on $(X, \MA)$ and $\mu$ a $\sig$-finite measure on $(X, \MA)$. Then there exists a  complex measure $\lambda$ on $(X, \MA)$ and $f \in L^1(\mu)$ such that $\lam \perp \mu$ and $d \nu = d \lam + f d\mu$ and such that for each complex measure $\lam '$ on $(X, \MA)$, $f' \in L^1(\mu)$, if $\nu = d \lam '+ f'd \mu$, then $\lam = \lam '$ and $f = f'$  $\mu$-a.e.
\end{thm}

\begin{thm}
Let $\nu$ be a complex measure on $(X, \MA)$ and $\mu$, $\lam$ $\sig$-finite measures on $(X,\MA)$. Suppose that $\nu \ll \mu$ and $\mu \ll \lam$. Then 
\begin{enumerate}
\item for each $g \in L^1(\nu)$, $g(d\nu/d\mu) \in  L^1(\mu)$ and $$\int g d\nu = \int g \frac{d\nu}{d\mu} d\mu$$
\item $\nu \ll \lam$ and $$\frac{d \nu}{d\lam} = \frac{d \nu}{d\mu} \frac{d\mu}{d\lam} \hspace{4mm} \lam \text{-a.e.}$$
\end{enumerate}
\end{thm}

\begin{defn}Let $(X,\MA)$ be a measurable space and $\nu = \nu_1 + i \nu_2$ a complex measure on $(X, \MA)$. Define $\mu = |\nu_1| + |\nu_2|$. Then $\nu \ll \mu$ and thus There exists $f \in L^1(\mu)$ such that $d\nu = f d\mu$. Define $|\nu|: \MA \rightarrow \Rg$ by $|\nu|(E) = \int_E |f|d\mu$ for each $E \in \MA$. We call $|\nu|$ the \textbf{total variation of $\nu$}. 
\end{defn}

\begin{ex}
Let $\nu$ be a complex measure on $(X, \MA)$ and $\mu$ a $\sig$-finite measures on $(X,\MA)$. If $\nu \ll \mu$, then $\{x \in X: d\nu / d \mu(x) = 0 \}$ is $\nu$-null.
\end{ex}

\begin{proof}
Define $f = d\nu / d \mu$ and $E = \{x: f(x) = 0\}$. Let $A \in \MA$ and suppose that $A \subset E$. Then 
\begin{align*}
\nu(A) 
&= \int_A f d\mu\\
&= 0
\end{align*} 
\end{proof}

\begin{ex}
Let $(X, \MA)$ be a measurable space and $\nu = \nu_1 + i\nu_2$ a complex measure on $(X, \MA)$. Then $|\nu_1|, |\nu_2| \leq |\nu| \leq |\nu_1| + |\nu2|$.

\end{ex}

\begin{proof}
Let $\mu$ and $f$ be as in the definition of $|\nu|$. Since for each $E \in \MA$, we have 
\begin{align*}
\nu(E) 
&= \int_E f d\mu\\
&= \int_E f_1 d \mu + i \int_E f_2 d\mu
\end{align*}

and $$\nu(E) = \nu_1(E) + i\nu_2(E)$$

we know that $\nu_1 = f_1 d\mu$ and $\nu_2 = f_2 d \mu$. 

A previous exercise tells us that $d|\nu_1| = |f_1|d\mu$ and $d |\nu_2| = |f_2|d \mu$. Since $|f_1|, |f_2| \leq |f| \leq |f_1|+|f_2|$, we have that 
\begin{align*}
|\nu_1|, |\nu_2| 
&\leq |\nu| \\
&\leq |\nu_1| + |\nu_2|\\
\end{align*}
\end{proof}

\begin{ex}
Let  $(X, \MA)$ be a measurable space, $\nu$ a complex measure on $(X, \MA)$ and $c \in \C$. Then $\vert c \nu \vert = \vert c \vert \vert \nu \vert$.
\end{ex}

\begin{proof}
Define $\mu$ and $f$ as before so that $d \nu = f d \mu$. Then $d (c \nu) = c f d \mu$. Hence 
\begin{align*}
d \vert c \nu \vert 
&= \vert cf \vert d \mu \\
&= \vert c \vert \vert f \vert d \mu\\
&= \vert c \vert d\vert \nu \vert
\end{align*}
So $\vert c \nu \vert = \vert c \vert \vert  \nu \vert$.
\end{proof}

\begin{ex}
Let $(X, \MA)$ be a measurable space and $\nu$ a complex measure on $(X, \MA)$. Then 

\begin{enumerate}
\item for each $E \in \MA$, $|\nu(E)| \leq |\nu|(E)$. 
\item $\nu \ll |\nu|$ and $\big|d \nu /d |\nu|\big| = 1$ $|\nu|$-a.e.
\item $L^1(\nu) = L^1(|\nu|)$ and for each $g \in L^1(\nu)$, $|\int g d\nu| \leq \int |g|d |\nu|$
\end{enumerate}
\end{ex}

\begin{proof}
Let $\mu$, $f \in L^1(\mu)$ be as in the definition of $|\nu|$.
\begin{enumerate}
\item Let $E \in \MA$. Then 
\begin{align*}
|\nu(E)| 
& = \bigg|\int_E f d\mu\bigg|\\
& \leq \int_E |f| d\mu\\
&= |\nu|(E)
\end{align*}

\item Let $E \in \MA$ and suppose that $|\nu|(E)=0$. The previous part implies $|\nu(E)|=0$ and $\nu \ll |\nu|$. Put $g = d \nu / d|\nu|$. Then 
\begin{align*}
f 
&= \frac{d\nu}{d\mu}\\
&= g|f| \hspace{2mm }\mu\text{-a.e.}
\end{align*}

Hence $|f| = |g||f|$ $\mu$-a.e. Since $|\nu| \ll \mu$, $|f| = |g||f|$ $|\nu|$-a.e.

A previous exercise tells us that $|f| \neq 0$ $|\nu|$-a.e. Thus $|g|=1$ $|\nu|$-a.e.\\

\item Write $\nu = \nu_1 + i\nu_2$ and $f = f_1 + if_2$. First we observe that
\begin{align*}
L^1(\nu)
&= L^1(\nu_1) \cap L^1(\nu_2) \\
&= L^1(|\nu_1|) \cap L^1(|\nu_2|)\\
&= L^1(|\nu_1| + |\nu_2|)\\
&= L^1(\mu)
\end{align*}

The previous exercise tells us that 
\begin{align*}
|\nu_1|, |\nu_2| 
&\leq |\nu| \\
&\leq |\nu_1|+ |\nu_2| \\
&= \mu
\end{align*}

Let $g \in L^1(\mu)$. Then 
\begin{align*}
\int |g| d |\nu| 
&\leq \int |g| d \mu \\
&< \infty
\end{align*}
So $g \in L^1(|\nu|)$.

Conversely, let $g \in L^1(|\nu|)$. Then 
\begin{align*}
\int |g| d|\nu_1|, \int |g| d |\nu_2| 
& \leq \int |g|d |\nu|\\
& < \infty
\end{align*}

So 
\begin{align*}
\int |g| d\mu
& =\int |g| d|\nu_1| + \int |g| d |\nu_2| \\
& < \infty
\end{align*}

and $g \in L^1(\mu)$. Hence $L^1(\nu) = L^1(|\nu|)$. 

Now, let $g \in L^1(\nu) = L^1(|\nu|)$, then 
\begin{align*}
\bigg| \int g d\nu \bigg| 
&= \bigg| \int g f d\mu \bigg| \\
& \leq \int |g||f|d\mu\\
& = \int |g| d |\nu|
\end{align*}

\end{enumerate}
\end{proof}

\subsection{Differentiation}

\begin{defn}
Let $f: \R^n \rightarrow \C$. Then $f$ is said to be \textbf{locally integrable} (with respect to Lebesgue measure) if $f$ is measurable and for each $K \subset \R$, $K$ is compact implies $\int_K |f| dm < \infty$. We define $L^1_{\text{loc}}(\R^n) = \{f:\R^n \rightarrow \C: f \text{ is locally integrable}\}$
\end{defn}

\begin{defn}
For $f \in \Ll$, $r>0$, $x \in \R^n$, we define the \textbf{average of $f$ over $B(x,r)$}, denoted by $Af(x,r)$, to be $$Af(x,r) = \frac{1}{m(B(x,r))}\int_{B(x,r)}fdm$$
\end{defn}

\begin{ex}
Let $f \in \Ll$. Define $$H^*f(x) = \sup\{\frac{1}{m(B)}\int_{B}|f|dm: B \text{ is a ball and } x \in B\} \hspace{4mm} (x \in \R^n)$$

Then $Hf \leq H^*f \leq 2^n Hf$. 
\end{ex}

\begin{proof}
Let $x \in \R^n$. Then $$\bigg \{ \frac{1}{m(B(x,r))}\int_{B(x,r)}|f|dm: r >0\bigg \} \subset \bigg\{ \frac{1}{m(B)}\int_{B}|f|dm: B \text{ is a ball and } x \in B \bigg\} $$

So $Hf(x) \leq H^*f(x)$. Let $B$ be a ball. Then there exists $y \in \R^n$, $R>0$ such that $B = B(y,R)$ Suppose that $x \in B$. Then $B \subset B(x,2R)$. Since $m(B(x,2R)) = 2^n m(B(y,R))$, we have that 
\begin{align*}
\frac{1}{m(B)}\int_{B}|f|dm
& \leq \frac{1}{m(B)} \int_{m(B(x,2R))}|f|dm\\
&= \frac{2^n}{m(B(x,2R))} \int_{m(B(x,2R))}|f|dm
\end{align*}

Thus $H^*f(x) \leq 2^n Hf(x)$.
\end{proof}

\begin{lem}
Let $f \in \Ll$, then $Af:\R^n \times (0, \infty)\rightarrow \R$ is continuous.
\end{lem}

\begin{defn}
Let $f \in \Ll$. We define its \textbf{Hardy Littlewood maximal function}, denoted by $Hf$ to be $$Hf(x) = \sup_{r>0} A|f|(x,r) \hspace{4mm} x \in \R^n$$
\end{defn}

\begin{thm}
There exists $C >0$ such that for each $f \in L^1(m)$ and $\al > 0$, $$m(\{x \in \R^n: Hf(x) > \al\}) \leq \frac{C}{a} \int |f|dm$$
\end{thm}

\begin{ex}
Let $f \in \L^1(\R^n)$. Suppose that $||f||_1>0$. Then there exist $C,R>0$ such that for each $x \in \R^n$, if $|x| > R$, then $Hf(x) \geq C|x|^{-n}$. Hence there exists $C' > 0$ such that for each $\al >0$, $m(\{x \in X: Hf(x)>\alpha\}) > C'/\al$ when $\al$ is small. 
\end{ex}

\begin{proof}
Since $||f||_1 >0$, there exists $R>0$ such that $\int_{B(0,R)}|f|dm >0$. Recall that there exists $K>0$ such that for each $x \in R^n$ and $r>0$, $m(B(x,r)) = Kr^n$ Choose $$C = \frac{\int_{B(0,R)}|f|dm}{K2^n}$$. Let $x \in \R^n$. Suppose that $|x|>R$. Then $B(0,R) \subset B(x,2|x|)$. Thus 
\begin{align*}
Hf(x) 
&\geq \frac{1}{m(B(x,2|x|))}\int_{B(x,2|x|)}|f|dm\\
&= \frac{1}{K2^n|x|^n}\int_{B(x,2|x|)}|f|dm \\
&\geq \frac{1}{K2^n|x|^n}\int_{B(0,R)}|f|dm \\
&= \frac{C}{|x^n|}
\end{align*}

Let $a\ < \frac{C}{2R^n}$. Then $R^n < \frac{C}{2 \al}$. Choose $C' =\frac{KC}{2}$. Let $A = \{x \in \R^n: R < |x|< (\frac{C}{\al})^{\frac{1}{n}}\}$. For $x \in A$, 
\begin{align*}
Hf(x) 
&\geq \frac{C}{|x|^n} \\
& > \al
\end{align*}

Thus $A \subset m(\{x \in R^n: Hf(x)> \al\})$ and therefore 
\begin{align*}
m(\{x \in R^n: Hf(x)> \al\}) 
&\geq m(A) \\
&= m(B(0,(C/\al)^{1/n})) - m(B(0,R)) \\
&= K\bigg [\frac{C}{\al} - R^n \bigg] \\
&> K\bigg[\frac{C}{\al} - \frac{C}{2 \al}\bigg] \\
&= \frac{KC}{2 \al}\\
&= \frac{C'}{\al}
\end{align*}
\end{proof}

\begin{thm}
Let $f \in \Ll$, then for a.e. $x \in \R^n$, $$\lim_{r \rightarrow 0} Af(x,r) =f(x)$$. Equivalently, for a.e. $x \in \R^n$, $$ \lim_{r \rightarrow 0} \bigg[ \frac{1}{m(B(x,r))}\int_{B(x,r)}[f(y)-f(x)]dm(y)\bigg] =0$$
\end{thm}

\begin{note}
We can a stronger result of the same flavor.
\end{note}

\begin{defn}
Let $f \in \Ll$. We define the \textbf{Lebesgue set of $f$}, denoted by $L_f$, to be 
\begin{align*}
L_f 
&= \{x \in \R^n: \lim_{r \rightarrow 0} A|f-f(x)|(x,r) =0 \}\\
&= \bigg \{x \in \R^n: \lim_{r \rightarrow 0} \bigg[ \frac{1}{m(B(x,r))}\int_{B(x,r)}|f(y) - f(x)|dm(y)\bigg] =0 \bigg \}
\end{align*}
\end{defn}

\begin{ex}
Let $f \in \Ll$ and $x \in \R^n$. If $f$ is continuous at $x$, then $x \in L_f$.
\end{ex}

\begin{proof}
Suppose that $f$ is continuous at $x$. Let $\ep > 0$. By assumption, there exists $\del >0$ such that for each $y \in \R^n$, if $|x-y|< \del$, then $|f(x)-f(y)| < \ep$. Let $r >0$. Suppose that $r< \del$. Then for each $y \in \R^n$, $y \in B(x,r)$ implies that $|f(x) - f(y)| < \ep$ and thus 
\begin{align*}
\frac{1}{m(B(x,r))}\int_{B(x,r)}|f(y) - f(x)|dm(y)
& \leq \frac{1}{m(B(x,r))} \ep m(B(x,r))\\
&=\ep
\end{align*}
Hence $$\lim_{r \rightarrow 0} \bigg[ \frac{1}{m(B(x,r))}\int_{B(x,r)}|f(y) - f(x)|dm(y)\bigg] =0$$ 
and $x \in L_f$.
\end{proof}

\begin{thm}
Let $f \in \Ll$. Then $m((L_f)^c) = 0$
\end{thm}

\begin{defn}
Let $x \in \R^n$ and $(E_r)_{r>0} \subset \MB(\R^n)$. Then $(E_r)_{r>0}$ is said to \textbf{shrink nicely to $x$} if 

\begin{enumerate}
\item for each $r>0$, $E_r \subset B(x,r)$
\item there exists $\al >0$ such that for each $r>0$, $m(E_r)> \al m(B(x,r))$
\end{enumerate} 
\end{defn}

\begin{thm}
Let $f \in \Ll$ and $(E_r)_{r>0} \subset \MB(\R^n)$. Then for each $x \in L_f$, 

$$\lim_{r \rightarrow 0} \bigg[ \frac{1}{m(E_r)}\int_{E_r}|f(y) - f(x)|dm(y)\bigg] =0$$

and 

$$\lim_{r \rightarrow 0}  \frac{1}{m(E_r)}\int_{E_r}fdm = f(x)$$
\end{thm}

\begin{defn}
Let $\mu:\MB(\R^n) \rightarrow \RG$ be a Borel measure. Then $\mu$ is said to be \textbf{regular} if 
\begin{enumerate}
\item for each $K \subset \R^n$, if $K$ is compact, then $\mu(K)< \infty$
\item for each $E \in \MB(\R^n)$, $\mu(E) = \inf \{\mu(U): U \text{ is open and }E \subset U\}$
\end{enumerate}

Let $\nu$ be a signed or complex Borel measure on $\R^n$. Then $\nu$ is said to be regular if $|\nu|$ is regular.
\end{defn}

\begin{thm}
Let $\nu$ be a regular signed or complex measure on $\R^n$. Let $d\nu = d\lam + f dm$ be the Lebesgue decomposition of $\nu$ with respect to $m$. Then for $m$-a.e. $x \in \R^n$ and $(E_r)_{r >0} \subset \MB(R^n)$, if $(E_r)_{r >0}$ shrinks nicely to $x$, then 

$$\lim_{r \rightarrow 0} \frac{\nu(E_r)}{m(E_r)} = f(x)$$
\end{thm}

\subsection{Functions of Bounded Variation}

\begin{defn}
Let $F:\R \rightarrow \R$ be increasing. Define $F_+:\R \rightarrow \R$ by $$F_+(x) = \lim_{t \rightarrow x^+}F(t) = \inf \{F(t): t>x \}$$
\end{defn}

\begin{note}
Observe that $F \leq F_+$ and $F_+$ is increasing.
\end{note}

\begin{ex}
Let $F:\R \rightarrow \R$ be increasing. Then for each $x \in \R$ and $ \ep>0$, there exists $\del >0$ such that for each $y \in (x,x+\del)$, $0 \leq F_+(y) - F(y) \leq \ep$.
\end{ex}

\begin{proof}
For the sake of contradiction, suppose not. Then there exists $x \in R$ and $\ep >0$ such that for each $\del >0$, there exist $y \in (x,x+\del)$ such that $F_+(y) - F(y) > \ep$. Then there exists a sequence $(y_n)_{n \in \N} \subset \R$ such that for each $n \in \N$, $y_n \in (x, x+\frac{1}{n})$, $y_n > y_{n+1}$ and $F_+(y_n) - F(y_n) > \ep$. Choose $N \in \N$ such that $(N-1)\ep > F(y_1) - F(x)$. Then 
\begin{align*}
F(y_1) - F(x) 
&= \sum_{i=1}^{N-1} \bigg[F(y_i)-F_+(y_{i+1}) + F_+(y_{i+1}) - F(y_{i+1}) \bigg] + F(y_N)- F(x)\\
& = \sum_{i=1}^{N-1} \bigg[F(y_i)-F_+(y_{i+1}) \bigg] + \sum_{i=1}^{N-1} \bigg[ F_+(y_{i+1}) - F(y_{i+1}) \bigg] + F(y_N)- F(x) \\
& \geq (N-1)\ep \\
& > F(y_1) - F(x)
\end{align*}
This is a contradiction, so the claim holds.
\end{proof}

\begin{ex}
Let $F:\R \rightarrow \R$ be increasing. Then $F_+$ is right continuous. 
\end{ex}

\begin{proof}
Let $x \in \R$. Let $\ep >0$. Then there exists $\del_1>0$ such that for each $y \in (x,x+\del_1)$ $0 \leq F(y)-F_+(x) < \ep/2$. There exists $\del_2 >0$ such that for each $y \in (x,x+\del_2)$, \\$0 \leq F_+(y)-F(y) < \ep/2$. Choose $\del = \min\{\del_1, \del_2\}$. Let $y \in (x, x+\del)$.
\begin{align*}
|F_+(x) - F_+(y)|
& \leq |F_+(x) - F(y)| + |F(y)- F_+(y)| \\
& = (F(y) - F_+(x)) + (F_+(y) - F(y)) \\
& \leq \frac{\ep}{2} + \frac{\ep}{2}\\
& = \ep
\end{align*}

So $\lim\limits_{t \rightarrow x^+} F_+(t) = F_+(x)$ and $F_+$ is right continuous.
\end{proof}

\begin{thm}
Let $F:\R \rightarrow \R$ be increasing. Then 
\begin{enumerate}
\item $\{x \in \R: F \text{ is not continuous at }x\}$ is countable
\item $F$ and $F_+$ are differentiable a.e. and $F' = F_+'$ a.e.
\end{enumerate}
\end{thm}

\begin{defn}
Let $F:\R \rightarrow \C$. Define $T_F:\R \rightarrow \R$ by $$T_F(x) = \sup\bigg \{\sum_{i=1}^{n}|F(x_{i}) - F(x_{i-1})|: (x_i)_{i=0}^n \subset \R \text{ is increasing and } x_n=x  \bigg \} \hspace{4mm} (x \in \R)$$

$T_F$ is called the \textbf{total variation function of $F$}.
\end{defn}

\begin{ex}
Let $F:\R \rightarrow \C$. Then $T_F$ is increasing.
\end{ex}

\begin{proof}
Let $x,y \in \R$. Suppose that $x<y_2$. \\Define  $A_x = \big \{\sum_{i=1}^{n}|F(x_{i}) - F(x_{i-1})|: (x_i)_{i=0}^n \subset \R \text{ is increasing and } x_n=x  \big \}$ and \\$A_y = \big \{\sum_{i=1}^{n}|F(x_{i}) - F(x_{i-1})|: (x_i)_{i=0}^n \subset \R \text{ is increasing and } x_n=y  \big \}$. Let $z \in A_x$. Then there exists $(x_i)_{i=0}^n \subset \R$ such that $(x_i)_{i=0}^n$ is increasing,\\ $x_n=x$ and $z = \sum_{i=1}^n |F(x_{i}) - F(x_{i-1})|$. Then

\begin{align*}
z 
& \leq z+|F(y)-F(x)|\\
&= \sum_{i=1}^n |F(x_{i}) - F(x_{i-1})| + |F(y)-F(x)|\\
& \in A_y\\
\end{align*} 
So $z \leq \sup A_y = T_F(y) $ and thus $F_T(x)  = \sup A_x \leq T_F(y)$
\end{proof}

\begin{lem}
Let $F:\R \rightarrow \R$. Then $T_F+F$ and $T_F-F$ are increasing.
\end{lem}

\begin{ex}
For each $F:\R \rightarrow \C$, $T_{|F|} \leq T_F$.
\end{ex}

\begin{proof}
Let $F:\R \rightarrow \C$, $x \in R$ and $(x_i)_{i=0}^n \subset \R$. Suppose that $(x_i)_{i=0}^n$ is increasing and $x_n=x$. Then by the reverse triangle inequality, $$ \sum_{i=1}^n\big||F(x_i)|-|F(x_{i-1})|\big|
\leq \sum_{i=1}^n\big|F(x_i)-|F(x_{i-1})\big|$$
Thus 
\begin{align*}
T_{|F|}(x) 
&= \sup\bigg \{\sum_{i=1}^{n}\big||F(x_{i})| - |F(x_{i-1})|\big|: (x_i)_{i=0}^n \subset \R \text{ is increasing and } x_n=x  \bigg \} \\
& \leq \sup\bigg \{\sum_{i=1}^{n}|F(x_{i}) - F(x_{i-1})|: (x_i)_{i=0}^n \subset \R \text{ is increasing and } x_n=x  \bigg \} \\
&= T_F(x)
\end{align*}
Hence $T_{|F|} \leq T_F$
\end{proof}

\begin{defn}
Let $F:\R \rightarrow \C$. Then $F$ is said to have \textbf{bounded variation} if $\lim \limits_{x \rightarrow \infty}T_F(x)<\infty$. The \textbf{total variation of $F$}, denoted by $TV(F)$, is defined to be $TV(F) = \lim\limits_{x\rightarrow \infty}T_F(x)$.
We define $BV = \{F:\R \rightarrow \C: TV(F)<\infty \}$.
\end{defn}

\begin{defn}
Let $a,b \in \R$ and $F:[a,b] \rightarrow \C$. Define $G_F:\R \rightarrow \C$ by $G_F = F(a)\chi_{(-\infty,a)} + F\chi_{[a,b]}+F(b)\chi_{(b,\infty)}$. Then $F$ is said to have \textbf{bounded variation on $[a,b]$} if $G_F \in BV$. The \textbf{total variation of $F$ on $[a,b]$}, denoted by $TV(F, [a,b])$, is defined to be $TV(F, [a,b]) = TV(G_F)$ We define $BV([a,b]) = \{F:[a,b] \rightarrow \C: TV(F, [a,b]) < \infty\}$.
\end{defn}

\begin{note}
Equivalently, $TV(F, [a,b]) = \sup \big \{\sum_{i=1}^{n}|F(x_{i}) - F(x_{i-1})|: (x_i)_{i=0}^n \subset [a,b] \text{ is increasing, } x_0=a \text{, and } x_n=b\big \}$ and $F \in BV([a,b])$ iff $TV(F, [a,b]) < \infty$. In general, 
\end{note}

\begin{ex}
Let $F \in BV$. Then $F$ is bounded.
\end{ex}

\begin{proof}
If $F$ is unbounded, then the supremum in the previous definition is clearly infinite.
\end{proof}

\begin{ex}
Let $F:\R \rightarrow \R$. If $F$ is bounded and increasing, then $F \in BV$.
\end{ex}

\begin{proof}
Suppose that $F$ is bounded and increasing. Then $-\infty<\inf_{x \in \R}F(x) \leq \sup_{x \in \R}F(x)<\infty$. Let $x \in \R$ and $(x_i)_{i=0}^n \subset \R$. Suppose that $(x_i)_{i=0}^n$ is increasing and $x_n=x$. Then 
\begin{align*}
\sum_{i=1}^n|F(x_i)-F(x_{i-1})| 
&= \sum_{i=1}^n F(x_i)-F(x_{i-1})\\
&= F(x)-F(x_0)
\end{align*}

Thus $$T_F(x) = F(x)-\inf_{x \in \R}F(x)$$. This implies that 
\begin{align*}
TV(F) 
&= \sup_{x \in \R}F(x)-\inf_{x \in \R}F(x)\\
&<\infty
\end{align*}

Hence $F \in BV$.
\end{proof}

\begin{ex}
Let $F:\R \rightarrow \C$. If $F$ is differentiable and $F'$ is bounded on $[a,b]$, then, $F \in BV([a,b])$. 
\end{ex}

\begin{proof}
Suppose that $F$ is differentiable and $F'$ is bounded on $[a,b]$. Then there exists $M>0$ such that for each $x \in [a,b]$, $|F(x)| \leq M$. Let $(x_i)_{i=1}^n \subset [a,b]$. Suppose that $(x_i)_{i=1}^n$ is strictly increasing, $x_0=a$ and $x_n=b$. By the mean value theorem, for each $i =1,2, \cdots, n$, there exists $c_i\in (x_{i-1}, x_i)$ such that $F(x_i)-F(x_{i-1})=F'(c_i)(x_i-x_{i-1})$. Then 
\begin{align*}
\sum_{i=1}^n|F(x_i)-F(x_{i-1})|
&= \sum_{i=1}^n|F'(c_i)(x_i-x_{i-1})|\\
&\leq  \sum_{i=1}^nM(x_i-x_{i-1})\\
&=M(b-a)
\end{align*}

Hence $TV(F, [a,b]) \leq M(b-a)$.
\end{proof}

\begin{ex}
Define $F,G:\R \rightarrow \R$ by 
\[ F(x) = \begin{cases}
x^2sin(x^{-1}) & x \neq 0\\
0 & x=0\\
\end{cases}$$ and $$G(x)=
\begin{cases}
x^2sin(x^{-2}) & x \neq 0\\
0 & x=0
\end{cases}
\]

Then $F$ and $G$ are differentiable, $F \in BV([-1,1])$ and $G \not \in BV([-1,1])$.
\end{ex}

\begin{proof}
On $\R \setminus \{0\}$, 
\begin{align*}
F'(x) 
&= 2xsin(x^{-1})-sin(x^{-1})\\
&= sin(x^{-1})(2x-1)
\end{align*} We see that $F$ is also differentiable at $x=0$ since 
\begin{align*}
F'(0) 
&= \lim_{x \rightarrow 0} \frac{F(x)-F(0)}{x-0}\\
&= \lim_{x \rightarrow 0} \frac{x^2sin(x^{-1})}{x}\\
&= \lim_{x \rightarrow 0} xsin(x^{-1})\\
&=0
\end{align*}

Therefore for each $x \in [-1,1]$, $|F'(x)| \leq 3$. Which by a previous exercise implies that $F \in BV([-1,1])$.

On $\R \setminus \{0\}$, 
\begin{align*}
G'(x)
&= 2xsin(x^{-2})-\frac{2sin(x^{-2})}{x}\\
&= sin(x^{-2})(2x-\frac{2}{x})
\end{align*}
We see that $G$ is also differentiable at $x=0$ since 
\begin{align*}
G'(0) 
&= \lim_{x \rightarrow 0} \frac{G(x)-G(0)}{x-0}\\
&= \lim_{x \rightarrow 0} \frac{x^2sin(x^{-2})}{x}\\
&= \lim_{x \rightarrow 0} xsin(x^{-2})\\
&=0
\end{align*}

For $n \in \N$, define $(x_i)_{i=0}^n \subset [-1,1]$ by $$x_i= \frac{-1}{\sqrt{\frac{\pi}{2}+i\pi}}$$

Then for each $n \in \N$, $(x_i)_{i=1}^n$ is strictly increasing and for each $i=1,2,\cdots, n$ we have that 
\begin{align*}
|G(x_i)-G(x_{i-1})|
&=\frac{1}{\frac{\pi}{2}+i\pi}+ \frac{1}{\frac{\pi}{2}+(i-1)\pi}\\
&=\frac{2}{\pi}\bigg[\frac{(2i-1)+(2i+1)}{(2i+1)(2i-1)}\bigg]\\
&=\frac{2}{\pi}\bigg[\frac{4i}{4i^2-1}\bigg]\\
& > \frac{2}{i\pi}
\end{align*}
\newpage
Hence for each $n \in \N$,
\begin{align*}
TV(G,[-1,1]) 
&\geq \sum_{i=1}^n|G(x_i)-G(x_{i-1})| \\
& > \frac{2}{\pi}\sum_{i=1}^n \frac{1}{i}
\end{align*}
Therefore $G \not \in BV([-1,1])$.
\end{proof}

\begin{ex} The following is stated for $BV$, but is also true for $BV([a,b])$.

\begin{enumerate} 
\item For each $F,G \in BV$, $T_{F+G} \leq T_F + T_G$ and therefore $BV$ is a vector space. 
\item For each $F: \R \rightarrow \C$, $F \in BV$ iff $Re(f) \in BV$ and $Im(F) \in BV$.
\item For each $F:\R \rightarrow \R$, $F \in BV$ iff there exist functions $F_1,F_2:\R \rightarrow \R$ such that $F_1,F_2$ are bounded, increasing and $F=F_1-F_2$
\item For each $F \in BV$ and $x \in \R$, $\lim_{t \rightarrow x^+}F(t)$ and $\lim_{t \rightarrow x^-}F(t)$ exist. 
\item For each $F \in BV$, $\{x \in R: F \text{ is not continuous at }x\}$ is countable.
\item For each $F \in BV$, $F$ and $F_+$ are differentiable a.e. and $F'=(F_+)'$ a.e.
\item For each $F \in BV, c \in \R$, $F-c \in BV$
\end{enumerate}
\end{ex}

\begin{proof}
\begin{enumerate}
\item Let $F, G \in BV$, $x \in \R$ and $\ep>0$. Since $T_{F+G}(x) < \infty$, $T_{F+G}(x)-\ep< T_{F+G}(x)$. Thus there exists $(x_i)_{i=0}^n \subset \R$ such that $(x_i)_{i=0}^n$ is increasing, $x_n=x$ and $T_{F+G}(x) < \sum_{i=1}^n |(F+G)(x_i) - (F+G)(x_{i-1}))|+ \ep$. Thuerefore 

\begin{align*}
T_{F+G}(x)
& < \sum_{i=1}^n |(F+G)(x_i) - (F+G)(x_{i-1}))|+ \ep\\
& \leq \sum_{i=1}^n |F(x_i)-F(x_{i-1})|+ \sum_{i=1}^n|G(x_i)-G(x_{i-1})| + \ep\\
& \leq T_F(x) + T_G(x) + \ep
\end{align*}
Since $\ep >0$ is arbitrary, $T_{F+G}(x) \leq T_F(x)+T_G(x)$. Therefore $TV(F+G) \leq TV(F)+TV(G)<\infty$. Thus $F+G \in BV$. It is straight forward to verify the other requirements needed to show that $BV$ is a vector space.

\item Let $F: \R \rightarrow \C$. Write $F=F_1+iF_2$ with $F_1, F_2:\R \rightarrow \R$. Suppose that $F \in BV$. Note that for each $x_1,x_2 \in \R$ and $j =1,2$, $|F_j(x_1)-F_j(x_2)| \leq |F(x_1)-F(x_2)|$. Let $x\in \R$ and $(x_i)_{i=0}^n \subset \R$. Suppose that $(x_i)_{i=0}^n$ is increasing and $x_n=x$. Then for $j=1,2$ $$\sum_{i=1}^n|F_j(x_i)-F_j(x_{i-1})| \leq \sum_{i=1}^n|F(x_i)-F(x_{i-1})|$$. 
Thus for $j=1,2$ we have that $T_{F_j}(x) \leq T_F(x)$ which implies that $Re(f),Im(F) \in BV$. Conversely, Suppose that $Re(f), Im(F) \in BV$. Then $F=Re(f)+iIm(f) \in BV$ by (1).
\item Suppose that $F \in BV$. Choose $F_1=\frac{1}{2}(T_F-F)$ and $F_2=\frac{1}{2}(T_F+F)$. Then $F_1,F_2$ are bounded, increasing and $F=F_1+F_2$. Conversely, if there exist $F_1,F_2:\R \rightarrow \R$ such that $F_1,F_2$ are bounded, increasing and $F=F_1-F_2$, then $F_1, F_2 \in BV$. By (1) $F \in BV$.
\item This is clear by previous results and (3)
\item This is clear by previous results and (3)
\item This is clear by previous results and (3)
\item Clearly constant functions have zero total variation. The rest is implied by (1).
\end{enumerate}
\end{proof}

\begin{lem}
Let $F \in BV$. Then $\lim_{x \rightarrow -\infty}T_F(x)=0$ and if $F$ is right continuous, then $T_F$ is right continuous.
\end{lem}

\begin{defn}
Define $NBV=\{F \in BV: F \text{ is right continuous and }\lim_{x \rightarrow -\infty}F(x)=0\}$.
\end{defn}

\begin{thm}
Let $M(\R)$ be the set of complex Borel measures on $\R$. For $F \in NBV$, define $\mu_F \in M(\R)$ by $\mu_F((-\infty, x]) = F(x)$. Then $F \mapsto \mu_F$ defines a bijection $NBV \rightarrow M(\R)$. In addition, $|\mu_F| = \mu_{T_F}$
\end{thm}

\begin{thm}
Let $F \in NBV$. Then $F' \in L^1(m)$, $\mu_F \perp m$ iff $F' =0$ a.e. and $\mu_F \ll m$ iff for each $x \in \R$, $\int_{(-\infty, x]}F'dm = F(x)$
\end{thm}

\begin{defn}
Let $F: \R \rightarrow \C$. Then $F$ is said to be \textbf{absolutely continuous} if for each $\ep>0$, there exists $\del>0$ such that for each $((a_i, b_i))_{i=1}^n \subset \MB(\R)$, $\sum_{i=1}^n b_i-a_i < \del$ implies that $\sum_{i=1}^n|F(b_i)-F(a_i)| < \ep$.
\end{defn}

\begin{defn}
Let $F: [a,b] \rightarrow \C$. Then $F$ is said to be \textbf{absolutely continuous on $[a,b]$} if for each $\ep>0$, there exists $\del>0$ such that for each $((a_i, b_i))_{i=1}^n \subset \MB([a,b])$, $\sum_{i=1}^n b_i-a_i < \del$ implies that $\sum_{i=1}^n|F(b_i)-F(a_i)| < \ep$.
\end{defn}

\begin{prop}
Let $F:[a,b] \rightarrow \C$. If $F$ is absolutely continuous on $[a,b]$, then $F \in BV[a,b]$.
\end{prop}

\begin{ex}
Let $F: \R \rightarrow \C$. Suppose that there exists $f \in L^1(m)$ such that $F(x) = \int_{(-\infty, x}fdm$. Then $F \in NBV$.
\end{ex}

\begin{proof}
Let $x \in \R$ and $(x_i)_{i=1}^n \subset \R$. Suppose that $(x_i)_{i=1}^n$ is increasing and $x_n=x$. Then 
\begin{align*}
\sum_{i=1}^n|F(x_i)-F(x_{i-1})| 
&= \sum_{i=1}^n \bigg| \int_{(x_{i-1},x_i]}fdm \bigg|\\
& \leq \sum_{i=1}^n \int_{(x_{i-1},x_i]}|f|dm \\
& = \int_{(x_0,x]}|f|dm\\
& < \int|f|dm
\end{align*}

Hence $T_F(x) \leq \int |f|dm$. Since $x \in \R$ is arbitrary, $TV(F) \leq \int |f|dm$. Therefore $F \in BV$. By the continuity from above and below for measures and the fact that $m({x})=0$ for each $x \in \R$, $F$ is continuous. By continuity from above for measures, $\lim\limits_{x \rightarrow -\infty} F(x) =0$. So $F \in NBV$.
\end{proof}

\begin{lem}
Let $F \in NBV$. Then $F$ is absolutely continuous iff $\mu_F \ll m$.
\end{lem}

\begin{ex}{Fundamental Theorem of Calculus:}
Let $F:[a,b] \rightarrow \C$. The following are equivalent:
\begin{enumerate}
\item $F$ is absolutely continuous on $[a,b]$.
\item there exists $f \in L^1([a,b], m)$ such that for each $x \in [a,b]$, $F(x)-F(a)= \int_{(a,x]}fdm$
\item $F$ is differentiable a.e. on $[a,b]$, $F' \in L^1([a,b], m)$ and for each $x \in [a,b]$, $F(x)-F(a)=\int_{(a,x]}F'dm$
\end{enumerate}
\end{ex}

\begin{proof}
$(1) \implies (3)$ \\
Suppose that $F$ is absolutely continuous on $[a,b]$. Then $F \in BV[a,b]$. Extend $F$ to $\R$ by setting $F(x) = F(a)$ for $x<a$ and $F(x)=F(b)$ for $x>b$. Then $G=F-F(a) \in NBV$ and is absolutely continuus. The previous lemma implies that there exists $f \in L^1(m)$ such that $\mu_G = fdm$. A previous theorem implies that for a.e. $x \in [a,b]$
\begin{align*}
F'(x) 
&= \lim_{r \rightarrow x} \frac{\mu_G((x,x+r])}{m((x,x+r])}\\
&= f(x)
\end{align*}  
So $F$ is differentiable a.e. on $[a,b]$, $F' \in L^1([a,b], m)$ and by construction, for each $x \in [a,b]$, we have that
\begin{align*}
F(x)-F(a)
&= \mu_G((a,x])\\
&= \int_{(a,x]}fdm\\
&= \int_{(a,x]}F'dm
\end{align*}
$(3) \implies (2)$\\
Trivial.\\
$(2) \implies (1)$\\
Suppose that there exists $f \in L^1([a,b], m)$ such that for each $x \in [a,b]$, $F(x)-F(a)=\int_{(a,x]}fdm$. Extend $F$ as before and obtain $G$ as before. Note that a previous exercise implies that $G \in NBV$. Since $\mu_G \ll m$, the previous lemma implies that $G$ is absolutely continuous.
\end{proof}

\begin{ex}
Let $F: \R \rightarrow \C$. If $F$ is absolutely continuous. Then $F$ is differentiable a.e.
\end{ex}

\begin{proof}
Let $n \in \N$. Since $F$ is absolutely continuous on $\R$, $F$ is absolutely continuous on $[-n,n]$. The FTC implies that $F$ is differentiable a.e. on $[-n,n]$. Since $n \in \N$ is arbitrary, $F$ is differentiable a.e on $\R$.
\end{proof}

\begin{ex}
Let $F: \R \rightarrow \C$. Then $F$ is Lipschitz continuous iff $F$ is absolutely continuous and $F'$ is bounded a.e.
\end{ex}

\begin{proof}
Suppose that $F$ is Lipschitz continuous. Then there exists $M>0$ such that for each $x,y \in \R, |F(x)-F(y)| \leq M|x-y|$. Let $\ep >0$. Choose $\del = \frac{\ep}{M}$. Let $((a_i, b_i))_{i=1}^n \subset \MB(\R)$, Suppose that $\sum_{i=1}^n b_i-a_i < \del$. Then 
\begin{align*}
\sum_{i=1}^n|F(b_i)-F(a_i)| 
&\leq \sum_{i=1}^n M(b_i - a_i)\\
&< M\del\\
&= \ep
\end{align*}
Hence $F$ is absolutely continuous. For each $x, y \in \R$, if $x \neq y$, then $\bigg|\frac{F(x)-F(y)}{x-y}\bigg| \leq M$. Hence for a.e. $x \in \R$, $|F'(x)| \leq M$. Conversely, suppose that $F$ is absolutely continuous and $F'$ is bounded a.e. Then there exits $M> 0$ such that for a.e. $x \in \R$, $|F'(x)| \leq M$. Let $x,y \in \R$. Suppose $x<y$. Then the FTC implies that 
\begin{align*}
|F(y)-F(x)|
& = \bigg|\int_{(x,y]}F'dm\bigg|\\
&\leq \int_{(x,y]}|F'|dm\\
&=M|y-x|
\end{align*}
and $F$ is Lipschitz continuous.
\end{proof}

\begin{ex}
Construct an increasing function $F: \R \rightarrow \R$ whose discontinuities is $\Q$.
\end{ex}

\begin{proof}
Let $(q_n)_{n\in \N}$ be an ennumeration of $\Q$. Define $F:\R \rightarrow \R$ by $$F =\sum_{n\in \N} 2^{-n}\chi_{[q_n, \infty)}$$. Equivalently, if we define $S_x=\{n \in \N: q_n \leq x\}$, then we may write $$F(x) = \sum_{n \in S_x}2^{-n}$$\\ Let $x, y \in \R$. Suppose that $x < y$. Then $S_x \subsetneq S_y$. So $F(x)<F(y)$ and therefore $F$ is strictly increasing.\\ 
For each $x,y \in R$ with $x<y$, define $S_{x,y}= \{n \in \N:x< q_n \leq y\}$. Note that $\lim\limits_{y\rightarrow x^+}\min(S_{x,y}) = \infty$ and if $y \in \R \setminus \Q$, then $\lim\limits_{x\rightarrow y^-}\min(S_{x,y}) = \infty$.\\
Now, let $x \in \R$ and $\ep>0$. Choose $N \in \N$ such that $\sum\limits_{n = N}^{\infty}2^{-n}<\ep$. Choose $\del>0$ such that $\min (S_{x,x+\del}) \geq N$.  Let $y \in [x,\infty)$. Suppose that $|x-y| < \del$. Then 
\begin{align*}
|F(x)-F(y)| 
&= \sum_{n \in S_y}2^{-n} - \sum_{n \in S_x}2^{-n}\\
& = \sum_{n \in S_{x,y}}2^{-n}\\
& \leq \sum_{n=N}^{\infty}2^{-n}\\
&<\ep
\end{align*} 

Hence $F$ is right continuous. Now let $x \in \R\setminus \Q$ and $\ep>0$. Choose $N \in \N$ as before and $\del>0$ such that $\min(S_{x-\del,x})\geq N$. Let $y \in (-\infty, x]$. Suppose that $|x-y|<\del$. Then 
\begin{align*}
|F(x)-F(y)| 
&= \sum_{n \in S_x}2^{-n} - \sum_{n \in S_y}2^{-n}\\
& = \sum_{n \in S_{y,x}}2^{-n}\\
& \leq \sum_{n=N}^{\infty}2^{-n}\\
&<\ep
\end{align*}

Hence $F$ is left continuous on $\R\setminus \Q$.\\
Now, let $x \in \Q$. Then there exists $j \in \N$ such that $q_j=x$. Choose $\ep=2^{-j}$. Let $\del>0$. Choose $y=x-\frac{\del}{2}$. Then $|x-y|< \del$ and 
\begin{align*}
|F(x)-F(y)|
&= \sum_{n \in S_{y,x}}2^{-n}\\
&\geq 2^{-j}\\
&= \ep
\end{align*}

Hence $F$ is discontinuous from the left at $x$. Since $x \in \Q$ is arbitrary, $F$ is discontinuous from the left on $\Q$.  
\end{proof}

\begin{ex}
Let $(F_n)_{n\in \N} \in NBV$ be a sequence of nonnegative, increasing functions. If for each $x \in \R$, $F(x)=\sum_{n \in \N}F_n(x)< \infty$, then for a.e. $x \in \R$, $F$ is differentiable at $x$ and $F'(x) = \sum_{n\in \N}F_n'(x)$. 
\end{ex}

\begin{proof}

Define $\mu = \sum_{n \in \N}\mu_{F_n}$. Note that 
\begin{align*}
\mu((-\infty,x]) 
&= \sum_{n \in \N}\mu_{F_n}((-\infty,x]) \\
&= \sum_{n \in \N}F_n(x)\\
&= F(x)
\end{align*}
Hence $F \in NBV$ and $\mu=\mu_F$. For each $n \in \N$, there exist $\lam_n \in M(\R)$ and $f \in L^1(\R)$ such that $d\mu_{F_n} = d\lam_n + f_ndm$ and $\lam \perp m$. Since for each $n \in \N$, $\lam_n, f_n$ are nonnegative, we have that $d\mu_F =  \sum_{n \in \N} d\lam_n + (\sum_{n \in \N}f_n)dm$. By a previous theorem, for a.e. $x \in \R$, 
\begin{align*}
F'(x) 
&= \lim_{r \rightarrow 0}\frac{\mu_F((x,x+r])}{m((x,x+r])} \\
&= \sum_{n \in \N}f_n(x)\\
&= \sum_{n \in \N}\lim_{r \rightarrow 0}\frac{\mu_{F_n}((x,x+r])}{m((x,x+r])} \\
&= \sum_{n \in \N}F_n'(x)
\end{align*}
\end{proof}

\begin{ex}
Let $F:[0,1]\rightarrow [0, 1]$ be the Cantor function. Extend $F$ to $\R$ by setting $F(x) = 0$ for $x<0$ and $F(x)=1$ for $x>1$. Let $([a_n,b_n])_{n \in \N}$ be an ennumeration of the closed subintervals of $[0,1]$ with rational endpoints. For $n \in \N$, define $F_n:\R \rightarrow [0,1]$ by $F_n(x) = F(\frac{x-a_n}{b_n-a_n})$. Define $G:\R \rightarrow \R$ by $G = \sum_{n \in \N}2^{-n}F_n$. Then $G$ is continuous, strictly increasing on $[0,1]$ and $G'=0$ a.e.
\end{ex}

\begin{proof}
Since $F$ is continuous on $\R$, we have that for each $n \in \N$, $F_n$ is continuous on $\R$. We observe that for each $x \in \R$ and $n \in \N$, $|2^{-n}F_n(x)| \leq 2^{-n}$. Thus the Weierstrass M-test implies that $G$ converges uniformly on $\R$ and is therefore continuous. Since $F$ is increasing, for each $n \in \N$, $F_n$ is increasing. Let $x, y \in \R$. Suppose that $x<y$. Choose $j \in \N$ such that $x<a_j<y<b_j$. Then 
\begin{align*}
G(x) 
&= \sum_{n \in \N}2^{-n}F_n(x)\\
&= \sum_{\substack{n \in \N \\ n \neq j}}2^{-n}F_n(x) + 0\\
& < \sum_{\substack{n \in \N \\ n \neq j}}2^{-n}F_n(y) + 2^{-j}F_n(y)\\
&=\sum_{n \in \N}2^{-n}F_n(y)\\
&=G(y)
\end{align*}
So $G$ is strictly increasing.\\
Now we observe that for each $n \in \N$, $F_n \in NBV$. The previous exercise implies that $$G' = \sum 2^{-n}F_n'=0 \text{ a.e.}$$
\end{proof}

\section{Topology}

\begin{defn}
Let $(X,\MA)$ and $(Y,\MB)$ be topological spaces and $f:X \rightarrow Y$. Then 
\begin{enumerate}
\item $f$ is said to be \textbf{continuous} if for each $B \in \MB$, $f^{-1}(B) \in \MA$.
\item $f$ is said to be open if for each $A \in \MA$, $f(A) \in \MB$.
\item $f$ is said to be \textbf{closed} if for each $A \subset X$, if $A^c \in \MA$, then $f(A)^c \in \MB$.
\end{enumerate}
\end{defn}

\section{$L^{p}$ Spaces}

\begin{defn}
Let $(X, \MA, \mu)$ be a measure space and $p \in (0, \infty]$. Define $ \n \cdot \n_p : L^0(X, \MA, \mu) \rightarrow [0, \infty]$ by $$\n f \n_p = \bigg(\int \vert f \vert^p d\mu \bigg)^{\frac{1}{p}} \hspace{1.5cm}( p < \infty)$$ 
and 
$$\n f \n_{\infty} = \inf \bigg \{\lam >0: \mu\big(\{x \in X: \lam < \vert f(x) \vert  \}\big) = 0 \bigg \} $$


We define $$L^p(X, \MA, \mu) =  \{f \in L^0(X, \MA, \mu): \n f \n_p < \infty \}$$
\end{defn}

\begin{thm}{\textbf{Hölder's Inequality:}}
Let $(X, \MA, \mu)$ be a measure space, $p,q \in [1, \infty)$ and $f,g \in L^0$. Suppose that $\frac{1}{p} + \frac{1}{q} = 1$. Then $$\n fg\n_1 \leq \n f \n_p \n g \n_q$$
\end{thm}

\begin{ex}\textbf{Minkowski Inequality:}
Let $(X, \MA, \mu)$ be a measure space, $p \in [1, \infty)$ and $f,g \in L^p$. Then $f+g \in L^p$ and $$\n f+g\n \leq \n f\n_p + \n g\n_p $$
\end{ex}

\begin{proof}
Define $\phi:\R \rightarrow \Rg$ by $\phi(x) = \vert x \vert^p$. Then $\phi$ is convex because it is the composition of an increasing convex function with a convex function. By Jensen's inequality, we have that $$\phi\bigg(\frac{1}{2}[f+g] \bigg) \leq \frac{1}{2}[\phi(f)+\phi(g)]$$ 
This implies that $$\frac{1}{2^p} \vert f+g\vert^p \leq \frac{1}{2}\bigg(\vert f\vert^p +\vert g \vert^p\bigg)$$ 
Hence 
\begin{align*}
\int\vert f + g\vert^p d \mu 
& \leq 2^{p-1}\int \vert f\vert^p +\vert g\vert^p d\mu \\
& = 2^{p-1}\bigg(\int \vert f\vert^p d\mu + \int \vert g\vert^p d\mu \bigg) \\
&= 2^{p-1}\bigg( \n f \n_p^p + \n g \n_p^p\bigg) \\
& < \infty
\end{align*}
So $f+g \in L^p$. Now, it is not hard to see that $|f+g|^p \leq \big( |f| + |g| \big)|f+g|^{p-1}$. Let $q$ be the conjugate of $p$, so that $\frac{1}{p} + \frac{1}{q} = 1$. Then $q(p-1) = p$. We use Hölder's inequality to show that 
\begin{align*}
\n f+g \n_p ^p
&= \int  |f+g|^p d \mu \\
& \leq \int |f| |f+g|^{p-1} d \mu + \int |g| |f+g|^{p-1} d \mu \\
& \leq \n f\n_p \bigg(\int |f+g|^{(p-1)q} d\mu\bigg)^{\frac{1}{q}} + \n g\n_p \bigg(\int |f+g|^{(p-1)q}d\mu\bigg)^{\frac{1}{q}} \\
&= \n f\n_p \bigg(\int |f+g|^{p} d\mu\bigg)^{\frac{1}{q}} + \n g\n_p \bigg(\int |f+g|^{p}d\mu\bigg)^{\frac{1}{q}} \\ 
&= (\n f\n_p + \n g \n_p) \bigg(\int |f+g|^{p} d\mu\bigg)^{\frac{1}{q}}\\
&= (\n f \n_p + \n g \n_p) \n f+g \n_p^{p/q}
\end{align*}
Since $\n f+g \n_p < \infty$, we see that
\begin{align*}
\n f \n_p + \n g \n_p 
& \geq \n f+g \n_p ^{p - p/q} \\
&=  \n f+g \n_p ^{p(1 - 1/q)} \\
&= \n f+g \n_p ^{p/p} \\
&= \n f+g \n_p
\end{align*}
\end{proof}

\begin{ex}
Let $(X, \MA, \mu)$ be a measure space, $p,q \in (0, \infty]$. Suppose that $\mu(X) < \infty$ and $p < q$. Then $L^q \subset L^p$. In particular, if $\mu(X) = 1$, then for each $f \in L^q$, $\n f\n_p \leq \n f\n_q$.
\end{ex}

\begin{proof}
Suppose that $q = \infty$. Let $f \in L^q$. Then
\begin{align*}
\n f \n_p 
&= \bigg(\int \vert f \vert^p d \mu \bigg)^{\frac{1}{p}} \\
& \leq \bigg(\int \n  f \n_{\infty} ^p d \mu \bigg)^{\frac{1}{p}} \\
&= \n f \n_{\infty} \mu(X)^{\frac{1}{p}}
\end{align*} 
If $q < \infty$, then $\frac{q}{p} > 1$ and the conjugate of $\frac{q}{p}$ is $\frac{1}{1- p/q}$. By Hölder's inequality, we have that 
\begin{align*}
\n f \n_p^p 
&= \n f^p \n_1 \\
&\leq \n f^p \n_{\frac{q}{p}} \n 1 \n_{\frac{1}{1-p/q}} \\
&= \bigg(\int |f|^{\frac{pq}{p}} d \mu \bigg)^{\frac{p}{q}} \mu(X)^{1-\frac{p}{q}} \\
&= \bigg(\int |f|^{q} d \mu \bigg)^{\frac{p}{q}}\mu(X)^{1-\frac{p}{q}} \\
&= \n f \n_q^p\mu(X)^{1-\frac{p}{q}}
\end{align*}
Hence 
\begin{align*}
\n f \n_p 
&\leq \n f \n_q\mu(X)^{\frac{1}{p}-\frac{1}{q}} \\
&< \infty
\end{align*}
\end{proof}






























\section{Functional Analysis}

\subsection{Normed Vector Spaces}
\begin{note}
In the following, we will consider vector spaces over $\C$. There are analogous results for real vector spaces as well, just replace every $\C$ with $\R$.
\end{note}

\begin{defn}
Let $X$ be a normed vector space. Then $X$ is said to be a \textbf{Banach space} if $X$ is complete.  
\end{defn}

\begin{defn}
Let $X$ be a normed vector space and $(x_i)_{i=1}^n \subset X$. The series $\sum_{i =1}^{\infty}x_i$ is said to \textbf{converge} if the sequence $s_n := \sum_{i=1}^n x_i$ converges. The series $\sum_{i =1}^{\infty}x_i$ is said to \textbf{converge absolutely} if $\sum_{i\in \N}\n x_i \n < \infty$.
\end{defn}

\begin{thm}
Let $X$ be a normed vector space. Then $X$ is complete iff for each $\seq{i} \subset X$, $\sum_{i =1}^{\infty}x_i$ converges absolutely implies that $\sum_{i=1}^{\infty}x_i$ converges. 
\end{thm}

\begin{proof}
Suppose that $X$ is complete. Let $\seq{i} \subset X$. Suppose that $\sum_{i=1}^{\infty}x_i$ converges absolutely. Let $\ep >0$. Choose $N \in \N$ such that for each $m,n \in \N$, if $m, n \geq N$ and $m< n$, then $\sum_{m+1}^n \n x_i \n < \ep$. Let $m, n \in \N$. Suppose that $m<n$. Then 
\begin{align*}
\n s_n-s_m \n 
&= \bigg \n \sum_{i=1}^n x_i -\sum_{i=1}^m x_i\bigg \n \\
&= \bigg\n \sum_{i=m+1}^{n} x_i \bigg\n\\
& \leq \sum_{i=m+1}^n \n x_i \n \\
& < \ep
\end{align*}

Thus $(s_n)_{n \in N}$ is cauchy. Since $X$ is complete, $\sum_{i=1}^{\infty}x_i$ converges. \\
Conversely, Suppose that for each $\seq{i} \subset X$, $\sum_{i =1}^{\infty}x_i$ converges absolutely implies that $\sum_{i=1}^{\infty}x_i$ converges. Let $\seq{i} \subset X$ be cauchy. Proceed inductively to create a strictly increasing sequence $(n_i)_{i \in \N} \subset \N$ such that for each $m, n \in \N$, if $m,n \geq n_i$, then $ \n x_m-x_n \n < 2^{-i}$. Define $(y_i)_{i \in \N} \subset X$ by 
\[ y_i = \begin{cases}
x_{n_1} & i=1 \\
x_{n_i} - x_{n_{i-1}} & i \geq 2\\
\end{cases}\]

Then $\sum_{i=1}^k y_i = x_{n_k}$ and 
\begin{align*}
\sum_{i \in \N} \n y_i \n 
&= \n x_{n_1} \n+ \sum_{i \in \N} \n x_{n_i}-x_{n_{i-1}} \n \\
& \leq \n x_{n_1} \n + \sum_{i \in \N}2^{-i}\\
& = \n x_{n_1} \n +1
\end{align*}
Hence $(x_{n_k})_{k \in \N} = (\sum_{i=1}^k y_i)_{i\in \N}$ converges. Since $(x_i)_{i \in \N}$ is cauchy and has a convergent subsequence, it converges. So $X$ is complete.
\end{proof}

\begin{defn}
Let $X,Y$ be a normed vector spaces. A linear map $T:X \rightarrow Y$ is said to be \textbf{bounded} if there exists $C \geq 0$ such that for each $x \in X$, $\n Tx \n \leq C \n x \n$.
\end{defn}

\begin{ex}
Let $X,Y$ be a normed vector spaces and $T:X \rightarrow Y$ a linear map. Then $T$ is bounded iff there exists $r,s>0$ such that $T(B(0,r)) \subset B(0,s)$
\end{ex}

\begin{proof}
Suppose that $T$ is bounded. Then there exists $C \geq 0$ such that for each $x \in X$, $\n Tx \n \leq C \n x \n$. Thus $T(B(0,1)) \subset B(0,C+1)$. Conversely. Suppose that there exists $r,s >0$ such that $T(B(0,r)) \subset B(0,s)$. Define $C = \frac{2s}{r}$. Let $x \in X$. Put $\al = \frac{r}{2\n x \n}$ Then $\al x \in B(0,r)$. So $T(\al x ) = \al T(x) \in B(0,s)$. Hence 
\begin{align*}
\n T(\al x) \n 
&= \n \al T(x) \n \\
&= \vert \al \vert \n T(x) \n \\
& = \frac{r}{2 \n x \n}  \n T(x) \n \\
& < s.
\end{align*}
Thus $$\n Tx \n < \frac{2 s}{r} \n x \n = C \n x \n $$ So $T$ is bounded. 
\end{proof}

\begin{thm}
Let $X,Y$ be normed vector spaces and $T:X \rightarrow Y$ a linear map. Then the following are equivalent:
\begin{enumerate}
\item $T$ is continuous
\item $T$ is continuous at $x=0$
\item $T$ is bounded
\end{enumerate}
\end{thm}

\begin{proof}
$(1) \implies (2)$:\\
Trivial\vspace{1cm}\\
$(2) \implies (3)$:\\
Suppose that $T$ is continuous at $x=0$. Then there exists $\del>0$ such that for each $x \in X$, if $\n x \n < \del$, then $\n Tx \n< 1$. Choose $C = \frac{2}{\del}$. If $x=0$, then $\n Tx \n \leq C \n x \n$. Suppose that $\n x \n \neq 0$. Define $y = \frac{\del}{2 \n x \n}x$. Then $\n y \n < \del$. So $$ \n Ty \n
= \frac{\del}{2 \n x \n } \n Tx \n < 1$$
Thus 
\begin{align*}
\n Tx \n 
&< \frac{2}{\del} \n x \n\\
&=C \n x \n
\end{align*}

Hence $T$ is bounded.\vspace{1cm}\\
$(3) \implies (1)$\\
Suppose that $T$ is bounded. Then there exists $C \geq 0$ such that for each $x \in X$, $\n Tx \n \leq C\n x \n$. Let $\ep >0$. Choose $\del = \frac{\ep}{C+1}$. Let $x,y \in X$ Suppose that $\n x-y \n < \del$. Then 
\begin{align*}
\n Tx-Ty \n
& = \n T(x-y) \n\\
& \leq C \n x-y \n \\
&< (C+1) \del\\ 
&= \ep
\end{align*}

So $T$ is continuous.
\end{proof}

\begin{defn}
Let $X,Y$ be normed vector spaces. Define $L(X,Y) = \{T:X \rightarrow Y: T \text{ is bounded}\}$. Define $\n \cdot\n : L(X,Y)\rightarrow \Rg$ by $$\n T\n  = \inf \{C \geq 0: \text{for each }x \in X\text{, } \n Tx \n \leq C\n x\n \}$$ We call $\n \cdot \n $ the \textbf{operator norm on $L(X,Y)$}
\end{defn}

\begin{ex}
Let $X,Y$ be normed vector spaces. If $X\neq \{0\}$, then the operater norm on $L(X,Y)$ is given by: 
\begin{enumerate}
\item $\n T\n  = \sup\limits_{\n x\n =1}\n Tx\n$
\item $\n T\n  = \sup\limits_{x \neq 0}\n x\n^{-1} \n Tx\n $
\item $\n T\n  = \inf \{C \geq 0: \text{for each }x \in X\text{, } \n Tx \n \leq C\n x\n \}$
\end{enumerate}
\end{ex}

\begin{proof} Since $X \neq \{0\}$, the supremums in (1) and (2) are well defined. Let $T \in L(X,Y)$. By linearity of $T$, the sets over which the supremums are taken in (1) and (2) are the same. So (1) and (2) are equal.\vspace{1cm}\\

Now, put $M = \sup\limits_{\n x \n =1} \n Tx \n $, $m = \inf \{C \geq 0: \text{ for each }x \in X\text{, } \n Tx \n \leq C \n x \n \}$ and let $x \in X$. If $\n x \n =0$, then $\n Tx \n \leq M \n x \n$. Suppose that $\n x \n \neq 0$. Then 
\begin{align*}
\n Tx \n
&= \bigg(\big\n T(x/\n x\n )\big\n \bigg)\n x \n \\
& \leq M ||x||
\end{align*}

Hence $M \in \{C \geq 0: \text{ for each }x \in X\text{, } \n Tx \n \leq C \n x \n\}$. Therefore $m \leq M$

Let $C \in \{C \geq 0: \text{ for each }x \in X\text{, } \n Tx \n \leq C \n x\n \}$. Suppose that $\n x \n =1$. Then $\Vert Tx\Vert \leq C \n x \n = C$. So $M \leq C$. Therefore $M \leq m$. So $M=m$ and the supremum in (1) is the same as the infimum in (3). 
\end{proof}

\begin{note}
From here on, unless stated otherwise, we assume $X \neq 0$.
\end{note}

\begin{ex}
Let $X,Y$ be normed vector spaces and $T \in L(X,Y)$. Then for each $x \in X$, $\n Tx \n  \leq \n T\n \n x \n$
\end{ex}

\begin{proof}
This is just part of the previous exercise. Let $x \in X$. If $x = 0$, then $\n Tx \n \leq \n T \n \n x \n $. Suppose that $x \neq 0$. Then $\n Tx \n = T(x/\n x\n )\n x\n \leq \n T \n \n x \n$
\end{proof}

\begin{ex}
Let $X, Y$ be normed vector spaces. Then the operator norm is a norm on $L(X,Y)$.
\end{ex}

\begin{proof}
Let $S,T \in L(X,Y)$ and $\al \in \C$. For each $x \in X$, we have that 
\begin{align*}
\n (S+T)x \n
&= \n Sx+Tx \n \\
& \leq \n Sx \n + \n Tx \n \\
&\leq \n S \n \n x \n + \n T \n \n x \n \\
&= \big(\n S \n + \n T \n \big) \n x \n
\end{align*}

So $\n S+T \n \leq \n S \n + \n T \n$.\vspace{1cm}\\

Using the definition of $\n T \n$, we see that 
\begin{align*}
\n \al T \n 
&= \sup_{\n x \n =1} \n (\al T)x \n \\
&= \sup_{\n x \n =1} \vert \al \vert \n Tx \n\\
&=\vert \al \vert \sup_{\n x \n =1} \n Tx \n \\
&=\vert \al \vert \n T \n
\end{align*} 

So $\n \al S \n = \vert \al \vert \n S \n $. \vspace{1cm}\\ Suppose that $\n T \n = 0$. Let $x \in X$. Then $\n T x\n \leq \n T \n \n x \n = 0$. So $Tx=0$. Since $x \in X$ is arbitrary, we have that $T=0$. 
\end{proof}

\begin{ex}
Let $X$ be a normed vector space. Then addition and scalar multiplication are continuous on $X \times X$ and $\n \cdot \n:X \rightarrow \Rg$ is continuous.
\end{ex}

\begin{proof}
Let $\ep > 0$. Choose $\del = \frac{\ep}{2}$. Let $(x_1,y_1), (x_2,y_2) \in X \times X$. Suppose that $\n (x_1,y_1)-(x_2,y_2) \n  = \max\{\n x_1-x_2 \n , \n y_1 - y_2 \n \} < \del$. Then 
\begin{align*}
\n (x_1 + y_1) - (x_2+y_2) \n
&= \n (x_1-x_2) + (y_1-y_2) \n\\
& \leq \n  x_1-x_2 \n + \n y_1-y_2 \n\\
& < 2\del \\
&= \ep
\end{align*} 

Hence addition is uniformly continuous. \vspace{1cm}\\ Let $(\lam_1,x_1) \in \C \times X$ and $\ep >0$. Choose $\del = \min\{\frac{\ep}{2(\vert \lam_1 \vert + \n x_1 \n + 1)}, \frac{\sqrt{\ep}}{\sqrt{2}}\}$. Let $(\lam_2, x_2) \in \C \times X$. Suppose that $\n (\lam_1, x_1)-(\lam_2,x_2) \n  = \max\{\vert \lam_1-\lam_2 \vert , \n x_1 - x_2 \n \} < \del$. Then 
\begin{align*}
\n \lam_1x_1 - \lam_2x_2 \n
&= \n \lam_1x_1 - \lam_1x_2 + \lam_1x_2- \lam_2x_2 \n\\
&= \n \lam_1(x_1-x_2) + (\lam_1-\lam_2)x_2 \n \\
& \leq \vert \lam_1 \vert \n  x_1-x_2 \n + \vert \lam_1-\lam_2 \vert \n x_2\n\\
& \leq \vert \lam_1 \vert  \n  x_1-x_2 \n + \vert \lam_1-\lam_2 \vert (\n x_1 -x_2\n + \n x_1\n )\\
& < \vert \lam_1 \vert \del  +  \del( \del + \n x_1 \n)\\
&= (\vert \lam_1 \vert + \n x_1 \n) \del + \del^2 \\
&< \frac{\ep}{2}+ \frac{\ep}{2}\\
&= \ep
\end{align*}
Since $(\lam_1, x_1) \in \C \times X$ is arbitrary, scalar multiplication is continuous. \vspace{1cm} \\ Let $\ep > 0$. Choose $\del = \ep$. Let $x,y \in X$. Suppose that $\n x-y \n < \del$. Then 
\begin{align*}
\big \vert \n x \n - \n y \n \big  \vert
& \leq \n x - y \n\\
&< \del\\
&=\ep
\end{align*}  
So $\n \cdot \n : X \rightarrow \Rg$ is uniformly continuous.
\end{proof}

\begin{ex}
Let $X,Y$ be normed vector spaces. If $Y$ is complete, then so is $L(X,Y)$.
\end{ex}

\begin{proof}
Suppose that $Y$ is complete. Let $(T_n)_{n \in \N} \subset L(X,Y)$. Suppose that $(T_n)_{n \in \N}$ is Cauchy. Since for each $m,n \in \N$, $\big\vert \n T_m \n - \n T_n \n \big\vert \leq \n T_m -T_n \n$, we have that $(\n T_n \n )_{n \in \N} \subset \Rg$ is Cauchy. Hence $\lim\limits_{n \rightarrow \infty}\n T_n \n$ exists. \vspace{1cm} \\ Let $x \in X$ and $m,n \in \N$. Then 
\begin{align*}
\n T_m x - T_n x \n 
&= \n (T_m-T_n) x \n \\
&\leq \n T_m-T_n \n \n x \n
\end{align*}
So $(T_nx)_{n \in \N} \subset Y$ is Cauchy and hence converges. Define $T:X \rightarrow Y$ by $Tx = \lim\limits_{n \rightarrow \infty} T_nx$. \vspace{1cm}\\
Since addition and scalar multiplication are continuous, $T$ is linear. Let $x \in X$ and $\ep>0$. Choose $N \in \N$ such that for each $n \in N$, if $n \geq N$, then $\n Tx - T_n x\n < \ep$. Then for each $n \in \N$, if $n \geq N$ we have that 
\begin{align*}
\n Tx\n 
&\leq \n Tx-T_nx \n + \n T_nx \n\\
 &< \ep + \n T_nx \n\\
 &\leq \ep + \n T_n \n \n x \n 
\end{align*}  

Thus $\n Tx \n \leq \ep +(\lim\limits_{n \rightarrow \infty} \n T_n \n) \n x \n$. Since $\ep >0$ is arbitrary, $\n Tx \n \leq (\lim\limits_{n \rightarrow \infty} \n T_n \n) \n x \n$. Thus $T \in L(X,Y)$ and $\n T \n \leq \limn \n T_n \n$. \vspace{1cm} \\
 Note that since addition, scalar multiplication and $\n \cdot \n$ are continuous, we have that for each $n \in \N$ and $x \in X$, $\n (T_n-T_m)x \n $ converges to $\n (T_n-T)x \n$ because 
\begin{align*}
\lim_{m \rightarrow \infty} \n (T_n-T_m)x \n
&= \lim_{m \rightarrow \infty} \n T_nx-T_mx \n\\
&= \n T_nx-\lim_{m \rightarrow \infty}T_mx \n\\
&=\n T_nx-Tx \n\\
&= \n (T_n-T)x \n
\end{align*} 

Let $\ep >0 $. Choose $N \in \N$ such that for each $m, n \in \N$ if $n,m \geq N$, then $\n T_n - T_m \n < \ep$. Then for each $n \in \N$ if $n \geq N$, then for each $x \in X$, $$\n (T_n-T_m)x\n \leq \n (T_n-T_m)\n \n x \n < \ep \n x\n$$ Combining this with the previous fact, we see that for each $n \in N$, if $n \geq N$, then for each $x \in X$, $$\n (T_n -T) x\n \leq \ep \n x \n$$ In particular, for each $n \in \N$, if $n \geq N$, then $$ \n T_n -T \n = \sup\limits_{\n x \n = 1}\n (T_n - T)x \n \leq \ep$$ This implies that $T_n$ converges to $T$ in $L(X,Y)$. 
Since $$\big\vert \n T_n \n - \n T \n\big \vert \leq \n T_n - T \n$$ It is clear that $\limn \n T_n \n = \n T \n$
\end{proof}

\begin{defn}
Let $X$ be a normed vector space and $M \subset X$ a closed subspace. Define $\n \cdot\n :X/M \rightarrow \Rg$ by $$\n x+M\n  := \inf_{y \in M}\n x+y\n $$

We call $\n \cdot\n $ the \textbf{subspace norm on $X/M$}
\end{defn}

\begin{ex}
Let $X$ be a normed vector space and $M \subsetneq X$ a proper, closed subspace of $M$. 
Then 
\begin{enumerate}
\item The previously defined subspace norm on $X/M$ is well defined and is a norm. 
\item For each $\ep > 0$, there exists $x \in X$ such that $\n x\n =1$ and $\n x+M\n  \geq 1-\ep$.
\item The projection map $\pi:X \rightarrow X/M$ defined by $\pi(x) = x+M$ is continuous and $\n \pi\n =1$. 
\item If $X$ is complete, then $X/M$ is complete. 
\end{enumerate} 
\end{ex}

\begin{proof}
\begin{enumerate}
\item  Let $x, y \in X$ and $\al \in \C$. Suppose that $x+M =y+M$. Then there exists $m \in M$ such that $x=y+m$. Since $M$ is a subspace, the map $T:M \rightarrow M$ given by $Tx = x+m$ is a bijection. So $$\inf_{z \in M} \n y+m+z \n = \inf_{z \in M} \n y+z \n$$ which implies that 
\begin{align*}
\n x +M \n 
&= \inf_{z \in M} \n x+z \n \\
&= \inf_{z \in M} \n y+m+z \n \\
&= \inf_{z \in M} \n y+z \n \\
&= \n y+M \n
\end{align*} 
So $\n \cdot \n: X/M \rightarrow \Rg$ is well defined.\vspace{.5cm}\\
We observe that for each $z,w \in M$, $$\n x+y+z \n \leq \n x+w \n + \n y+w+z \n$$
Taking infimums over $M$ with respect to $z$ in this inequality implies that for each $w \in M$,
\begin{align*}
\inf_{z \in M}\n x+y+z \n 
&\leq \inf_{z \in M} \bigg( \n x+w \n + \n y+w+z \n \bigg) \\
&= \n x+w \n +\inf_{z \in M}\n y+w+z \n
\end{align*}
Again we use the fact that for each $w \in M$, $$\inf_{z \in M}\n y+w+z \n = \inf_{z \in M}\n y+z \n$$
This implies that for each $w \in M$, $$\inf_{z \in M}\n x+y+z \n \leq \n x+w \n + \inf_{z \in M}\n y+z \n$$

Therefore, taking infimums over $M$ with respect to $w$ in this inequality yields
\begin{align*}
\n x+y+M \n
&= \inf_{z \in M} \n x+y +z \n\\
& \leq \inf_{w \in M} \bigg(\n x+w \n + \inf_{z \in M}\n y+z \n \bigg)\\
&= \inf_{w \in M} \n x+w \n + \inf_{z \in M}\n y+z \n\\
&= \n x+M \n + \n y+M \n
\end{align*}
\vspace{.5cm}\\
If $\al =0$, then $\al x = 0$. Choosing $z = 0 \in M$ gives $\n \al x+M \n =0 = \vert \al \vert \n x+M \n $. Suppose that $\al \neq 0$. Then the map $T:M \rightarrow M$ given by $Tx = \al ^{-1}x$ is a bijection and thus $\inf\limits_{z \in M} \n x+\al^{-1}z \n = \inf\limits_{z \in M} \n x+z \n$. Hence we have that
\begin{align*}
\n \al x+M \n
&= \inf_{z \in M} \n \al x +z \n \\
&= \inf_{z \in M} \vert \al \vert \n x +\al^{-1}z \n \\
&= \vert \al \vert \inf_{z \in M}\n x +\al^{-1}z \n\\
&= \vert \al \vert \inf_{z \in M}\n x +z \n \\
&= \vert \al \vert \n x+M \n
\end{align*} 

Suppose that $\n x \n =0$. Choose a sequence $(z_n)_{n \in N} \subset M$ such that 
\begin{align*}
\lim\limits_{n \rightarrow \infty} \n x - z_n \n
& = \inf_{z \in M} \n x+ z \n\\
& = 0
\end{align*} 

Then $\limn z_n =x$. Since $M$ is closed, $x \in M$. Hence $x+M=0+M$. \vspace{1cm}\\
\item Since $M$ is a proper subspace, there exists $v \in X$ such that $v \not \in M$. Then $\n v +M \n \neq 0$. Let $\ep >0$. Then $(1-\ep)^{-1}\n v+M \n > \n v+M \n$. So there exists $z \in M$ such that $$0< \n v+M\n \leq \n v+z \n < (1-\ep)^{-1} \n v+M \n$$ Choose $x = \n v+z \n ^{-1}(v+z)$. Then $\n x \n =1$ and 
\begin{align*}
\n x+M \n
&= \n v+z \n ^{-1} \n v+z +M \n\\
&= \n v+z \n ^{-1} \n v +M \n\\
&> 1-\ep
\end{align*}\vspace{.5cm}\\
\item Let $x \in X$. Taking $z=0$, we we see that $\n \pi(x) \n =\n x+M \n \leq \n x+z \n = \n x \n$. So $\pi$ is bounded and in particular, $$\sup_{\n x \n =1} \n \pi(x) \n \leq 1$$ 
From (2) we see that $$\sup_{\n x \n =1} \n \pi(x) \n \geq 1$$
Hence $\n \pi\n = 1$. \vspace{.5cm}\\
\item Suppose that $X$ is complete. Let $(x_i+M)_{i\in \N} \subset X/M$. Suppose that $\sum\limits_{i\in \N} \n x_i+M \n < \infty$. Let $\ep>0$. Then for each $i \in \N$, there exists $z_i \in M$ such that $\n x_i +z_i \n < \n x_i +M \n + \ep2^{-i}$. Define the sequence $(a_i)_{i\in \N} \subset X$ by $a_i = x_i +z_i$. Then we have 
\begin{align*}
\sum_{i\in \N} \n a_i \n 
&= \sum_{i \in N} \n x_i + z_i \n \\
&\leq \sum_{i \in N} \bigg (\n x_i +M \n + \ep2^{-i} \bigg)\\
&= \sum_{i\in \N} \n x_i+M\n + \ep
\end{align*}
Since $\ep>0$ is arbitrary, it follows that $$\sum_{i\in \N} \n a_i \n \leq \sum_{i\in \N} \n x_i+M\n < \infty$$
Since $X$ is complete, $\sum\limits_{i=1}^{\infty}a_i$ converges in $X$. Define $(s_n)_{n \in \N} \subset X$ and $s \in X$ by $s_n = \sum\limits_{i =1}^n a_i$ and $s = \sum\limits_{i=1}^\infty a_i $. Since $\limn s_n = s$, and $\pi: X \rightarrow X/M$ is continuous, it follows that $\limn \pi(s_n) = \pi(s)$. Since 
\begin{align*}
\pi(s_n) 
&= \sum_{i=1}^n a_i +M\\
&= \sum_{i=1}^n x_i +M
\end{align*} 
We have that $\sum\limits_{i=1}^{\infty}x_i +M$ converges which implies that $X/M$ is complete.
\end{enumerate}
\end{proof}

\begin{ex}
Let $X,Y$ be normed vector spaces and $T \in L(X,Y)$. Then
\begin{enumerate}
\item $\ker T$ is closed
\item there exists a unique map $S :X/ \ker T \rightarrow T(X)$ such that $T = S \circ \pi$. Furthermore $S$ is a bounded linear bijection and $\n S \n = \n T \n$.
\end{enumerate}
\end{ex}

\begin{proof}
\begin{enumerate}
\item Since $T$ is continuous and $\ker T = T^{-1}(\{0\})$, we have that $\ker T$ is closed.
\item Suppose that there exists $S_1,S_2 \in L(X/ \ker T, T(X)) $ such that $T = S_1 \circ \pi$ and  $T = S_2 \circ \pi $. Let $x \in X$. Then $$S_1(x + \ker T) = S_1(\pi(x)) = T(x) = S_2(\pi(x)) = S_2(x + \ker T)$$ So $S_1 = S_2$. Therefore such a map is unique.\\
Define $S: X / \ker T \rightarrow T(X)$ by $S(x+ \ker T) = T(x)$. Then $S$ is clearly a linear bijection that satisfies $T = S \circ \pi$. Let $x \in X$ and $z \in \ker T$. Then 
\begin{align*}
\n S(x+ \ker T) \n
& = \n T(x) \n \\
& = \n T(x+z) \n \\
& \leq \n T \n \n x+ z \n
\end{align*} 
Thus $$\n S(x+ \ker T) \n \leq \n T \n \inf_{z \in \ker T}  \n x + z \n = \n T \n \n x + \ker T \n$$
So $S$ is bounded and $\n S \n \leq \n T \n$. This implies that $$\n T \n = \n S \circ \pi \n \leq \n S \n \n \pi \n = \n S \n$$
Thus $\n S \n = \n T \n$.
\end{enumerate}
\end{proof}

\begin{ex}
Let $X, Y$ be normed vector spaces. Define $\phi: L(X,Y) \times X \rightarrow Y$ by \\$\phi(T,x) = Tx$. Then $\phi$ is continuous.
\end{ex}

\begin{proof}
Let $(T_1, x_1) \in L(X,Y) \times X$ and $\ep > 0$. Choose $\del = \min \{\frac{\ep}{2(\n x_1 \n + \n T_1 \n +1)}, \frac{\sqrt{\ep}}{\sqrt{2}} \}$. Let $(t_2, x_2) \in L(X,Y) \times X$. Suppose that $$\n (T_1, x_1) - (T_2, x_2) \n = \max \{\n T_1 - T_2\n, \n x_1 -x_2 \n \} < \del$$. Then 
\begin{align*}
\n \phi(T_1, x_1) - \phi(T_2-x_2) \n
&= \n T_1 x_ - T_2 x_2 \n\\
&= \n T_1 x_1 - T_2 x_1 + T_2 x_1 - T_2 x_2 \n \\
& \leq \n (T_1 - T_2) x_1 \n + \n T_2(x_1 -x_2) \n \\
& \leq \n T_1 -T_2 \n \n x_1 \n + \n T_2 \n \n x_1 -x_2 \n \\
& \leq \n T_1 -T_2 \n \n x_1 \n + \big(\n T_1 - T_2 \n + \n T_1 \n \big)\n x_1 -x_2 \n\\
& < \del \n x_1 \n + (\del + \n T_1 \n ) \del \\
&= \del (\n T_1 \n + \n x_1 \n) + \del^2\\
& < \frac{\ep}{2} + \frac{\ep}{2}\\
&= \ep
\end{align*}
So $\phi$ is continuous.
\end{proof}

\begin{ex}
Let $X$ be a normed vector space and $M \subset X$ a subspace. Then $\overline{M}$ is a subspace.
\end{ex}

\begin{proof}
Let $x,y \in \overline{M}$ and $\al \in \C$. Then there exist sequences $(x_n)_{n \in \N} \subset M$ and $(y_n)_{n \in \N} \subset M$ such that $x_n \conv{} x$ and $y_n \conv{} y$. Since $M$ is a subspace, $(x_n +y_n)_{n \in \N} \subset M$ and $(\al x_n)_{n \in \N} \subset M$. Since addition and scalar multiplication are continuous, we have that $x_n + y_n \conv{} x+y$ and $\al x_n \conv{} \al x$. Thus $x+y \in \overline{M}$ and $\al x \in \overline{M}$ and hence $\overline{M}$ is a subspace.
\end{proof}

\begin{ex}
Let $X,Y,Z$ be normed vector spaces, $T \in L(X,Y)$ and $S \in L(Y,Z)$. Define $ST:X \rightarrow Z$ by $STx = S(Tx)$. Then $ST \in L(X,Z)$ and $\n ST \n \leq \n S \n \n T \n$. 
\end{ex}

\begin{proof}
Clearly $ST$ is linear. Let $x \in X$. Then 
\begin{align*}
\n ST x \n
& = \n S(Tx) \n\\
& \leq \n S \n \n Tx \n \\
& \leq \n S \n \n T \n \n x \n
\end{align*}

So $\n ST \n \leq \n S \n \n T \n$.
\end{proof}

\begin{defn}
Let $X$ be a Banach space and an associative algebra. Then $X$ is said to be a Banach algebra if for each $S,T \in X$, $\n ST \n \leq \n S \n \n T \n$. If there exists $I \in X$ such that $I \neq 0$ and for each $T \in X$, $IT = TI = T$, then $X$ is said to be \textbf{unital} with identity $I$. An element $T \in X$ is said to be \textbf{invertible} if there exists $S \in X$ such that $TS=ST = I$.
\end{defn}

\begin{ex}
Let $X$ be a unital Banach algebra. Then $\n I \n \leq 1$. 
\end{ex}

\begin{proof}
Since $I \neq 0$, $\n I \n \neq 0$. By definition, $$\n I \n = \n I I \n \leq \n I \n \n I \n $$ Hence $1 \leq \n I \n $.
\end{proof}

\begin{note}
If $X$ is a Banach space, then a previous exercise implies that $L(X,X)$ equipped with composition is a unital Banach algebra where $I$ is the identity operator. It is easy to see that $\n I \n =1$.
\end{note}

\begin{note}
Let $X$ be a  Banach algebra. Then the set of invertible elements in $X$ is a group.  
\end{note}

\begin{ex}
Let $X$ be a Banach algebra. Then mulitplication is continuous. 
\end{ex}

\begin{proof}
Let $(S_1,T_1) \in X \times X$ and $\ep > 0$. Choose $\del = \min\{\frac{\ep}{2(\n S_1 \n + \n T_1 \n +1)}, \frac{\sqrt{\ep}}{\sqrt{2}}\}$. Let $(S_2, T_2) \in X \times X$. Suppose that $$\n (S_1, T_1) = (S_2, T_2) \n = \max \{ \n S_2 -S_2 \n , \n T_1 - T_2 \n\} < \del$$. Then 
\begin{align*}
\n S_1T_1 - S_2T_2 \n 
&= \n S_1T_1 - S_2T_1 +S_2T_1 - S_2T_2 \n \\
& \leq \n S_1 -S_2 \n \n T_1 \n + \n S_2 \n \n T_1 - T_2 \n \\
& \leq \n S_1 -S_2 \n \n T_1 \n + \big( \n S_1-S_2 \n + \n S_1 \n\big) \n T_1 - T_2 \n\\
& \leq \del \n T_1 \n +(\del + \n S_1 \n) \del \\
&= \del (\n S_1 \n + \n T_1 \n) + \del^2 \\
& < \frac{\ep}{2} + \frac{\ep}{2}\\
&= \ep
\end{align*}
\end{proof}

\begin{defn}
Let $X,Y$ be a normed vector spaces and $T \in L(X,Y)$. Then $T$ is said to be \textbf{invertible} or an \textbf{isomorphism} if $T$ is a bijection and $T^{-1} \in L(Y,X)$.
\end{defn}

\begin{defn}
Let $X$ be a Banach space. Define $GL(X) := \{T \in L(X,X): T \text{ is invertible}\}$.
\end{defn}

\begin{ex}
Let $X$ be a Banach space. Then 
\begin{enumerate}
\item For each $T \in L(X,X)$, if $\n I- T \n < 1$, then $T$ is invertible and $$T^{-1} = \sum_{n=0}^{\infty}(I-T)^n$$
\item For each $S,T \in L(X,X)$, if $S$ is invertible and $\n S-T \n < \n S^{-1} \n^{-1}$, then $T$ is invertible. 
\item $GL(X)$ is open.
\end{enumerate}
\end{ex}

\begin{proof}
\begin{enumerate}
\item Let $T \in L(X,X)$. Suppose that $\n I-T \n < 1$. Then $$\sum_{n=0}^{\infty} \n (I -T)^n \n  \leq \sum_{n=0}^{\infty} \n I -T \n ^{n} < \infty$$. Since $X$ is a complete, so is $L(X,X)$ and thus $\sum\limits_{n=0}^{\infty}(I-T)^n$ converges in $L(X,X)$.

Define $(S_k)_{k=0}^{\infty} \subset L(X,X)$ and $S \in L(X,X)$ by $S_k = \sum\limits_{n=0}^{k} (I-T)^n$ and \\ $S = \sum\limits_{n=0}^{\infty}(I-T)^n$. Then for each $k \in \N$,
\begin{align*}
S_k T
&= S_k - S_k(I-T) \\
&= (I-T)^0 - (I-T)^{k+1} \\
&= I - (I-T)^{k+1}
\end{align*}
and $\n S_kT - I \n \leq \n I-T \n ^{k+1}$. Since multiplication on Banach algebras is continuous, we have that $$ST = (\lim_{k \rightarrow \infty} S_k)T = \lim\limits_{k \rightarrow \infty}S_kT = I$$
Similarly $TS = I$. Thus $T$ is invertible and $T^{-1} = S \in L(X,X)$. \vspace{.5cm}\\
\item  Let $S, T \in L(X,X)$. Suppose that $S$ is invertible and $\n S-T \n < \n S^{-1} \n ^{-1}$. Then 
\begin{align*}
\n I - S^{-1}T \n
& = \n S^{-1}(S - T) \n \\
& \leq \n S^{-1} \n \n S -T \n \\
&< 1
\end{align*}
So $S^{-1}T$ is invertible. Thus $T = S (S^{-1}T)$ is invertible. \vspace{.5cm}\\
\item Let $T \in GL(X)$. Choose $\del = \n T^{-1}\n^{-1}$. By (2), $B(T, \del) \subset GL(X)$.
\end{enumerate}
\end{proof}

\begin{ex}
Let $M(X, \MA)$ denote the set of complex measures on the measurable space $(X, \MA)$. Define $\n \cdot \n : M(X, \MA) \rightarrow \Rg$ by $\n \mu \n = \vert \mu \vert (X)$. Then $\n \cdot \n$ is a norm on $M(X, \MA)$. 
\end{ex}

\begin{proof}
Let $\mu, \nu \in M(X, \MA)$ and $\al \in \C$. Exercises in a previous section tell us that $\vert \mu + \nu \vert \leq \vert \mu \vert + \vert \nu \vert$ and $\vert \al \mu \vert = \vert \al \vert \vert \mu \vert$. So clearly $\n \mu + \nu \n \leq \n \mu \n + \n \nu \n$ and $\n c \mu \n = \vert c \vert \n \mu \n $. If $\n \mu \n = 0$, then $X$ is $\mu-null$ and $\mu$ is the zero measure.
\end{proof}

\subsection{Linear Functionals}

\begin{defn}
Let $X$ be a normed vector space and $T :X \rightarrow \C$. Then $T$ is said to be a \textbf{linear functional on} $X$ if $T$ is linear and $T$ is said to be a \textbf{bounded linear functional on} $X$ if $T \in L(X, \C)$. We define the \textbf{dual space of} $X$, denoted $X^*$, by $X^* = L(X, \C)$.
\end{defn}

\begin{defn}
Let $X$ be a normed vector space and $p:X \rightarrow \R$. Then $p$ is said to be a \textbf{sublinear functional} if for each $x,y \in X$, $\lam \geq 0$, 
\begin{enumerate}
\item $p(x+y) \leq p(x) + p(y)$
\item $p(\lam x ) = \lam p(x)$
\end{enumerate}  
\end{defn}

\begin{note}
Let $X$ be a vector space and $\n \cdot \n : X \rightarrow \Rg$ be a seminorm, then $\n \cdot \n$ is a sublinear functional.
\end{note}

\begin{thm}{Hahn-Banach Theorem:}
Let $X$ be a vector space, $p:X \rightarrow \R$ a sublinear functional, $M \subset X$ a subspace and $f:M \rightarrow C$ a linear functional. If for each $x \in M$, $\vert f(x) \vert \leq p(x)$, then there exists a linear functional $F:X \rightarrow \C$ such that for each $x \in X$, $\vert F(x) \vert \leq p(x)$ and $F|_{M}=f$.
\end{thm}

\begin{ex}
Let $X$ be a normed vector space, $M \subset X$ a subspace and $f \in M^*$. Then there exists $F \in X^*$ such that $\n F \n = \n f \n$ and $F|_M = f$.  
\end{ex}

\begin{proof}
If $f =0$, Choose $F=0$. Suppose $f \neq 0$. Then $\n f \n \neq 0$ and there exists $x_0 \in M$ such that $x_0  \neq 0$. Thus $\n f \n = \sup \{ \vert f(x) \vert: x \in M \text{ and } \n x \n = 1\}$. Define $p:X \rightarrow \Rg$ by $ p(x) = \n f \n \n x \n $. Then $p$ is a sublinear functional on $X$ and for each $x \in M$, $\vert f(x) \vert \leq p(x)$. So there exists a linear functional $F:X \rightarrow \C$ such that for each $x \in X$, $\vert F(x) \vert \leq p(x) = \n f \n \n x \n$ and $F|_M = f$. Thus $F \in X^*$ with $\n F \n \leq \n f \n$. Also $$\n F \n = \sup_{\substack{ x \in X \\ \n x \n = 1}} \vert F(x) \vert \geq  \sup_{\substack{ x \in M \\ \n x \n = 1}} \vert F(x) \vert = \sup_{\substack{ x \in M \\ \n x \n = 1}} \vert f(x) \vert = \n f \n $$

So $\n F \n = \n f \n$.
\end{proof}

\begin{ex}
Let $X$ be a normed vector space, $M \subsetneq X$ a proper closed subspace and $x \in X \setminus M$. Then there exists $F \in X^*$ such that $F|_M = 0$, $\n F \n =1$ and $ F(x) = \n x+M \n \neq 0$. (Hint: Consider $f:M+\C x \rightarrow \C$ defined by $f(m+\lam x) = \lam \n x +M \n$.)
\end{ex}

\begin{proof}
Define $f:M+\C x \rightarrow \C$ as above. Clearly $f$ is linear and $f|M = 0$. Let $m \in M$ and $\lam \in \C$. If $\lam = 0$, then $\vert f(m +\lam x) \vert = 0 \leq \n m+ \lam x \n$. Suppose that $\lam \neq 0$. Then 
\begin{align*}
\vert f(m+\lam x) \vert 
& = \vert \lam \vert \n x+M \n\\
& =  \n \lam x+M \n\\
& = \inf_{z \in M} \n z+ \lam x \n \\
& \leq  \n m+ \lam x  \n \\
\end{align*} 
So $f \in (M+\C x )^*$ and $\n f \n \leq 1$. Let $\ep >0$. A previous exercise tells us that there exist $m \in M, \lam \in \C$ such that $\n m+ \lam x \n = 1$ and $\n m+ \lam x +M \n > 1- \ep$. Then 
\begin{align*}
\vert f(m + \lam x) \vert
&= \vert \lam \vert \n x+M\n\\
&=\n \lam x +M \n\\
&= \n m + \lam x +M \n\\
&> 1-\ep
\end{align*}

So $$ \n f \n = \sup_{\substack{z \in M + \C x \\ \n z \n =1}} \vert f(z) \vert \geq 1$$ Hence $\n f \n =1$. 
The same exercise also tells us that $f(x) = \n x+M\n \neq 0$. Using the previous exercise, there exists $F \in X^*$ such that $\n F \n = \n f \n = 1$ and $F|_{M+\C x} = f$.
\end{proof}

\begin{ex}
Let $X$ be a normed vector space and $x \in X$. If $x \neq 0$, then there exists $F \in X^*$ such that $\n F \n = 1$ and $F(x) = \n x \n$.
\end{ex}

\begin{proof}
Define $f:\C x \rightarrow \C$ by $f(\lam x) = \lam \n x \n$. Then $f$ is linear and $f(x) = \n x \n$. Clearly $$\sup_{\substack{z \in \C x \\ \n z \n =1}}\vert f(z) \vert = 1$$ 
So $f \in (\C x)^*$ and $\n f \n = 1$. By a previous exercise, there exists $F \in X^*$ such that $\n F \n = \n f \n =1$ and $F|_{\C x} = f$. 
\end{proof}

\begin{ex}
Let $X$ be a normed vector space. Then $X^*$ separates the points of $X$. 
\end{ex}

\begin{proof}
Let $x, y \in X$. Suppose that $x \neq y$. Then $x-y \neq 0$. The previous exercies implies that there exists $F \in X^*$ such that $\n F \n = 1$ and $$F(x) - F(y) = F(x-y) = \n x-y \n \neq 0$$ Thus $F(x) \neq F(y)$ and $X^*$ separates the points of $X$.
\end{proof}

\begin{defn}
Let $X, Y$ be metric spaces and $T : X \rightarrow Y$. Then $T$ is said to be an \textbf{isometry} if for each $x_1, x_2 \in X$, $d( Tx_1, Tx_2) = d(x_1,x_2) $.
\end{defn}

\begin{ex}
Let $X,Y$ be metric spaces and $T:X \rightarrow Y$ and isometry. Then $T$ is injective.
\end{ex}

\begin{proof}
Let $x_1, x_2 \in X$. Suppose that $Tx_1=Tx_2$. Then $0= d( Tx_1, Tx_2) = d(x_1,x_2)$. So $x_1 = x_2$. Hence $T$ is injective.
\end{proof}

\begin{note}
Let $X,Y$ be metric spaces and $T:X \rightarrow Y$ an isometry. Then $T$ is clearly continuous. If $T$ is surjective, then $T^{-1}$ is an isometry and therefore continuous. Hence $T$ is a homeomorphism.
\end{note}

\begin{ex}
Let $X$ be a normed vector space and $x \in X$. Define $\hat{x}:X^* \rightarrow \C$ by $\hat{x}(f) = f(x)$. Then $\hat{x} \in X^{**}$ and $\n \hat{x} \n = \n x \n$.
\end{ex}

\begin{proof}
Let $f,g \in X^*$ and $\lam \in \C$. Then $$\hat{x}(f+\lam g) = (f+ \lam g)(x) = f(x) + \lam g(x) = \hat{x}(f) + \lam \hat{x}(g)$$
So $\hat{x}$ is linear. For each $f \in X^*$, $$\vert \hat{x}(f) \vert = \vert f(x) \vert \leq \n x \n \n f \n $$ Hence $\hat{x} \in X^{**}$ with $\n \hat{x} \n \leq \n x \n$. If $x=0$, then $\hat{x} = 0$ and $\n \hat{x} \n = \n x \n$. Suppose that $x \neq 0$. Then a previous exercise implies that there exists $F \in X^*$ such that $\n F \n =1$ and $F(x) = \n x \n$. Then we have that $$\sup_{\substack{f \in X^* \\ \n f \n = 1 } } \vert \hat{x}(f) \vert  = \sup_{\substack{f \in X^* \\ \n f \n = 1 }}  \vert f(x) \vert \geq \vert F(x) \vert = \n x \n$$
Hence $\n \hat{x} \n = \n x \n$.
\end{proof}


\begin{ex}
Let $X$ be a normed vector space. Define $\phi : X \rightarrow X^{**}$ by $\phi(x) = \hat{x}$. Then $\phi$ is a linear isometry. 
\end{ex}

\begin{proof}
Let $x,y \in X$ and $\lam \in \C$. Then for each $f \in X^*$, we have that 
\begin{align*}
\phi(x+ \lam y)(f) 
&= \widehat{x+ \lam y}(f) \\
&= f(x+\lam y) \\
&= f(x) + \lam f(y) \\
&= \hat{x}(f) + \lam \hat{y}(f)\\
&= \phi(x)(f) + \lam \phi(y)(f)
\end{align*} 
So $\phi(x+ \lam y) = \phi(x) + \lam \phi(y)$ and $\phi$ is linear. The previous exercise tells us that 
\begin{align*}
\n \phi(x) - \phi(y) \n 
&= \n \phi(x-y)\n \\
&= \n \widehat{x-y} \n = \n x-y \n
\end{align*}
So $\phi$ is an isometry.
\end{proof}

\begin{defn}
Let $X$ be a normed vector space and define $\phi:X \rightarrow X^{**}$ as above. We define $\widehat{X} = \phi(X) \subset X^{**}$. Since $\widehat{X}$ and $X$ are isomorphic, we may identify $X$ as a subset of $X^{**}$. 
\end{defn}

\begin{defn}
Let $X$ be a normed vector space and define $\phi:X \rightarrow X^{**}$ as above. Then $X$ is said to be reflexive if $\phi$ is surjective. In this case $\phi$ is then an isomorphism
\end{defn}

\begin{ex}
Let $X$ be a normed vector space and $f:X \rightarrow \C$ a linear functional on $X$. Then $f$ is bounded iff $\ker f$ is closed. 
\end{ex}

\begin{proof}
Suppose that $f$ is continuous. Since $\{0\}$ is closed, we have that $\ker f = f^{-1}(\{0\})$ is closed. Conversely, suppose that $\ker f$ is closed. If $\ker f = X$, then $f =0$ and $f$ is continuous. Suppose that $\ker f \neq X$. Then $\ker f$ is a proper, closed subspace of $X$. A previous exercise tells us that there exists $x \in X$ such that $\n x \n = 1$ and $\n x + \ker f \n > \frac{1}{2}$. Let $y \in X$. Suppose that $\n y \n < \frac{1}{2}$. Then for each $z \in \ker f$, 
\begin{align*}
\n z -  (x+y)\n 
& = \n (z-x) -y \n \\
& \geq \n z-x \n - \n y \n \\
& > \frac{1}{2} - \frac{1}{2} \\
&=0
\end{align*}

So $x+y \not \in \ker f$. Therefore $f(B(x,\frac{1}{2})) \cap \{0\} = \varnothing$. If $f(B(x,\frac{1}{2})) $ is unbounded, then $f(B(x,\frac{1}{2})) = \C$ by linearity. This is a contradiction since $0 \not \in f(B(x,\frac{1}{2}))$. So There exists $s > 0$ such that $f(B(x,\frac{1}{2})) \subset B(0,s)$ and thus $f$ is bounded. 
\end{proof}

\begin{ex}
Let $X$ be a normed vector space. 
\begin{enumerate}
\item Let $M \subsetneq X$ be a proper closed subspace of $X$ and $x \in X \setminus M$. Then $M + \C x$ is closed.
\item Let $M \subset X$ be a finite dimensional subspace of $X$. Then $M$ is closed.
\end{enumerate}
\end{ex}

\begin{proof}
\begin{enumerate}
\item Let $y \in X$ and $(y_n)_{n \in \N} \subset M+ \C x$. Suppose that $y_n \conv{} y$. If $y \in M$, then $y \in M+ \C x$. Suppose that $y \not \in M$. For each $n \in \N$, there exists $m_n \in M$ and $\lam_n \in \C$ such that $y_n = m_n + \lam_nx$. A previous exercise tells us that there exists $F \in X^*$ such that $\n F \n = 1$, $F|_M = 0$ and $F(x) = \n x+M \n \neq 0$. Since $F$ is continuous, $F(y_n) \conv{} F(y)$. Since for each $n \in \N$, $$F(y_n) = F(m_n + \lam_n x) = F(m_n)+ \lam_n (F_x) = \lam_n F(x)$$ we have that $\lam_n F(x) \conv{} F(y)$. Since $F(x) \neq 0$, this implies that $\lam_n \conv{} F(x)^{-1} F(y)$. It follows that $\lam_n x \conv{}F(x)^{-1}F(y)x$. Since  for each $n \in \N$, $m_n = y_n - \lam_nx$, we know that $m_n \conv{} y-F(x)^{-1}F(y)x$. Since $(m_n)_{n \in \N} \subset M$ and $M$ is closed, we have that $y-F(x)^{-1}F(y)x \in M$ and therefore $y \in M+\C x$. Hence $M+\C x$ is closed. \vspace{.5cm}\\
\item If $M = X$, then $M$ is closed. Suppose that $M \neq X$. Let $(x_i)_{i=1}^n$ be a basis for $M$. Define $N_0 = \{0\}$ and for each $i =1,2, \cdots, n$, define $N_i = N_{i-1}+\C x_i$. Since $N_0$ is a proper closed subpace of $X$ and $x_1 \in X \setminus N_0$, (1) implies that $N_1$ is closed. Proceed inductively to obtain that $M = N_n$ is closed.
\end{enumerate}
\end{proof}

\begin{ex}
Let $X$ be an infinite-dimensional normed vector space. 
\begin{enumerate}
\item There exists a sequence $(x_n)_{n\in \N} \subset X$ such that for each $m, n \in \N$, $\n x_n \n = 1$ and if $m \neq n$, then $\n x_m - x_n \n > \frac{1}{2}$.
\item $X$ is not locally compact. 
\end{enumerate}
\end{ex}

\begin{proof}
\begin{enumerate}
\item Define $N_0 = \{0\}$. Then $N_0$ is a closed proper subspace of $X$. Choose $x_1 \in X$ such that $\n x_1 \n = 1$. Using the results of previous exercises, we proceed inductively. For each $n \geq 2$ we define $N_{n-1} = \text{span}(x_1, x_2, \cdots, x_{n-1})$. Then $N_{n-1}$ is a closed proper subspace of $X$. Thus we may choose $x_n \in X$ such that $\n x_n \n = 1$ and $\n x_n + N_{n-1} \n >  \frac{1}{2}$. Let $m,n \in \N$. Suppose that $m<n$. Then $x_m \in N_{n-1}$. Thus $\n x_n - x_m \n \geq \n x_n + N_{n-1} \n > \frac{1}{2}$\vspace{.5cm}\\
\item Suppose that $X$ is locally compact. Then $\overline{B(0,1)}$ is compact and therefore sequentially compact. Using $(x_n)_{n \in \N} \subset \overline{B(0,1)}$ defined in (1), we see that there exists a subsequence $(x_{n_k})_{k \in \N}$, $x \in \overline{B(0,1)}$ such that $x_{n_k} \conv{} x$. Then $(x_{n_k})_{k \in \N}$ is Cauchy. So there exists $N \in N$ such that for each $j, k \in \N$, if $j, k \geq N$, then $\n x_{n_j} - x_{n_k} \n < \frac{1}{2}$. Then $\n x_{n_N} - x_{n_{N+1}} \n  < \frac{1}{2}$. This is a contradiction since by construction, $\n x_{n_N} - x_{n_{N+1}} \n  > \frac{1}{2}$. Thus $X$ is not locally compact.
\end{enumerate}
\end{proof}

\begin{ex}
Let $X,Y$ be normed vector spaces and $T \in L(X,Y)$. 
\begin{enumerate}
\item Define the \textbf{adjoint of $T$}, $T^*:Y^* \rightarrow X^*$ by $T^*(f) = f \circ T$. Then $T^* \in L(Y^*, X^*)$.
\item Applying the result from (1) twice, we have that $T^{**} \in L(X^{**},Y^{**})$. We have that for each $x \in X$, $T^{**}(\hat{x}) = \widehat{T(x)}$.
\item $T^*$ is injective iff $T(X)$ is dense in $Y$.
\item If $T^*(Y^*)$ is dense in $X^*$, then $T$ is injective. The converse is true if $X$ is reflexive.
\end{enumerate}
\end{ex}

\begin{proof}
\begin{enumerate}
\item Let $f \in Y^*$. Then $\n T^* (f) \n = \n f \circ T \n \leq  \n T \n  \n f \n$. So $T^* \in L(Y^*, X^*)$ with $\n T^* \n \leq \n T \n$.\vspace{.5cm}\\
\item Let $x \in X$. Let $f \in Y^*$. Then 
\begin{align*}
T^{**}(\hat{x})(f) 
&= \hat{x} \circ T^{*}(f) \\
&= \hat{x}(T^* (f)) \\
&= \hat{x}(f \circ T) \\
&= f \circ T (x) \\
&= f(T(x)) \\
&= \widehat{T(x)}(f)
\end{align*} 
Hence $T^{**}(\hat{x}) = \widehat{T(x)}$.\vspace{.5cm}\\
\item Suppose that $T(X)$ is not dense in $Y$. Then $\overline{T(X)} \neq Y$. So $T(X)$ is a proper closed subspace of $Y$ and there exists $y \in Y$ such that $y \not \in \overline{T(X)}$. By a previous exercise, there exists $f \in Y^*$ such that $f(y) = \n y+\overline{T(X)} \n \neq 0$, $\n f \n =1$ and $f|_{\overline{T(X)}} = 0$. Let $x \in X$. Then $T^*(f)(x) = f \circ T(x) = 0$. Hence $T^*(f) = 0 = T^*(0)$. Since $f \neq 0$, $T^*$ is not injective.\\ Now suppose that $T(X)$ is dense in $Y$. Let $f,g \in Y^*$. Define $h \in Y^*$ by $h = f-g$ Suppose that $T*(f) = T^*(g)$ Then $T^*(h) = 0$. So for each $x \in X$, $h(T(x)) = 0$. Let $y \in Y$ and $\ep >0$. By continuity, there exists $\del > 0 $ such that for each $y' \in Y$, if $\n y - y' \n < \del$, then $\n h(y) - h(y') \n < \ep$. Since $T(X)$ is dense in $Y$, there exists $x \in X$ such that $\n y - T(x) \n < \del$. Thus 
\begin{align*}
\n h (y) \n 
&\leq \n h(y) - h(T(x)) \n + \n h(T(x)) \n\\
& = \n h(y) - h(T(x)) \n  \\
& < \ep
\end{align*} 
Since $\ep > 0$ is arbitrary, $\n h(y) \n = 0$. This implies that $h(y) = 0$ and therefore $f(y) = g(y) $. Since $y \in Y$ is arbitrary, $f=g$ and $T^*$ is injective. \vspace{.5cm}\\
\item For the sake of contradiction, suppose that $T^*(Y^*)$ is dense in $X^*$ and $T$ is not injective. Then there exist $x_1, x_2 \in X$ such that $x_1 \neq x_2$ and $T(x_1) = T(x_2)$. Define $x = x_1-x_2$. Then $x \neq 0$ and $T(x) = 0$. A previous exercise implies that there exists $F \in X^*$ such that $F(x) = \n x\n \neq 0$ and $\n F \n = 1$. Let $\ep >0$. Choose $g \in Y^*$ such that $\n F - T^*(g) \n < \ep$. Then 
\begin{align*}
\n x \n 
&= \vert F(x) \vert \\
&\leq \vert F(x) - T^*(g)(x) \vert + \vert T^*(g)(x) \vert \\
& < \ep \n x \n + \vert g(T(x)) \vert\\
&= \ep \n x \n
\end{align*}

Since $\ep > 0$ is arbitrary, we have that $\n x \n =0$ which is a contradiction. Hence if $T^*(Y^*) $ is dense in $X^*$, then $T$ is injective. \vspace{.5cm}\\ 
Now, suppose that $X$ is reflexive and $T$ is injective. Let $\phi_1, \phi_2 \in X^{**}$. Suppose that $T^{**}(\phi_1) = T^{**}(\phi_2)$. Then $T^{**}(\phi_1 - \phi_2) = 0$. Since $X$ is reflexive, there exist $x_1, x_2 \in X$ such that $\phi_1 = \hat{x_1}$ and $\phi_2 = \hat{x_2}$. Define $x = x_1 - x_2$. Then $T^{**}(\hat{x}) = 0$. So for each $f \in Y^*$, 
\begin{align*}
T^{**}(\hat{x})(f) 
&= \hat{x} \circ T^*(f)\\
&= \hat{x}( T^*(f))\\
&= \hat{x} (f \circ T)\\
&= f \circ T(x)\\
&= f(T(x))\\
&= 0 
\end{align*}
Suppose that $T(x) \neq 0$. Then a previous exercise implies that there exists $g \in Y^*$ such that $g(T(x)) = \n T(x) \n \neq 0$ and $\n g \n  = 1$. This is a contradiction since $g(T(x)) = 0$. So $T(x) = 0$. Since $T$ is injective, this implies that $x = 0$. Hence $\hat{x}=0$ and thus $\phi_1 = \phi_2$. Thus $T^{**}$ is injective. By (3), we have that $T^*(Y^*)$ is dense in $X^*$.
\end{enumerate}
\end{proof}

\begin{ex}
Let $X$ be a normed vector space. Then $X$ is reflexive iff $X^*$ is reflexive. 
\end{ex}

\begin{proof}
Suppose that $X$ is reflexive. Let $\al \in X^{***}$. Define $f :X \rightarrow \C$ by $f(x) = \al(\hat{x})$. Clearly $f$ is linear and a previous exercise tells us that for each $x \in X$, 
\begin{align*}
\vert f(x) \vert 
& \leq \n \al \n \n \hat{x} \n\\
&= \n \al \n \n x \n
\end{align*}
So $f \in X^*$.
Let $\phi \in X^{**}$. Since $X$ is reflexive, there exists $x \in X$ such that $\phi = \hat{x}$. Then 
\begin{align*}
\al(\phi)
&= \al(\hat{x})\\
&= f(x)\\
&= \hat{x}(f)\\
&= \hat{f}(\hat{x})\\
&= \hat{f}(\phi)
\end{align*}
Hence $\al = \hat{f}$. Thus the map $X^* \rightarrow X^{***}$ given by $f \mapsto \hat{f} $ is surjective and so $X^{*}$ is reflexive.\vspace{.5cm}\\
Conversely, suppose that $X^*$ is reflexive. Since $\phi:X \rightarrow X^{**}$ given by $\phi(x) = \hat{x}$ is an isometry, $\widehat{X} \subset X^{**}$ is closed. For the sake of contradiction, suppose that $\widehat{X} \neq X^{**}$. Then there exists $\al \in X^{**}$ such that $\al \not \in \widehat{X}$. Thus there exists $F \in X^{***}$ such that $\n F \n = 1$, $F(\al) = \n \al + \widehat{X} \n \neq 0$ and $F|_{\widehat{X}}=0$. Since $X^*$ is reflexive, there exists $f \in X^*$ such that $F = \hat{f}$. A previous exercise tells us that $\n f \n = \n \hat{f} \n = \n F \n = 1$. Since for each $x \in X$, $f(x) = \hat{x}(f) = \hat{f}(\hat{x}) = F(\hat{x}) = 0$, we have that $f = 0$. Thus $\n f \n = 0$, a contradiction. So $\widehat{X} = X^{**}$ and $X$ is reflexive.

\end{proof}

\subsection{The Baire Category Theorem and Consequences}

\begin{thm}
Let $X, Y$ be Banach spaces and $T\in L(X,Y)$. If $T$ is surjective, then $T$ is open.
\end{thm}

\begin{cor}
Let $X, Y$ be Banach spaces and $T \in L(X,Y)$. If $T$ is a bijection, then $T^{-1} \in L(X,Y)$.
\end{cor}

\begin{defn}
Let $X,Y$ be sets and $f:X \rightarrow Y$. We define the \textbf{graph of f}, $\Gam(f)$, by $\Gam(f) = \{(x,y) \in X \times Y: f(x) = y\}$.
\end{defn}

\begin{thm}
Let $X, Y$ be Banach spaces and $T:X \rightarrow Y$ a linear map. If $\Gam(T)$ is closed, then $T \in L(X,Y)$.  
\end{thm}

\begin{note}
We recall that $\Gam(T)$ is closed iff for each $(x_n)_{n \in \N} \subset X$, $x \in X$ and $y \in Y$ if $x_n \conv{} x$ and $T(x_n) \conv{} y$, then $T(x) = y$. 
\end{note}

\begin{thm}

Let $X, Y$ be Banach spaces and $S \subset L(X,Y)$. If for each $x \in X$, $$\sup_{T \in S} \n Tx \n < \infty$$ then $$\sup_{T \in S} \n T \n < \infty$$
\end{thm}

\begin{ex}
Let $\mu$ be counting measure on $(N, \MP(\N))$. Define $h: \N \rightarrow \N$ and $ \nu$ on $(N, \MP(\N))$ by $h(n) = n$ and $d \nu = h d \mu$. Define $X=L^1(\nu)$ and $Y = L^1(\mu)$. Equip both $X$ and $Y$ with the $L^1$ norm with respect to $\mu$. 
\begin{enumerate}
\item We have that $X$ is a proper subspace of $Y$ and therefore $X$ is not complete.
\item Define $T: X \rightarrow Y$ by $Tf(n) = nf(n)$. Then $T$ is linear, $\Gam(T)$ is closed, and $T$ is unbounded.
\item Define $S:Y \rightarrow X$ by $Sg(n) = \frac{1}{n}g(n)$. Then $S \in L(Y,X)$, $S$ is surjective and $S$ is not open. 
\end{enumerate}
\end{ex}

\begin{proof}
\begin{enumerate}
\item Note that for each $f: \N \rightarrow \C$, 
\begin{align*}
{\n f \n}_{\mu, 1}
&= \sum_{n=1}^{\infty} \vert f(n) \vert  \\
& \leq \sum_{n=1}^{\infty} n \vert f(n) \vert  \\
& = \n f \n_{\nu,1} 
\end{align*} 
Hence $X$ is a subspace of $Y$. Define $f : \N \rightarrow \C$ by $f(n) = \frac{1}{n^2}$. Then $$\n f \n_{\mu, 1} = \sum_{n=1}^{\infty} \frac{1}{n^2} < \infty$$ So  $f \in Y$. However $$\n f \n_{\nu, 1} = \sum_{n=1}^\infty \frac{1}{n} = \infty$$ So $f \not \in X$. Thus $X$ is a proper subspace of $Y$. Let $g \in Y$ and $\ep >0$. Since the simple functions are dense in $L^1(\mu)$, there exists $\phi \in L^1(\mu)$ such that $\phi$ is simple and $\n g - \phi \n_{\mu ,1} < \ep$. Then there exist $(c_i)_{i=1}^k \subset \C$ and $ (E_i)_{i=1}^k \subset \MP(\N)$ such that for each $i =1,2,\cdots, k$, $E_i$ is finite and  $$\phi = \sum_{i=1}^kc_i \chi_{E_i}$$ Define $c = \max\{\vert c_i \vert: i=1,2,\cdots k\}$ and $m = \max \bigcup_{i=1}^k E_i$. Then 
\begin{align*}
\n \phi \n_{\nu,1} 
&=  \sum_{n=1}^m n \vert \phi(n) \vert \\
& \leq \sum_{n=1}^m  mc \\
& = c m^2 \\
& < \infty
\end{align*}

Hence $\phi \in X$ and $X$ is dense in $Y$. Since $X$ is a dense, proper subspace, it is not closed. Since $Y$ is complete and $X \subset Y$ is not closed, we have that $X$ is not complete.
\item Clearly $T$ is linear. Let $(f_j)_{j \in \N} \subset X$, $f \in X$ and $g \in Y$. Suppose that $f_j \conv{L^1(\mu)} f$ and $Tf_j \conv{L^1(\mu)} g$. 

Note that for each $j \in \N$ and $n \in \N$, $$\vert f_j(n) - f(n) \vert \leq \sum_{n =1}^{\infty}\vert f_j(n) - f(n) \vert = \n f_j-f \n_{\mu, 1}$$ and $$\vert nf_j(n) - g(n) \vert \leq \sum_{n =1}^{\infty}\vert nf_j(n) - g(n) \vert = \n Tf_n - g\n_{\mu, 1}$$  

Thus for each $n \in \N$, $f_j(n) \conv{j} f(n)$ and $nf_j(n) \conv{j} g(n)$. This implies that for each $n \in \N$, $nf(n) = g(n)$. Thus $Tf = g$ which implies that $\Gam(T)$ is closed. Suppose, for the sake of contradiction, that $T$ is bounded. Then there exists $C \geq 0$ such that for each $f \in X$, $\n Tf \n_{\mu,1} \leq C \n f \n_{\mu, 1}$. Choose $n \in \N$ such that $n > C$. Define $f: \N \rightarrow \C$ by $f = \chi_{\{n\}}$. As established above, $S^+ \subset L^1(\mu)$. Then $\n f \n_{\mu,1} = 1$ and
\begin{align*}
\n Tf \n_{\mu,1}
& = n \\
&> C\\
& = C \n f \n_{\mu,1}
\end{align*}
which is a contradiction. So $T$ is unbounded.
\item Clearly $S$ is linear. Let $g \in Y$. Then \begin{align*}
\n Sg \n_{\mu,1} 
&= \sum_{n =1}^{\infty} \frac{1}{n} \vert g(n) \vert \\
& \leq  \sum_{n =1}^{\infty} \vert g(n) \vert \\
& = \n g \n_{\mu,1}
\end{align*}
So $S$ is bounded and $\n S \n \leq 1$. Thus $S \in L(Y,X)$. Let $f \in X$. Define $g: \N \rightarrow \C$ by $g(n) = nf(n)$. By defnition, $g \in Y$ and we have that
\begin{align*}
Sg(n) 
&= \frac{1}{n}g(n) \\
& = f(n)
\end{align*}
Hence $Sg =f$ and thus $S$ is surjective. Let $g \in Y$. Suppsose that $Sg = 0$. Then $$\sum_{n=1}^{\infty} \frac{1}{n}\vert g(n)\vert =\n Sg \n  = 0$$ Thus for each $n \in \N$, $g(n) = 0$. Hence $\ker g = \{0\}$ and $g$ is injective. Note that $S^{-1} = T$. If $g$ is open, then $T$ is continuous which as shown above is a contradiction. So $g$ is not open. 
\end{enumerate}
\end{proof}

\begin{ex}
Let $X = C^1([0,1])$ and $Y=C([0,1])$. Equip both $X$ and $Y$ with the uniform norm. 
\begin{enumerate}
\item Then $X$ is not complete
\item Define $T: X \rightarrow Y$ by $Tf = f'$. Then $\Gam(T)$ is closed and $T$ is not bounded. 
\end{enumerate}
\end{ex}

\begin{proof}
\begin{enumerate}
\item Recall that for each $a,b \geq 0$ and $p \in \N$, $$(a^{\frac{1}{p}}+b^{\frac{1}{p}})^p = \sum_{n=0}^p  {p \choose n} a^{\frac{n}{p}}b^{\frac{p-n}{p}} \geq a + b$$ Thus $(a+b)^{\frac{1}{p}} \leq a^{\frac{1}{p}}+b^{\frac{1}{p}}$.\\
For each $n \in \N$, define $f_n: [0,1] \rightarrow \C$ by $f_n(x) = \sqrt{(x-\frac{1}{2})^2+ \frac{1}{n^2}}$. Then $(f_n)_{n \in \N} \subset X$. Define $f:[0,1] \rightarrow \C$ by $f(x) = \vert x-\frac{1}{2}\vert$. Then $f \in Y \cap X^c$. Note that for each $n \in \N$, $f \leq f_n$. Our observation above implies that for each $x \in X$,
\begin{align*}
f_n(x) 
&= \bigg[ (x-\frac{1}{2})^2 + \frac{1}{n^2} \bigg]^{\frac{1}{2}}\\
& \leq \vert x-\frac{1}{2} \vert + \frac{1}{n}
\end{align*}
Thus $0 \leq f_n - f \leq \frac{1}{n} $. This implies that $f_n \convt{u} f$. Since $f \not \in X$, $X$ is not complete. \vspace{.5cm}\\
\item Let $(f_n)_{n \in \N} \subset X$, $f \in X$ and $g \in Y$. Suppose that $f_n \convt{u} f$ and $Tf_n \convt{u} g$. Let $x \in [0,1]$. Then $f_n(x) \conv{} f(x)$ and $f_n(0) \conv{} f(0)$ and $f_n' \conv{u} g$. Applying the DCT to this sequence of integrable functions that converges uniformly to an integrable function on a finite measure space (a previous exercise) we have that
\begin{align*}
f_n(x) - f_n(0) 
&= \int_{[0,x]} f_n' dm \\
& \conv{} \int_{[0,x]} g dm \\ 
\end{align*} 
Since $f_n(x) - f_n(0) \conv{} f(x) - f(0)$, we know that $$f(x) - f(0) = \int_{[0,x]} g dm$$. Thus $Tf = g$ and $\Gam(T)$ is closed. \\
Suppose for the sake of contradiction that $T$ is bounded. Then there exists $C \geq 0$ such that for each $f \in X$, $\n T f \n \leq C \n f \n$. Choose $n \in \N$ such that $n > C$. Define $f \in X$ by $f(x) = x^n$. Then $\n f \n = 1$ and 
\begin{align*}
\n Tf \n 
&= \n f' \n \\
&= n \\
&> C \\
&= C \n f \n
\end{align*}
which is a contradiction. So $T$ is not bounded.
\end{enumerate}
\end{proof}

\begin{ex}
Let $X, Y$ be Banach spaces and $T \in L(X,Y)$. Then $X/\ker T \cong T(X)$ iff $T(X)$ is closed.
\end{ex}

\begin{proof}
Since $X$ is a banach space and $T$ is continuous, we have that $\ker T$ is closed and $X/ \ker T$ is a Banach space. Suppose that $X/ \ker T \cong T(X)$. Then $T(X)$ is complete. Since $Y$ is complete, this implies that $T(X)$ is closed. \\
Conversely Suppose that $T(X)$ is closed. Then $T(X)$ is complete. Define $S: X/ \ker T \rightarrow T(X)$ by $S(x + \ker T) = T(x)$. A previous exercise tells us that the map $S: X/ \ker T \rightarrow T(X)$ defined by $S(x + \ker T) = T(x)$ is a bounded linear bijection. Since $T(X)$ is complete and $S$ is surjective, $S^{-1}$ is bounded and thus $S$ is an isomorphism.   
\end{proof}

\begin{ex}
Let $X$ be a separable Banach space. Define $B_X = \{x \in X: \n x \n < 1\}$. Let $(x_n)_{n \in \N} \subset B_X $ a dense subset of the unit ball and $\mu$ the counting measure on $(\N, \MP(\N))$. Define $T: L^1(\mu) \rightarrow X$ by $$Tf = \sum_{n=1}^{\infty}f(n)x_n$$ Then 
\begin{enumerate}
\item $T$ is well defined and $T \in L(L^1(\mu), X)$
\item $T$ is surjective
\item There exists a closed subspace $K \subset L^1(\mu)$ such that $L^1(\mu)/K \cong X$ 
\end{enumerate} 
\end{ex}

\begin{proof}
\begin{enumerate}
\item Let $f \in L^1(\mu)$. Since $X$ is complete and 
\begin{align*}
\sum_{n=1}^{\infty}\n f(n)x_n \n
& = \sum_{n=1}^{\infty} \vert f(n) \vert \n x_n \n \\
& \leq \sum_{n=1}^{\infty} \vert f(n) \vert \\
&< \infty 
\end{align*}
we have that $\sum_{n=1}^{\infty} f(n)x_n $ converges and thus $Tf \in X$. Hence $T$ is well defined. \vspace{.5cm}\\
Clearly $T$ is linear. Let $f \in L^1(\mu)$. Then
\begin{align*}
\n Tf \n 
&= \n  \sum_{n=1}^{\infty} f(n)x_n \n \\
& \leq \sum_{n=1}^{\infty} \n f(n)x_n \n \\
& \leq \sum_{n=1}^{\infty} \vert f(n) \vert \\
&= \n f \n_1
\end{align*}
So $T$ is bounded with $\n T \n \leq 1$.\vspace{.5cm}\\
\item Let $x \in X$. Suppose that $\n x \n < 1$. Then $x \in B_X$. So there exists $n_1 \in \N$ such that $\n x - x_{n_1} \n < \frac{1}{2}$. Then $2(x-x_{n_1}) \in B_X$. Since for each $j \in \N$, $B_X\setminus (x_n)_{n=1}^j$ is dense in $B_X$, there exists $n_2 \in \N$ such that $x_{n_2} \not \in (x_n)_{n=1}^{n_1}$ and $\n 2(x- x_{n_1}) - x_{n_2} \n < \frac{1}{2}$ which implies that $\n x- (x_{n_1} - \frac{1}{2}x_{n_2}) \n < \frac{1}{4}$. \vspace{.5cm}\\ 
Proceed inductively to obtain a subsequence $(x_{n_k})_{k \in \N}$ such that for each $k \geq 2$, $x_{n_k} \not \in (x_n)_{n=1}^{n_{k-1}}$ and $\n x - \sum_{j=1}^k 2^{1-j}x_{n_j} \n < \frac{1}{2^k}$. Then $x = \sum_{k=1}^{\infty}2^{1-k}x_{n_k}$. \vspace{.5cm} \\ 
Define $f:\N \rightarrow \C$ by $f = \sum_{k=1}^{\infty}2^{1-k}\chi_{\{n_k\}}$. Then $\n f \n_1 = \sum_{k=1}^{\infty}2^{1-k}< \infty$, so $f \in L^1(\mu)$ and $Tf = \sum_{k=1}^{\infty}2^{1-k}x_{n_k} = x$. Now, suppose that $\n x \n \geq 1$, then $\frac{1}{2\n x \n}x \in B_X$. The above argument shows that there exists $f \in L^1(\mu)$ such that $Tf = \frac{1}{2\n x \n}x$. Then $2 \n x \n f \in L^1(\mu)$ and $T(2 \n x \n f) = 2 \n x \n Tf =x$. \\
So for each $x \in X$, there exists $f \in L^1(\mu)$ such that $Tf = x$ and thus $T$ is surjective. 
\item Since $X$ is a Banach space and $T$ is surjective, the previous exercise implies that $L^1(\mu)/\ker T \cong X$. 
\end{enumerate}
\end{proof}

\begin{ex}
Let $X, Y$ be Banach spaces and $T:X \rightarrow Y$ a linear map. If for each $f \in Y^*$, $f \circ T \in X^*$, then $T \in L(X,Y)$. 
\end{ex}

\begin{proof}
Suppose that for each $f \in Y^*$, $f \circ T \in X^*$. Let $x \in X$, 
\end{proof}




























\newpage

\section{Probability}
\subsection{Distributions}

\begin{defn}
Let $\Om$ be a set and $\MP \subset \MP(X)$. Then $\MP$ is said to be a \textbf{$\pi$-system} on $\Om$ if for each $A,B \in \MP$, $A \cap B \in \MP$.
\end{defn}

\begin{defn}
Let $Om$ be a set and $\ML \subset \MP(\Om)$. Then $\ML$ is said to be a \textbf{$\lam$-system} on $\Om$ if 
\begin{enumerate}
\item $\ML \neq \varnothing$
\item for each $A \in \ML$, $A^c \in \ML$
\item for each $(A_n)_{n \in \N} \subset \ML$, if $(A_n)_{n \in \N}$ is disjoint, then $\bigcup\limits_{n \in \N}A_n \in \ML$
\end{enumerate}
\end{defn}

\begin{ex}
Let $\Om$ be a set and $\ML$ a $\lam$-system on $\Om$. Then 
\begin{enumerate}
\item $\Om, \varnothing \in \ML$
\end{enumerate} 
\end{ex}

\begin{proof}
Straightforward.
\end{proof}

\begin{defn}
Let $\Om$ be a set and $\MC \subset \MP(\Om)$. Put $$\MS = \{\ML \subset \MP(\Om): \ML \text{ is a }\lam\text{-system on }\Om \text{ and } \MC \subset \ML\}$$ We define the \textbf{$\lam$-system on $\Om$ generated by $\MC$}, $\lam(\MC)$, to be $$\lam(\MC) = \bigcap_{\ML \in \MS}\ML$$
\end{defn}

\begin{ex}
Let $\Om$ be a set and $\MC \subset \MP(\Om)$. If $\MC$ is a $\lam$-system and $\MC$ is a $\pi$-system, then $\MC$ is a $\sig$-algebra.
\end{ex}

\begin{proof}
Suppose that $\MC$ is a $\lam$-system and $\MC$ is a $\pi$-system. Then we need only verify the third axiom in the definition of a $\sig$-algebra. Let $(A_n)_{n \in \N} \subset \MC$. Define $B_1 = A_1$ and for $n \geq 2$, define $B_n = A_n \cap \bigg( \bigcup\limits_{k=1}^{n-1}A_k \bigg)^c = A_n \cap \bigg( \bigcap\limits_{k=1}^{n-1}A_k^c \bigg) \in \MC$. Then $(B_n)_{n \in \N}$ is disjoint and therefore $\bigcup\limits_{n \in \N}A_n = \bigcup\limits_{n \in \N}B_n \in \MC$.
\end{proof}

\begin{thm}(Dynkin's Theorem) \\
Let $\Om$ be a set.
\begin{enumerate}
\item Let $\MP$ be a $\pi$-system on $\Om$ and $\ML$ a $\lam$-system on $\Om$. If $\MP \subset \ML$, then $\sig(\MP) \subset \ML$.
\item Let $\MP$ be a $\pi$-system on $\Om$. Then $\sig(\MP) = \lam(\MP)$
\end{enumerate} 

\end{thm}

\begin{ex}
Let $(\Om, \MF)$ be a measurable space and $\mu, \nu$ probability measures on $(\Om, \MF)$. Put $\ML_{\mu,\nu} = \{A \in \MF: \mu(A) = \nu(A)\}$. Then $\ML_{\mu, \nu}$ is a $\lam$-system on $\Om$.
\end{ex}

\begin{proof}\
\begin{enumerate}
\item $\varnothing \in \ML_{\mu, \nu}$.
\item Let $A \in \ML_{\mu, \nu}$. Then $\mu(A) = \nu(A)$. Thus 
\begin{align*}
\mu(A^c) 
&= 1-\mu(A) \\
&= 1 -\nu(A) \\
&= \nu(A^c)
\end{align*}
So $A^c \in \ML_{\mu, \nu}$. 
\item Let $(A_n)_{n \in \N} \subset \ML_{\mu, \nu}$. So for each $n \in \N$, $\mu(A_n) = \nu(A_n)$.  Suppose that $(A_n)_{n \in \N}$ is disjoint. Then 
\begin{align*}
\mu\bigg(\bigcup_{n \in \N} A_n\bigg) 
&= \sum_{n \in \N} \mu(A_n) \\
&= \sum_{n \in \N} \nu(A_n) \\
&= \nu\bigg(\bigcup_{n \in \N} A_n\bigg) 
\end{align*}
Hence $\bigcup_{n \in \N} A_n \in \ML_{\mu, \nu}$.
\end{enumerate}
\end{proof}

\begin{ex}
Let $(\Om, \MF)$ be a measurable space, $\mu, \nu$ probability measures on $(\Om, \MF)$ and $\MP \subset \MA$ a $\pi$-system on $\Om$. Suppose that for each $A \in \MP$, $\mu(A) = \nu(A)$. Then for each $A \in \sig(\MP)$, $\mu(A) = \nu(A)$.
\end{ex}

\begin{proof}
Using the previous exercise, we see that $\MP \subset \ML_{\mu, \nu}$. Dynkin's theorem implies that $\sig(\MP) \subset \ML_{\mu, \nu}$. So for each $A \in \sig(\MP)$, $\mu(A) = \nu(A)$.
\end{proof}








\begin{defn}
Let $F: \R \rightarrow \R$. Then $F$ is said to be a \textbf{probability distribution function} if 
\begin{enumerate}
\item $F$ is right continuous
\item $F$ is increasing
\item $F(-\infty)  = 0$ and $F(\infty)  = 1$
\end{enumerate}
\end{defn}

\begin{defn}
Let $P $ be a probability measure on $(\R, \MB(\R))$. We define $F_P: \R \rightarrow \R$, by $$F_P(x) = P((-\infty, x])$$ We call $F_P$ the \textbf{probability distribution function of $P$}.
\end{defn}

\begin{ex}
Let $(\Om, \MF, P)$ be a probability measure. Then $F_P$ is a probability distribution function.
\end{ex}

\begin{proof}
\begin{enumerate}
\item Let $x \in \R$ and $(x_n)_{n \in \N} \subset [x, \infty)$. Suppose that $x_n \rightarrow x$. Then $(x, x_n] \rightarrow \varnothing$ because $\limsup\limits_{n \rightarrow \infty} (x,x_n] = \varnothing$. Thus $$F(x_n) - F(x) = P((x, x_n]) \rightarrow P(\varnothing) = 0$$This implies that $$F(x_n) \rightarrow F(x)$$. So $F$ is right continuous.
\item Clearly $F_P$ is increasing.
\item Continuity from below tells us that $$F(-\infty) = \lim_{n \rightarrow -\infty}F(n) = \lim_{n \rightarrow -\infty}P((-\infty,n]) = 0$$ and continuity from above tell us that $$F(\infty)  = \lim_{n \rightarrow \infty}F(n) = \lim_{n \rightarrow \infty}P((-\infty, n]) = 1$$ 
\end{enumerate}
\end{proof}

\begin{ex}
Let $\mu, \nu$ be probability measures on $(\R, \MB(\R))$. Then $F_{\mu} = F_{\nu}$ iff $\mu = \nu$.  
\end{ex}

\begin{proof}
Clearly if $\mu = \nu$, then $F_{\mu} = F_{\nu}$. Conversely, suppose that $F_{\mu} = F_{\nu}$. Then for each $x \in \R$, 
\begin{align*}
\mu((-\infty,x]) 
&= F_{\mu}(x) \\
&= F_{\nu}(x)  \\
&= \nu((-\infty,x])
\end{align*}
Put $\MC = \{(-\infty,x]:x \in \R\} $. Then $\MC$ is a $\pi$-system and for each $A \in \MC$, $\mu(A) = \nu(A)$. Hence for each $A \in \sig(C) = \MB(\R)$, $\mu(A) = \nu(A)$. So $\mu = \nu$. 
\end{proof}

\begin{defn}
Let $(\Om, \MF)$ be a measurable space and $X:\Om \rightarrow \R$. Then $X$ is said to be a \textbf{random variable} on $(\Om, \MF)$ if $X$ is $\MF$-$\MB(R)$ measurable. 
\end{defn}

\begin{defn}
Let $(\Om, \MF, P)$ be a probability space and $X$ a random variable on $(\Om,\MF)$. We define the \textbf{probability distribution} of $X$, $P_X:\MB(R) \rightarrow [0,1]$, to be the measure $$P_X = X_*P$$ so that for each $A \in \MB(\R)$, $$P_X(A) = P(X^{-1}(F))$$ \\ We define the \textbf{probability distribution function} of $X$, $F_X:\R \rightarrow [0,1]$, to be $$F_X = F_{P_X}$$
\end{defn}

\begin{defn}
Let $(\Om, \MF, P)$ be a probability space and $X$ a random variable on $(\Om,\MF)$. If $P_X \ll m$, we define the \textbf{probability density} of $X$, $f_X: \R \rightarrow \R$, by $$f_X = \frac{dP_X}{dm}$$ 
\end{defn}

\begin{ex}
Let $(\Om, \MF, P)$ be a probability space and $(X_n)_{n \in \N}$ be a sequence of random variables on $(\Om, \MF)$. Then for each $x \in \R$, $$\P\bigg(\liminf_{n \rightarrow \infty}X_n > x\bigg) \leq \liminf_{n \rightarrow \infty} P(X_n > x)$$  
\end{ex}

\begin{proof}
Let $\om \in \bigg \{ \liminf\limits_{n \rightarrow \infty} X_n > x \bigg \}$. Then $x< \liminf\limits_{n \rightarrow \infty} X_n (\om) = \sup\limits_{n \in \N} \bigg( \inf\limits_{k \geq n} X_k(\om)\bigg)$. So there exists $n^* \in \N$ such that $x< \inf\limits_{k \geq n^*} X_k(\om)$. Then for each $k \in \N$, $k \geq n^*$ implies that $x < X_k(\om)$. So there exists $n^* \in \N$ such that for each $k \in \N$, $k \geq n^*$ implies that $\mathbf{1}_{\{X_k > x\}}(\om) =1$. Hence $\inf\limits_{k \geq n^*} \mathbf{1}_{\{X_k > x\}}(\om)  = 1$. Thus  $\liminf\limits_{n \rightarrow \infty} \mathbf{1}_{\{X_k > x\}}(\om) = \sup\limits_{n \in \N} \bigg( \inf_{k \geq n} \mathbf{1}_{\{X_k > x\}}(\om) \bigg) = 1$. Therefore $\om \in \liminf\limits_{n \rightarrow \infty} \{X_k > x\}$ and we have shown that $$\bigg \{ \liminf\limits_{n \rightarrow \infty} X_n > x \bigg \} \subset \liminf\limits_{n \rightarrow \infty} \{X_k > x\}$$ Then 
\begin{align*}
P \bigg( \liminf\limits_{n \rightarrow \infty} X_n > x \bigg)
& \leq P \bigg( \liminf\limits_{n \rightarrow \infty} \{X_k > x\} \bigg) \\
& \leq \liminf_{n \rightarrow \infty} P(\{X_k > x\})
\end{align*}
\end{proof}


\begin{defn}
Let $(\Om, \MF, P)$ be a probability space and $X \in L^+(\Om) \cup L^1$. Define the \textbf{expectation of X}, $E[X]$, to be $$\E[X] = \int X dP$$.
\end{defn}


\subsection{Independence}

\begin{defn}
Let $(\Om, \MF, P)$ be a probability space and $\MC \subset \MF$. Then $\MC$ is said to be \textbf{independent} if for each $(A_i)_{i=1}^n \subset \MC$, $$P \bigg( \bigcap_{k=1}^nA_k\bigg) = \prod_{k=1}^{n}P(A_k)$$
\end{defn}

\begin{defn}
Let $(\Om, \MF, P)$ be a probability space and $\MC_1, \cdots, \MC_n \subset \MF$. Then $\MC_1, \cdots, \MC_n $ are said to be \textbf{independent} if for each $A_1 \in \MC_1,  \cdots, A_n \in \MC_n$, $A_1, \cdots, A_n$ are independent. 
\end{defn}

\begin{note}
We will explicitely say that for each $i=1, \cdots , n$, $\MC_i$ is independent when talking about the independence of the elements of $\MC_i$ to avoid ambiguity.
\end{note}

\begin{defn}
Let $(\Om, \MF, P)$ be a probability space and $X_1, \cdots, X_2$ random variables on $(\Om, \MF)$. Then $X_1, \cdots, X_n$ are said to be \textbf{independent} if for each $B_1, \cdots, B_n \in \MB(\R)$, $X_1^{-1}B_1, \cdots, X_n^{-1}B_n$ are independent.
\end{defn}

\begin{ex}
Let $(\Om, \MF, P)$ be a probability space and $X_1, \cdots, X_n$ random variables on $(\Om, \MF)$. Then $X_1, \cdots, X_n$ are independent iff $\sig(X_1), \cdots, \sig(X_n)$ are independent.
\end{ex}

\begin{proof}
Suppose that $X_1, \cdots, X_n$ are independent. Let $A_1, \in \sig(X_1), \cdots, A_n \in \sig(A_n)$. Then for each $i = 1, \cdots, n$, there exists $B_i \in \MB(\R)$ such that $A_i = X_i^{-1}(B_i)$. Then $A_1, \cdots, A_n$ are independent. Hence $\sig(X_1), \cdots, \sig(X_n)$ are independent. Conversely, suppose that $\sig(X_1), \cdots, \sig(X_n)$ are independent. Let $B_1, \cdots, B_n \in \MB(\R)$. Then for each $i = 1, \cdots, n$, $X_i^{-1}B_i \in \sig(X_i)$. Then $X_1^{-1}B_1, \cdots, X_n^{-1}B_n$ are independent. Hence $X_1, \cdots, X_n$ are independent.
\end{proof}

\begin{ex}
Let $(\Om, \MF, P)$ be a probability space, $X_1, \cdots, X_n$ random variables on $(\Om, \MF)$ and $\MF_1, \cdots, \MF_n \subset \MF$ a collection of $\sig$-algebras on $\Om$. Suppose that for each $i = 1, \cdots, n$, $X_i$ is $\MF_i$-measurable. If $\MF_1, \cdots, \MF_n$ are independent, then $X_1, \cdots, X_n$ are independent. 
\end{ex}

\begin{proof}
For each $i =1, \cdots, n$, $\sig(X_i) \subset \MF_i$. So $\sig(X_1), \cdots, \sig(X_n)$ are independent. Hence $X_1, \cdots, X_n$ are independent.
\end{proof}

\begin{ex}
Let $(\Om, \MF, P)$ be a probability space and $\MC_1, \cdots, \MC_n \subset \MF$. Suppose that for each $i = 1, \cdots, n$, $\MC_i$ is a $\pi$-system and $\MC_1, \cdots, \MC_n$ are independent, then $\sig(\MC_1), \cdots, \sig(\MC_n)$ are independent.
\end{ex}

\begin{proof}
Let $A_2 \in \MC_2$. Define $\ML = \{A \in \MF: P(A\cap A_2) = P(A)P(A_2)\}$. Then 
\begin{enumerate}
\item $\Om \in \ML$
\item If $A \in \ML$, then 
\begin{align*}
P(A^c \cap A_2) 
&= P(A_2) - P(A_2 \cap A) \\
&= P(A_2) - P(A_2) P(A) \\
&= (1- P(A))P(A_2) \\
&= P(A^c)P(A_2)
\end{align*}
So $A^c \in \ML$
\item If $(B_n)_{n \in \N} \subset \ML$ is disjoint, then 
\begin{align*}
P\bigg( \bigg[\bigcup_{n \in \N}B_n \bigg] \cap A_2\bigg) 
&= P \bigg( \bigcup_{n \in \N}B_n \cap A_2 \bigg) \\
&= \sum_{n \in \N}P(B_n \cap A_2) \\
&= \sum_{n \in \N}P(B_n) P(A_2) \\
&=   \bigg[\sum_{n \in \N}P(B_n)\bigg]P(A_2)  \\
&=  P\bigg( \bigcup_{n \in \N} A_n\bigg) P(A_2) 
\end{align*} 
So $\bigcup\limits_{n \in \N}B_n \in \ML$. 
\end{enumerate}
Thus $\ML$ is a $\lam$-system. Since $\MC_1 \subset \ML$ is a $\pi$-system, Dynkin's theorem tells us that $\sig(\MC_1) \subset \ML$. Since $A_2 \in \MC_2$ is arbitrary $\sig(\MC_1)$ and $\MC_2$ are independent. The same reasoning implies that $\sig(\MC_1)$ and $\sig(\MC_2)$ are independent. Let $A_2 \in \MC_1, \cdots, A_n \in \MC_n$ We may do the same process with $$\ML = \bigg \{A \in \MF: P\bigg(A \cap \bigg(\bigcap_{i=2}^n A_i\bigg) \bigg) = P(A)\prod_{i=2}^n P(A_i)\bigg\}$$ and conclude that $\sig(\MC_1), \MC_2, \cdots, \MC_n$ are independent. Which, using the same reasoning would imply that $\sig(\MC_1), \cdots, \sig(\MC_n)$ are independent.
\end{proof}

\begin{ex}
Let $(\Om, \MF, P)$ be a probability space, $X_1, \cdots, X_n$ random variables on $(\Om, \MF)$. Then $X_1, \cdots, X_n$ are independent iff for each $x_1, \cdots, x_n \in \R$, $$P(X_1 \leq x_1, \cdots, X_n \leq x_n ) = \prod_{i=1}^nP(X_i \leq x_i)$$
\end{ex}

\begin{proof}
Suppose that $X_1, \cdots, X_n$ are independent. Then $\sig(X_1), \cdots, \sig(X_n)$ are independent. Let $x_1, \cdots, x_n \in \R$. Then for each $i=1, \cdots, n$, $\{X_i \leq x_i\} \in \sigma(X_i)$. Hence \\$P(X_1 \leq x_1, \cdots, X_n \leq x_n ) = \prod\limits_{i=1}^nP(X_i \leq x_i)$. Conversely, suppose that for each \\$x_1, \cdots, x_n \in \R$, $P(X_1 \leq x_1, \cdots, X_n \leq x_n ) = \prod\limits_{i=1}^nP(X_i \leq x_i)$. Define $\MC = \{ (-\infty, x]: x \in \R \}$. Then $\MB(\R) = \sig(\MC)$. For each $i =1, \cdots, n$, define $\MC_i = X_i^{-1}\MC$. Then for each $i =1, \cdots, n$, $\MC_i$ is a $\pi$-system and 
\begin{align*}
\sig(\MC_i) 
&= \sig(X^{-1}(\MC)) \\
&= X_i^{-1}(\sig(\MC)) \\
&= X_i^{-1}(\MB(\R)) \\
&= \sig(X_i)
\end{align*}
By assumption, $\MC_1, \cdots, \MC_n$ are independent. The previous exercies tells us that $\sig(X_1), \cdots, \sig(X_n)$ are independent. Then $X_1, \cdots, X_n$ are independent. 
\end{proof}

\begin{ex}
Let Let $(\Om, \MF, P)$ be a probability space and $X_1, \cdots, X_n$ random variables on $(\Om, \MF)$. Define $X = (X_1, \cdots, X_n)$. If $X_1, \cdots, X_n$ are independent, then $$P_{X} = \prod\limits_{i=1}^nP_{X_i}$$.
\end{ex}

\begin{proof}
Let $A_1, \cdots, A_n \in \MB(\R)$. Then 
\begin{align*}
P_X(A_1 \times \cdots \times A_n) 
&= P(X \in A_1 \times \cdots \times \in A_n)\\
&= P(X_1 \in A_1, \cdots, X_n \in A_n) \\
&= P(X_1 \in A_1) \cdots P(X_n \in A_n) \\
&= P_{X_1}(A_1) \cdots P_{X_n}(A_n) \\
&= \prod_{i=1}^nP_{X_i}(A_1 \times \cdots \times  A_n)
\end{align*}
Put $$\MP = \{ A_1 \times \cdots \times A_n: A_1 \in \MB(R), \cdots, A_n \in \MB(R) \}$$ Then $\MP$ is a $\pi$-system and $$\sig(\MP) = \MB(R) \otimes \cdots \otimes \MB(R) = \MB(\R^n)$$
A previous exercise then tells us that $P_X = \prod\limits_{i=1}^nP_{X_i}$ 
\end{proof}

\begin{ex}
Let Let $(\Om, \MF, P)$ be a probability space, $X_1, \cdots, X_n$ random variables on $(\Om, \MF)$ and $f_1, \cdots, f_n: \R \rightarrow \R \in L^0$. Suppose that $f_1 \circ X_1, \cdots, f_n \circ X_n\in L^+(\Om)$ or $f_1 \circ X_1, \cdots, f_n \circ X_n \in L^1(\Om)$. If $X_1, \cdots, X_n$ are independent, then $$E[f_1(X_1) \cdots f_n(X_n)] = \prod_{i=1}^n E[f_i(X_i)]$$
\end{ex}

\begin{proof}
Define the random vector $X : \Om \rightarrow \R^n$ by $X = (X_1, \cdots, X_n)$ and $g:\R^n \rightarrow \R$ by $g(x_1, \cdots, x_n) = f_1(x_1) \cdots f_n(x_n)$. Suppose that for each $i = 1, \cdots, n$, $f_i \in L^+(\R)$. Then $g \in L^+(\R^n)$ and by change of variables,  
\begin{align*}
E[f_1(X_1) \cdots f_n(X_n)] 
&= E[g(X)] \\
&= \int_{\Om} g \circ X dP \\
&= \int_{\R^n} g(x) dP_X(x) \\
&= \int_{R^n} g(x) d \prod\limits_{i=1}^nP_{X_i}(x) \\
&=  \prod_{i=1}^n \int_{\R}f_i(x) d P_{X_i}(x) \\ 
&= \prod_{i=1}^n \int_{\Om}f_i \circ X d P \\
&= \prod_{i=1}^n E[f_i(X_i)]
\end{align*}
If for each $i = 1, \cdots, n$, $f_i \in L^1(\R, P_{X_i})$, then following the above reasoning with $\vert g \vert $ tells us that $g \in L^1(\R^n, P_X)$ and we use change of variables and Fubini's theorem to get the same result.  
\end{proof}

\subsection{$L^p$ Spaces for Probability}

\begin{note}
Recall that for a probability space $(\Om, \MF, P)$ and $1 \leq p \leq q \leq \infty$ we have $L^q \subset L^p$ and for each $X \in L^q, \n X \n_p \leq  \n X\n_q$. Also recall that for $X,Y \in \L^2$, we have that $\n XY \n_1 \leq \n X\n_2 \n X\n_2$.
\end{note}

\begin{defn}
Let $(\Om, \MF, P)$ be a probability space and $X \in L^2$. Define the \textbf{variance of X}, $Var(X)$, to be $$Var(X) = \E\big[\big(X-E[X]\big)^2\big]$$.
\end{defn}

\begin{defn}
Let $(\Om, \MF, P)$ be a probability space and $X,Y \in L^2$. Define the  
\end{defn}


\begin{defn}
Let $(\Om, \MF, P)$ be a probability space and $X,Y \in L^2$. Define the \textbf{covariance of $X$ and $Y$}, $Cov(X,Y)$, to be $$Cov(X,Y) = E[(X-E[X])(Y-E[Y])]$$
\end{defn}

\begin{ex}
Let $(\Om, \MF, P)$ be a probability space and $X,Y \in L^2$. Then the covariance is well defined and $Cov(X,Y) ^2 \leq Var(X)Var(Y)$

\begin{proof}
By Holder's inequality, 
\begin{align*}
\vert Cov(X,Y) \vert 
&= \bigg \vert \int (X-E[X])(Y-E[Y]) dP \bigg \vert \\
&\leq \int \vert (X-E[X])(Y-E[Y]) \vert dP \\
&= \n (X-E[X])(Y-E[Y]) \n_1 \\
&\leq \n X-E[X] \n_2 \n (Y-E[Y] \n_2  \\
&= \bigg(\int \vert X - E[X] \vert^2 dP\bigg)^{\frac{1}{2}} \bigg( \vert Y - E[Y] \vert^2 \bigg)^{\frac{1}{2}} \\
&= Var(X)^{\frac{1}{2}}Var(Y)^{\frac{1}{2}}
\end{align*}

So $Cov(X,Y) ^2 \leq Var(X)Var(Y)$.
\end{proof}

\end{ex}

\begin{ex}
Let $(\Om, \MF, P)$ be a measure space and $X,Y \in L^2$. Then
\begin{enumerate}
\item $Cov(X,Y) = E[XY]-E[X]E[Y]$
\item If $X,Y$ are independent, then $Cov(X,Y) = 0$
\item $Var(X) = E[X^2] - E[X]^2$
\item for each $a,b \in \R$, $Var(aX + b) = a^2Var(X)$.
\item $Var(X+Y) = Var(X) + Var(Y) + 2 Cov(X,Y)$
\end{enumerate}
\end{ex}

\begin{proof}\
\begin{enumerate}
\item We have that
\begin{align*}
Cov(X,Y) 
&= E[(X-E[X])(Y-E[Y])] \\
&= E[XY -E[Y]X - E[X]Y + E[X]E[Y]] \\
&= E[XY] - E[X]E[Y] - E[X]E[Y] + E[X]E[Y] \\ 
&= E[XY] - E[X]E[Y]
\end{align*}
\item Suppose that $X,Y$ are independent. Then $E[XY] = E[X]E[Y]$. Hence 
\begin{align*}
Cov(X,Y) 
&= E[XY] - E[X]E[Y] \\
&= E[X]E[Y] - E[X]E[Y] \\
&= 0
\end{align*}
\item Part (1) implies that 
\begin{align*}
Var(X) 
&= Cov(X, X) \\
&= E[X^2] -E[X]^2
\end{align*}
\item Let $a,b \in \R$. Then
\begin{align*}
Var(aX+b)
&= E[(aX+b)^2] - E[aX+b]^2 \\
&= E[a^2X^2 + 2abX +b^2] - (aE[X]+b)^2 \\
&= a^2E[X^2]+2abE[X] + b^2 - (a^2E[X]^2 +2abE[X]+b^2) \\
&= a^2(E[X^2]-E[X]^2) \\
&= a^2Var(X)
\end{align*} 
\item We have that 
\begin{align*}
Var(X+Y) 
&= E[(X+Y)^2] - E[X+Y]^2 \\
&= E[X^2 +2XY + Y^2] -(E[X]+E[Y])^2 \\
&= E[X^2] + 2E[XY] + E[Y^2] - (E[X]^2 + 2E[X]E[Y] + E[Y]^2) \\
&= (E[X^2] - E[X]^2) + (E[Y^2]- E[Y]^2) + 2(E[XY] - E[X]E[Y]) \\
&= Var(X) + Var(Y) + 2Cov(X,Y)
\end{align*}
\end{enumerate}
\end{proof}

\begin{defn}
Let $(\Om, \MF, P)$ be a probability space and $X,Y \in L^2$. The \textbf{correlation of X and Y}, $Cor(X,Y)$, is defined to be $$Cor(X,Y) = \frac{Cov(X,Y)}{\sqrt{Var(X)Var(Y)}}$$
\end{defn}

\begin{ex}

\end{ex}

\begin{ex}{Jensen's Inequality}
Let $(\Om, \MF, P)$ be a probability space, $X \in L^1$ and $\phi:\R \rightarrow \R$. If $\phi$ is convex, then $$\phi(E[X]) \leq E[\phi(X)]$$
\end{ex}

\begin{proof}
Put $x_0 = E[X]$. Since $\phi$ is convex, there exist $a,b \in \R$ such that $\phi(x_0) = ax_0+b$ and for each $x \in \R$, $\phi(x) \geq ax+b$. Then \begin{align*}
E[\phi(X)] 
&= \int\phi(X) dP \\
&\geq \int[ aX+b ]dP \\
&= a\int X dP +b \\
&= aE[X] +b \\
&= ax_0+b \\
&= \phi(x_0) \\
&= \phi(E[X])
\end{align*}
\end{proof}

\begin{ex}{Markov's Inequality:}
Let $(\Om, \MF, P)$ be a probability space and $X \in L^+$. Then for each $a \in (0,\infty)$, $$P(X \geq a) \leq \frac{E[X]}{a}$$
\end{ex}

\begin{proof}
Let $a \in (0,\infty)$. Then $a \mathbf{1}_{\{X \geq a\}} \leq X \mathbf{1}_{\{X \geq a\}}$. Thus 
\begin{align*}
a P(X \geq a) 
&= \int a \mathbf{1}_{\{X \geq a\}} dP\\
&= \int X \mathbf{1}_{\{X \geq a\}} dP\\
&\leq \int X dP \\
&= E[X]
\end{align*}
Therefore $$P(X \geq a) \leq \frac{E[X]}{a}$$. 
\end{proof}

\begin{ex}{Chebychev's Inequality:}
Let $(\Om, \MF, P)$ be a probability space and $X \in L^2$. Then for each $a \in (0, \infty)$, $$P(\vert X - E[X] \vert \geq a) \leq \frac{Var(X)}{a^2}$$
\end{ex}

\begin{proof}
Let $a \in (0, \infty)$. Then 
\begin{align*}
P(\vert X - E[X] \vert \geq a) 
&= P((X - E[X])^2 \geq a^2) \\
&\leq \frac{E[(X - E[X])^2]}{a^2} \\
&= \frac{Var(X)}{a^2}  
\end{align*}
\end{proof}

\begin{ex}{Chernoff's Bound:}
Let $(\Om, \MF, P)$ be a probability space and $X \in L^2$. Then for each $a, t \in (0, \infty)$, $$P( X \geq a) \leq e^{-ta}E[e^{tX}]$$
\end{ex}

\begin{proof}
Let $a, t \in (0, \infty)$. Then 
\begin{align*}
P( X \geq a)
&= P(tX \geq ta) \\
&= P(e^{tX} \geq e^{ta}) \\
& \leq e^{-ta}E[e^{tX}]
\end{align*}
\end{proof}

\begin{ex}{Weak Law of Large Numbers:}
Let $(\Om, \MF, P)$ be a probability space $(X_i)_{i\in \N} \subset L^2$. Suppose that $(X_i)_{i\in \N}$ are iid. Then $$\frac{1}{n}\sum_{i=1}^n X_i \conv{P} E[X_1]$$
\end{ex}

\begin{proof}
Put $\mu = E[X_1]$ and $\sig^2 = Var(X_1)$. Then 
\begin{align*}
E[\frac{1}{n}\sum_{i=1}^n X_i] 
&= \frac{1}{n} \sum_{i=1}^nE[X_i] \\
&= \frac{1}{n} \sum_{i=1}^n \mu \\
&= \mu
\end{align*} and 
\begin{align*}
Var(\frac{1}{n}\sum_{i=1}^n X_i) 
&= \frac{1}{n^2} Var(\sum_{i=1}^n X_i) \\
&= \frac{1}{n^2} \sum_{i=1}^n Var(X_i) \\
&= \frac{1}{n^2} \sum_{i=1}^n \sig^2 \\
&= \frac{\sig^2}{n}
\end{align*}
Let $\ep >0$. Then 
\begin{align*}
P\bigg(\bigg \vert \frac{1}{n} \sum_{i=1}^n X_i - E[X_1] \bigg \vert \geq \ep \bigg) 
& = P\bigg(\bigg\vert \frac{1}{n} \sum_{i=1}^n X_i - \mu \bigg\vert \geq \ep\bigg) \\
& = P\bigg(\bigg\vert \frac{1}{n} \sum_{i=1}^n X_i - \E\bigg[\frac{1}{n}\sum_{i=1}^n X_i \bigg] \bigg\vert \geq \ep\bigg) \\ 
&\leq \frac{Var(\frac{1}{n} \sum_{i=1}^n X_i)}{\ep^2} \\
& =  \frac{\sig^2 / n}{\ep^2} \\
&= \frac{\sig^2}{n\ep^2} \rightarrow 0
\end{align*}

So $$\frac{1}{n}\sum_{i=1}^n X_i \conv{P} E[X_1]$$
\end{proof}

\subsection{Borel Cantelli Lemma} 

\begin{defn}
Let $(\Om, \MF, \P)$ be a probability space and $(A_n)_{n \in \N} \subset \MF$. We will define $$P(A_n \io) \coloneqq P(\limsup_{n \rightarrow \infty}A_n) $$ and $$P(A_n \ev) \coloneqq P(\liminf_{n \rightarrow \infty} A_n)$$ to be the \textbf{probability that $A_n$ happens infinitely often} and the \textbf{probability that $A_n$ happens eventually} respectively.  
\end{defn}

\begin{ex}\textbf{Borel Cantelli Lemma:}
Let $(\Om, \MF, \P)$ be a probability space and $(A_n)_{n \in \N} \subset \MF$.
\begin{enumerate}
\item If $\sum\limits_{n \in \N}P(A_n) < \infty$, then $P(A_n \io) = 0$.
\item If $(A_n)_{n \in \N}$ are independent and $\sum\limits_{n \in \N} P(A_n) = \infty$, then $P(A_n \io) = 1$  
\end{enumerate}
\end{ex}

\begin{proof}\
\begin{enumerate}
\item Suppose that $\sum\limits_{n \in \N}P(A_n) < \infty$. Recall that $$\limsup\limits_{n \rightarrow \infty}A_n = \bigg \{\om \in \Om: \sum\limits_{n \in \N}\mathbf{1}_{A_n}(\om) = \infty \bigg \}$$ Then \begin{align*}
\infty 
&> \sum_{n \in \N}P(A_n) \\
&= \sum_{n \in \N} \int \mathbf{1}_{A_n}dP \\
&= \int \sum_{n \in \N} \mathbf{1}_{A_n}dP \\
\end{align*}
Thus $\sum\limits_{n \in \N} \mathbf{1}_{A_n} < \infty$ a.e. and $P(A_n \io) = 0$.
\item Suppose that $(A_n)_{n \in \N}$ are independent and $\sum\limits_{n \in \N} P(A_n) = \infty$.
\end{enumerate}
\end{proof}

\begin{ex}
Let $(\Om, \MF, \P)$ be a probability space and $(X_n)_{n \in \N} \subset L^0$ and $X \in L^0$. 
\begin{enumerate}
\item If there exists $\ep >0 $ such that $\sum\limits_{n \in \N} P(|X_n -X| > \ep) < \infty$, then $X_n \rightarrow X$ a.s.
\item If $(X_n)_{n \in \N}$ are independent and there exists $\ep >0$ such that $\sum\limits_{n \in \N} P(|X_n -X| > \ep) = \infty$, then $X_n \not \rightarrow X$ a.s.
\end{enumerate}
\end{ex}

\begin{proof}
\begin{enumerate}
\item 
\end{enumerate}
\end{proof}







\newpage

\section{Appendix}

\subsection{Summation}

\begin{defn}
Let $f:X \rightarrow \Rg$, Then we define $$\sum_{x \in X} f(x) := \sup_{\substack{F \subset X \\ F \text{ finite}}} \sum_{x \in F} f(x)$$ This definition coincides with the usual notion of summation when $X$ is countable. For $f:X \rightarrow \C$, we can write $f = g +ih$ where $g,h:X \rightarrow \R$. If $$\sum_{x \in X}|f(x)| < \infty,$$ then the same is true for $g^+,g^-,h^+,h^-$. In this case, we may define $$\sum_{x \in X} f(x)$$ in the obvious way.
\end{defn} 

The following note justifies the notation $\sum_{x \in X}f(x)$ where $f:X \rightarrow \C$.

\begin{note}
Let $f:X \rightarrow \C$ and $\al:X \rightarrow X$ a bijection. If $\sum_{x \in X}|f(x)|< \infty$, then $\sum_{x \in X}f( \al (x)) = \sum_{x \in X}f(x) $.
\end{note}

\end{document}